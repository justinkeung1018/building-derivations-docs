\section{Concrete terms}
\label{section:term}
In this report, a \textit{concrete term} refers to an instance or realisation of a syntax rule in the derivation tree. A concrete term can correspond to a Curry type, a $\lambda$-term, a multiset of propositional formulae, or any syntax rule defined by the user. For example, $((1 \to 2) \to 3)$ is a valid concrete term according to the rule $A \Coloneqq 1 \alt 2 \alt 3 \alt (A \to A)$, since it is a valid instance of the rule.

A \textit{concrete statement} is an instance or realisation of the syntax rule defining a statement.

A valid derivation tree consists of valid concrete statements and rule names. \projectname{} verifies whether each concrete statement conforms to the syntax rule defining a statement. This section describes the parsing algorithm that achieves this goal.

\subsection{Parsing concrete terms}
The aim of this step is to convert a term, given as a string, into a list of abstract syntax trees (ASTs). After parsing a syntax rule into \lstinline{Token}s, we generate a sub-parser for each \lstinline{Token} and combine the sub-parsers sequentially to form an overall parser. Each sub-parser outputs an AST, and the overall parser collects the ASTs from all sub-parsers into a list. The first AST is produced by the sub-parser corresponding to the first \lstinline{Token}, and so on.

\subsubsection{Why a list of ASTs, but not a hierarchical structure?}
% \projectname{} uses a flat list of ASTs to represent concrete terms because there is no notion of operator or symbol precedence. Consider the concrete term $(1 \to 2)$ and the corresponding syntax rule $A \Coloneqq 1 \alt 2 \alt (A \to A)$. If one were to manually write a parser, it may convert the concrete term into an AST node with the following structure:
% \begin{center}
%     \begin{forest}
%         [$\to$
%             [1]
%             [2]
%         ]
%     \end{forest}
% \end{center}
% Here, we assume $\to$ should be at the root level because it is the main connective. In fact, it is the only connective. However, it is difficult to generalise the notion of a main connective to arbitrary syntax rules.
Consider the following syntax rule:
\[
    \text{Statement} \Coloneqq M: A \to A
\]
where $M$ represents a $\lambda$-term and $A$ represents a Curry type. The concrete term $x: 1 \to 2$ can be represented as a tree in several ways:
\begin{center}
    \begin{adjustbox}{valign=m}
        \begin{forest}
            [:
                [$x$]
                [$\to$
                    [1]
                    [2]
                ]
            ]
        \end{forest}
    \end{adjustbox}
    \tquad
    or
    \tquad
    \begin{adjustbox}{valign=m}
        \begin{forest}
            [$\to$
                [:
                    [$x$]
                    [1]
                ]
                [2]
            ]
        \end{forest}
    \end{adjustbox}
    \tquad
    or
    \tquad
    \begin{adjustbox}{valign=m}
        \begin{forest}
            [$:\ \to$
                [$x$]
                [1]
                [2]
            ]
        \end{forest}
    \end{adjustbox}
\end{center}
The first representation is favoured when : has a lower precedence than $\to$, i.e. it binds less strongly than $\to$. The second representation is favoured when : has a higher precedence than $\to$. The third representation is favoured when : and $\to$ have equal precedence.

\projectname{} assumes all symbols in a definition have equal precedence, i.e. it assumes the third representation. Any operator precedence should be reflected in the overall syntax definition. If the user wants : to have a lower precedence than $\to$ (i.e. they want the first representation), they should define the syntax rules as follows:
\begin{align*}
    \text{Statement} &\Coloneqq M: \nonterm{arrow} \\
    \nonterm{arrow} &\Coloneqq A \to A
\end{align*}
Even assuming equal precedence, the third representation cannot accurately represent the term as is. It needs to store the positions of $x$, 1, and 2 relative to the symbols : and $\to$. Otherwise, consider the same tree and the modified syntax rule
\[
    \text{Statement} \Coloneqq M: A \to A \alt M\ A\ A :\ \to
\]
Without knowing the positions of $x$, 1, and 2 relative to the symbols, there is no way to determine whether the tree represents $x: 1 \to 2$ or $x\ 1\ 2:\ \to$, yet the syntax rule is clearly unambiguous.

While the positions can be stored in the children nodes, the third representation offers no benefit over a flat list, which stores the positions and solves the problem automatically.

\subsubsection{Abstract syntax trees (ASTs)}
The ASTs are defined as follows:
\begin{lstlisting}
    class TerminalAST { constructor(readonly value: string) {} }

    class NonTerminalAST {
        constructor(readonly index: number, readonly children: AST[]) {} }

    class MultisetAST { constructor(readonly elements: AST[][]) {} }

    type AST = TerminalAST | NonTerminalAST | MultisetAST;
\end{lstlisting}

\subsubsection{Procedure}
\label{term:procedure}
A list of \textit{delayed parsers} is generated using the \lstinline{later} combinator. A \textit{delayed parser}\footnote{In \lstinline{parjs}, a delayed parser has type \lstinline{DelayedParjser<T>}.} is a parser that does not carry any logic and must be initialised before being used for parsing \cite{parjs}. A delayed parser can be chained and combined with other parsers like a normal parser. There are two main benefits of generating delayed parsers and later initialising them:
\begin{itemize}
    \item It is possible to generate parsers for recursive rules. When generating the parser for a rule containing references to itself, the parser must be incomplete at any point of reference since the reference is part of the definition. A similar logic applies to mutually recursive rules.
    \item There is no need to devise an order in which the parsers for each rule is generated (e.g. by topologically sorting the references, such that every rule only references ``earlier'' rules whose parsers would have been all initialised). In fact, such an order or topological sort does not exist when there is recursion, since the reference graph contains cycles.
\end{itemize}
A term parser is generated by iterating over the tokens in every alternative of the rule and examining the type of the token:
\begin{enumerate}
    \item \lstinline{Terminal}: parse the string verbatim.
    \item \lstinline{NonTerminal}: use the (delayed) parser corresponding to the rule number of the \lstinline{NonTerminal}.
    \item \lstinline{Multiset}: either parse the string \lstinline{\varnothing} (corresponding to the empty set) or a comma-separated list of one or more elements. To generate the parser for a multiset element, a parser is generated for each token of the element definition. Each of these parsers are chained together using the \lstinline{then} combinator.
\end{enumerate}