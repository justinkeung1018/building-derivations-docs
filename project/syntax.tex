\section{Syntax rules}
\label{section:syntax}
Syntax rules define how well-formed input in a proof system should look like. Input that does not adhere to the syntax rules is considered nonsense and rejected by the system.

Users may want to extend pre-existing proof systems with new syntactic constructs, or even define a new proof system from scratch. \projectname{} provides an interface for modifying syntax rules, checks whether the syntax rules are defined sensibly, and verifies user input against an arbitrary set of syntax rules.

\Cref{syntax:syntax} defines the syntax in which a user expresses a syntax rule in \projectname{}. \Cref{syntax:parsing} describes the procedure for transforming an arbitrary set of syntax rules into intermediate internal representations, in preparation for verifying user input against the syntax rules.

\subsection{Syntax of syntax rules}
\label{syntax:syntax}
A syntax rule consists of:
\begin{itemize}
    \item a list of comma-separated \textbf{metavariables}, and
    \item a \textbf{definition}, consisting of one or more non-empty alternatives separated by vertical bars \lstinline{|}
\end{itemize}
In rule definitions, curly braces \lstinline|{}| represent a multiset. The structure of each element of a multiset is defined by the content enclosed within the curly braces. For example, the definition \lstinline|{ M: A }| represents a possibly empty collection of possibly duplicate elements following the structure \lstinline{M: A}.

Every valid syntax definition (i.e. collection of syntax rules) must contain at least one rule for a statement. All non-statement rules must specify metavariables. Otherwise, there is no way to refer to the definition of such rules and they might as well be removed from the syntax definition. Every syntax rule, including the rule defining a statement, can contain one or more non-empty alternatives.

The only special character in the definition of metavariables is the comma. The special characters in the definition of a syntax rule are the vertical bar \lstinline{|} and the curly braces \lstinline|{}|. Unlike in extended Backus-Naur form, characters like parentheses \lstinline{()} and brackets \lstinline{[]} do not carry any special meaning anywhere in this system and are no different from any other ordinary character. Certain groups of characters, such as \lstinline{|-} and \lstinline{->}, are treated as one symbol and shorthands for \LaTeX{} commands, such as \lstinline{\vdash} and \lstinline{\rightarrow}.

A rule definition can refer to itself or other rules by their metavariables. In the definition of syntax rules, there is no difference using one metavariable over another to refer to the same rule. For example, given the metavariables \lstinline{A} and \lstinline{B} for a rule, the definitions \lstinline{(A -> A)}, \lstinline{(B -> B)}, \lstinline{(A -> B)}, and \lstinline{(B -> A)} are all identical. However, it is useful to have multiple metavariables because in inference rules, metavariables are treated as names representing concrete objects.

\subsection{Parsing syntax rules}
\label{syntax:parsing}
The aim of this step is to generate a list of \lstinline{Token}s for each rule alternative, and thus a two-dimensional array of \lstinline{Token}s for each rule definition. These \lstinline{Token}s will be used to generate term parsers, i.e. parsers for parsing instances of each rule.

\subsubsection{Tokens}
The tokens are defined as follows:
\begin{itemize}
    \item \begin{lstlisting}[style=ds]
        class Terminal { constructor(readonly value: string) {} }
    \end{lstlisting}
    A \lstinline{Terminal} represents a string.
    \item \begin{lstlisting}[style=ds]
        class NonTerminal { constructor(readonly index: number) {} }
    \end{lstlisting}
    The \lstinline{index} field of \lstinline{NonTerminal} is the position of the rule referenced by the \lstinline{NonTerminal} in the ordered list of syntax rules. For example, if a syntax definition contains two rules
    \begin{align*}
        S &\Coloneqq \ldots \\
        A, B &\Coloneqq \ldots
    \end{align*}
    Then any occurrences of the metavariable $S$ are represented by a \lstinline{NonTerminal} with \lstinline{index} 0 and any occurrences of either of the metavariables $A$ or $B$ are represented by a \lstinline{NonTerminal} with \lstinline{index} 1.
    \item \begin{lstlisting}[style=ds]
        class Multiset { constructor(readonly tokens: Token[]) {} }
    \end{lstlisting}
    The \lstinline{tokens} field of \lstinline{Multiset} defines the structure of an element of the multiset. For example, consider a rule definition that contains a multiset:
    \begin{align*}
        S &\Coloneqq \{ var: A \}
    \end{align*}
    Assuming $var$ and $A$ are metavariables, the rule definition is represented by a \lstinline{Multiset} with the \lstinline{tokens} field set to \lstinline{[NonTerminal(...), Terminal(":"), NonTerminal(...)]}.
\end{itemize}
Finally, the \lstinline{Token} type is defined as
\begin{center}
    \lstinline*type Token = Terminal | NonTerminal | Multiset;*
\end{center}

\subsubsection{Procedure}
The process consists of three steps: splitting the metavariables and alternatives of the rule definitions, normalising them, and generating a parser for a rule alternative.

\paragraph{Splitting metavariables and alternatives of rule definitions}
When the backend receives user input from the frontend, every rule is specified by a string representing the metavariables and a string representing the definition. The metavariable string is split on commas to give a list of metavariables. To preprocess the definition string, all occurrences of \lstinline{|-} are replaced with \lstinline{\vdash}, the \LaTeX{} command for a turnstile, then the string is split on vertical bars \lstinline{|} to give a list of alternatives. The replacement is necessary and correct as \lstinline{|-} contains a vertical bar \lstinline{|} while \lstinline{\vdash} does not.

\paragraph{Normalise metavariables and rule definitions}
\label{syntax:normalise}
Aliases of a symbol are unified. For example, the symbol for an arrow \lstinline{->} can be represented by \lstinline{\rightarrow}, \lstinline{\to}, \lstinline{->}, and the Unicode character →. In this case, all aliases are replaced with \lstinline{->} (other than having the fewest characters and better readability, the choice is entirely arbitrary). Prior to the replacements, a space is added after all substrings that resemble a \LaTeX{} command, i.e. any substring beginning with \lstinline{\} and a string of non-whitespace characters. This ensures we do not accidentally replace parts of a \LaTeX{} command and e.g. turn \lstinline{\toI} into \lstinline{->I}. The user may have intended the latter, but the former is a typo nonetheless.

\paragraph{Generate parser}
The parser outputs three types of tokens: \lstinline{Terminal}, \lstinline{NonTerminal}, and \lstinline{Multiset}. The final parser tries to match a non-terminal, then (if it fails) a multiset, then (if both fail) a terminal. Each of these sub-parsers is generated as follows:
\begin{itemize}
    \item \lstinline{Terminal}: it matches, in descending order of priority, any string that resembles a \LaTeX{} command, any special group of characters e.g. \lstinline{|-} and \lstinline{->}, or any single character.
    \item \lstinline{NonTerminal}: it matches any metavariable across all rules in descending order of length.
    \item \lstinline{Multiset}: it matches a sequence of zero or more \lstinline{Terminal}s or \lstinline{NonTerminal}s enclosed within curly braces \lstinline|{}|. The empty multiset definition \lstinline|{}| is technically valid but not very useful: the user can only supply the empty set or a string of commas.
\end{itemize}

\subsection{Difficulty in left-factoring}
\label{syntax:factorisation}
\Cref{parsing:thenor} introduces the idiomatic and non-idiomatic ways of using the \lstinline{then} and \lstinline{or} combinators of \lstinline{parjs} together. The idiomatic approach requires the syntax definition to be left-factored. This section explains the difficulties of left-factoring syntax definitions, and therefore, why the idiomatic approach is not preferred.

\begin{definition}[First set]
    The \textit{first set} of a set of alternatives is the set containing every leading token of every alternative and the contributions of each of the leading tokens:
    \begin{itemize}
        \item If an alternative begins with a \lstinline{Terminal}, the contribution of the \lstinline{Terminal} to the first set is itself.
        \item If an alternative begins with a \lstinline{NonTerminal}, the contribution of the \lstinline{NonTerminal} to the first set is the first set of the rule corresponding to the \lstinline{NonTerminal}.
        \item If an alternative begins with a \lstinline{Multiset}, the contribution of the \lstinline{Multiset} to the first set is the union of the singleton \{ \textbackslash varnothing \} and the contribution of the first token of the element definition.
    \end{itemize}
\end{definition}

For example, consider the following syntax definition:
\begin{align*}
    \text{Statement} &\Coloneqq \Gamma \\
    \Gamma &\Coloneqq \{ A \} \\
    A &\Coloneqq x \alt y \alt z \alt (A \to A)
\end{align*}
The first set of each of the rules is computed as follows:
\begin{itemize}
    \item The first set of the rule corresponding to $A$ is the set containing $x$, $y$, $z$, and the left parenthesis $($.
    \item The first set of the rule corresponding to $\Gamma$ is the set containing \textbackslash varnothing and the contribution of $A$, i.e. the set containing $x$, $y$, $z$, and the left parenthesis $($.
    \item The first set of the rule corresponding to a statement is the contribution of $\Gamma$, i.e. the set containing \textbackslash varnothing, $x$, $y$, $z$, and the left parenthesis $($.
\end{itemize}

\subsubsection{\texorpdfstring{\lstinline{NonTerminal}}{NonTerminal}s sharing non-disjoint first sets with \texorpdfstring{\lstinline{NonTerminal}}{NonTerminal}s and \mbox{\texorpdfstring{\lstinline{Terminal}}{Terminal}s}}
\label{syntax:factorisation:nonterm}
Consider the following syntax definition:
\begin{align*}
    \text{Statement} &\Coloneqq Ax \alt By \\
    A &\Coloneqq x \alt Cz \alt z \\
    B &\Coloneqq x \alt y \alt w \\
    C &\Coloneqq y \alt w
\end{align*}
In the definition of the rule for a statement, the first set of the alternative $Ax$ is $\{ x, y, w, z \}$ while the first set of the alternative $By$ is $\{ x, y, w \}$. Since the first sets are not disjoint, the rule for a statement is not left-factored. One solution is to substitute the offending \lstinline{NonTerminal}s (in this case, $A$ and $B$) with their definitions and remove the rules for the \lstinline{NonTerminal}s if they are not used elsewhere:
\begin{align*}
    \text{Statement} &\Coloneqq xx \alt Czx \alt zx \alt xy \alt yy \alt wy \\
    C &\Coloneqq y \alt w
\end{align*}
After the substitution, the first set of the alternative $Czx$ is $\{ y, w \}$, which is not disjoint with the first sets of the alternatives $yy$ and $wy$. The offending \lstinline{NonTerminal} (in this case, $C$) can be substituted with its definition again:
\[
    \text{Statement} \Coloneqq xx \alt yzx \alt wzx \alt zx \alt xy \alt yy \alt wy
\]
Since there are no more \lstinline{NonTerminal}s left, left-factoring is straightforward:
\[
    \text{Statement} \Coloneqq x(x|y) \alt y(zx|y) \alt wzx \alt zx
\]
The main disadvantage of this approach is that information about the \lstinline{NonTerminal}s is destroyed by the substitutions. The parser generated from the left-factored definition after substitutions only treats the input as a list of \lstinline{Terminal}s, instead of a \lstinline{NonTerminal} followed by a \lstinline{Terminal}. This complicates the process of matching abstract terms in inference rules.

\subsubsection{\texorpdfstring{\lstinline{Multiset}}{Multiset}s sharing non-disjoint first sets with \texorpdfstring{\lstinline{NonTerminal}}{NonTerminal}s and \texorpdfstring{\lstinline{Terminal}}{Terminal}s}
Consider the following syntax definition:
\begin{align*}
    \text{Statement} &\Coloneqq \Gamma \vdash x \alt A \vdash y \\
    \Gamma &\Coloneqq \{ A \} \\
    A &\Coloneqq x \alt y \alt z \alt (A \to A)
\end{align*}
Humans reading the syntax definition can quickly differentiate the two alternatives of a statement by their last character: the term $x \vdash x$ should be parsed using the alternative $\Gamma \vdash x$ because the last character is $x$, while the term $x \vdash y$ should be parsed using the alternative $A \vdash y$ because the last character is $y$.

However, a parser does not know it should look for these defining characters in general. When the parser parses the term $x \vdash y$ from left to right, the prefix $x \vdash$ matches both alternatives: at this point, should the parser treat $x$ as a multiset with one element, a non-terminal, or an indeterminate form?

One attempt to parse correctly without regards to the the original structure of the definitions is to first rewrite the definition of the offending \lstinline{Multiset}s, then substitute occurrences of the \lstinline{Multiset}s with their rewritten definitions. In this example, the rule corresponding to $\Gamma$ is rewritten as
\[
    \Gamma \Coloneqq \varnothing \alt A \ [\text{optionally followed by 0 or more repetitions of },A]
\]
The offending occurrence of $\Gamma$ in the definition of a statement is substituted with its rewritten definition:
\begin{align*}
    \text{Statement} \Coloneqq \ &\varnothing \vdash x \\
    \alt &A \ [\text{optionally followed by 0 or more repetitions of },A] \vdash x \\
    \alt &A \vdash y
\end{align*}
The substituted definition of a statement can now be left-factored:
\begin{align*}
    \text{Statement} \Coloneqq \ &\varnothing \vdash x \\
    \alt &A \ [\text{optionally followed by 0 or more repetitions of },A] \vdash (x|y)
\end{align*}
This solution has the same drawback as that in \Cref{syntax:factorisation:nonterm}: the structure of $\Gamma$ is destroyed by rewriting and substitution. It is not always possible to determine whether the current part of the input is a \lstinline{NonTerminal} or a \lstinline{Multiset} before parsing the next part of the input (e.g. $x \vdash y$), which complicates attempts to recover the destroyed structure after fully parsing the input.

\subsubsection{\texorpdfstring{\lstinline{Multiset}}{Multiset}s sharing non-disjoint first sets with other \texorpdfstring{\lstinline{Multiset}}{Multiset}s}
Consider the following syntax definition:
\begin{align*}
    \text{Statement} &\Coloneqq \Gamma \vdash x \alt \Delta \vdash y \\
    \Gamma &\Coloneqq \{ Ax \} \\
    \Delta &\Coloneqq \{ By \} \\
    A &\Coloneqq x \alt y \\
    B &\Coloneqq x \alt z
\end{align*}
Similar to the previous case, humans can distinguish between an element in $\Gamma$ and an element in $\Delta$ based on the second character of the element. If the second character is $x$, then the element belongs to $\Gamma$. Otherwise, the element belongs to $\Delta$.

However, if a parser parses the statement $xx \vdash x$ from left to right and first sees the character $x$, it cannot determine whether it is about to parse an element in $\Gamma$ or an element in $\Delta$. Like in the previous case, the parser must keep track of this ambiguity until it can be resolved, which can be arbitrarily late in the input in general.

The solution, once again, is to substitute and expand the offending multiset definitions. The rule for $\Gamma$ becomes
\begin{align*}
    \Gamma \Coloneqq \ &\varnothing  \\
    \alt &xx \ [\text{optionally followed by 0 or more repetitions of },xx] \\
    \alt &yx \ [\text{optionally followed by 0 or more repetitions of },yx]
\end{align*}
and the rule for $\Delta$ becomes
\begin{align*}
    \Gamma \Coloneqq \ &\varnothing  \\
    \alt &xy \ [\text{optionally followed by 0 or more repetitions of },xy] \\
    \alt &zy \ [\text{optionally followed by 0 or more repetitions of },zy]
\end{align*}
The final syntax definition becomes
\begin{align*}
    \text{Statement} \Coloneqq \ &\Sigma \vdash (x|y) \\
    \Sigma \Coloneqq \ &\varnothing \\
    \alt &x (x \ [\text{optionally 0 or more },xx] \alt y \ [\text{optionally 0 or more},xy]) \\
    \alt &yx \ [\text{optionally 0 or more },yx] \\
    \alt &zy \ [\text{optionally 0 or more },zy]
\end{align*}
which destroys the structure of $\Gamma$ and $\Delta$ in the original syntax definition.

\subsection{Limitation: metavariables cannot be used as terminals}
Suppose a user tries to define the syntax of $\lambda$-terms:
\begin{align*}
    M, N &\Coloneqq x \alt (\lambda x. M) \alt (MN) \\
    x &\Coloneqq x \alt y \alt z
\end{align*}
The second rule is necessary as the system does not ``know'' $x$ in the first rule is a variable: if a character is not a metavariable, it is treated as a \lstinline{Terminal}. However, contrary to what the user expects, the $x$ in the first alternative of the second rule is interpreted as a reference to its own rule. Since it is the first token of the alternative, the alternative is considered left-recursive by the parsing algorithm and causes the parsing algorithm to throw an error. The current workaround to this issue is as follows:
\begin{align*}
    M, N &\Coloneqq var \alt (\lambda var. M) \alt (MN) \\
    var &\Coloneqq x \alt y \alt z
\end{align*}
The use of $var$ is unconventional and differs from most textbook formulations. However, there is no good way to tell the parsing algorithm to treat $x$ as a literal string instead of a metavariable depending on where it appears. One solution is to treat lowercase metavariables as literal strings if it appears in the definition of its own rule and metavariables otherwise. However, this solution not only makes the system prejudiced against certain inputs, but is also equally problematic. A frustrated user may find the following definition of Curry types to be interpreted rather unexpectedly:
\[
  a, b \Coloneqq \varphi \alt (a \to b)
\]
Since the $a$ and $b$ in the second alternative are treated as strings instead of metavariables, this definition is actually not self-recursive.