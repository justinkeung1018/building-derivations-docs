\section{Inference rules}
\label{section:inference}
Inference rules define the valid proof steps in a proof system. The application supports inference rules of the form
\[
    \Inf[\text{Rule name}]{\text{Zero or more premises}}{\text{Conclusion}}
\]
which lets us derive the conclusion if all the premises hold. Inference rules can use metavariables to match any concrete term. Inference rules can also destructure concrete terms and reject concrete terms that cannot be destructured according to the inference rule. The premises and conclusion of an inference rule are specialised statements that carry structural information and should be parsed more like concrete terms than syntax rules. These specialised statements are called \textit{abstract statements}, consisting of \textit{abstract terms}, which are either metavariables or destructurings of concrete terms.

Like syntax rules, inference rules may be added, deleted, or replaced by the user. \projectname{} provides an interface for modifying inference rules, checks whether the inference rules are defined sensibly, and verifies whether the user has applied an inference rule correctly.

\Cref{inference:syntax} defines the syntax in which a user expresses an inference rule in the application, describes the concept of \textit{abstract terms}, and contrasts it with \textit{concrete terms} introduced in the previous chapter. \Cref{inference:parsing} describes the procedure for transforming an arbitrary set of inference rules into intermediate internal representations, in preparation for verifying applications of the inference rules in the derivation tree.

\subsection{Syntax of inference rules}
\label{inference:syntax}
An inference rule consists of:
\begin{itemize}
    \item A \textbf{name},
    \item One \textbf{conclusion} following the structure of a statement, and
    \item Zero or more \textbf{premises}, each following the structure of a statement
\end{itemize}
\projectname{} only supports \textit{localised} inference rules: checking whether an inference rule is applied correctly only requires information immediately above and below the step of applying the inference rule, but not any information further up or down the derivation tree.

\subsubsection{Abstract terms and statements}
In this report, an \textit{abstract statement} refers to one of the premises or the conclusion of an inference rule. Abstract statements can contain metavariables, while concrete statements cannot. Consider the following syntax rule:
\[
    A, B \Coloneqq \varphi \alt (A \to B)
\]
In a derivation, the user can only input concrete terms like $\varphi$, $(\varphi \to \varphi)$, and $((\varphi \to \varphi) \to \varphi)$. In an inference rule, however, the user can input abstract terms like $A$, $B$, $(A \to B)$, $((A \to B) \to A)$, and even $((B \to \varphi) \to B)$. An abstract term is either a metavariable of the corresponding rule, or partially expanded versions of its definition. For the syntax rule above, the following grammar specifies the set of abstract terms:
\[
    A_{abs}, B_{abs} \Coloneqq \varphi \alt (A_{abs} \to B_{abs}) \alt \text{``$A$''} \alt \text{``$B$''}
\]

All valid concrete terms are valid abstract terms.

\paragraph{Metavariables in syntax rule definitions vs. abstract statements}
Recall from \Cref{syntax:syntax} that in syntax rule definitions, metavariables for a rule can be used interchangeably. The following syntax rule is semantically identical to the example above:
\[
    A, B \Coloneqq \varphi \alt (A \to A)
\]
Even though $A$ appears twice in different positions of the same alternative, the two concrete terms that take their positions can be different. For example, $(\varphi \to (\varphi \to \varphi))$ is a valid instance of both versions of the syntax rule. However, the abstract term $(A \to A)$ is compatible with a concrete term if and only if the two user-inputted sub-terms that take the positions of the two occurrences of $A$ are identical. In this case, $(\varphi \to (\varphi \to \varphi))$ is incompatible with the abstract term $(A \to A)$ because the sub-terms $\varphi$ and $(\varphi \to \varphi)$ are different. The process of matching abstract terms with concrete terms is explained in detail in \Cref{chapter:checking}.

\paragraph{Multisets in inference rules}
\label{inference:multisets}
Consider the following syntax rules:
\begin{align*}
    \Gamma, \Delta &\Coloneqq \{ A \} \\
    A, B &\Coloneqq \varphi \alt (A \to B)
\end{align*}
On its own, $\Gamma$ represents a complete multiset. Now, consider the term $\Gamma, A$. In this term, $\Gamma$ represents a multiset with one element removed. The removed element is given the name $A$. Since an element can be removed, the user-inputted multiset must contain at least one element: trying to match the concrete term $\varnothing$ against the abstract term $\Gamma, A$ results in an error. Note that $\Gamma$ can be empty. Analogously, it is possible to:
\begin{itemize}
    \item Match multiple multiset elements, such as $\Gamma, A, B$ and $\Gamma, A, A, A$.
    \item Match multiple multisets, such as $\Gamma, \Delta$.
    \item Match multiple multisets and multiset elements, such as $\Gamma, \Delta, A, B$.
\end{itemize}

It is also possible to match multiset elements with specific structures. The following are all valid abstract terms:
\[
    \Gamma, (A \to B) \qquad \Gamma, (A \to (A \to A)) \qquad \Gamma, ((\varphi \to B) \to A)
\]
since $(A \to B)$, $(A \to (A \to A))$, and $((\varphi \to B) \to A)$ are all valid expansions of the definition of the second syntax rule, of which $A$ is a metavariable.

The order of the multiset metavariables and multiset element metavariables does not matter. This means a multiset term is identical to any of its permutations. The following terms are all identical:
\[
    \Gamma, \Delta, A, B \qquad \Gamma, B, \Delta, A \qquad A, B, \Delta, \Gamma \qquad \Delta, B, A, \Gamma \qquad \ldots
\]

Note that a metavariables (in this case, $\Gamma$ or $\Delta$) can represent a multiset if and only if at least one of its alternatives only consists of a multiset and no other tokens. For example, if the first syntax rule in the example above is changed to
\[
    \Gamma, \Delta \Coloneqq \{ A \}: B
\]
instead, things like $\Gamma, A$ and $\Gamma, \Delta, A$ are not valid abstract terms for the modified syntax rule.

\subsection{Parsing inference rules}
\label{inference:parsing}
The aim of this step is to generate parsers for parsing an abstract statement into a list of \lstinline{Matchable} tokens.

\subsubsection{\texorpdfstring{\lstinline{Matchable}}{Matchable} tokens}
\lstinline{Matchable} tokens are to inference rules like ASTs are to terms and \lstinline{Token}s are to syntax rules: they represent the structure of an abstract statement. \lstinline{Matchable} tokens are defined as follows:
\begin{itemize}
    \item \begin{lstlisting}[style=ds]
        class MatchableTerminal { constructor(readonly value: string) {} }
    \end{lstlisting}
    Match the string stored in the \lstinline{value} field.
    \item \begin{lstlisting}[style=ds]
        class Name {
            constructor(readonly index: number, readonly name: string) {}
        }
    \end{lstlisting}
    Assign exactly one AST to \lstinline{name} and ensure this assignment is compatible with all other occurrences of the same \lstinline{name} in other statements of the same inference rule. The AST must represent an instance of the syntax rule at \lstinline{index}\footnote{The \lstinline{index} field can be deduced from the \lstinline{name} field since the metavariables are unique across all syntax rules. The \lstinline{index} field exists merely to simplify bookkeeping.}.
    \item \begin{lstlisting}[style=ds]
        class MatchableNonTerminal {
            constructor(readonly index: number, readonly children: Matchable[]) {}
        }
    \end{lstlisting}
    Match an abstract term corresponding to the syntax rule at \lstinline{index}. The structure of the abstract term is given by the \lstinline{children} field.
    \item \begin{lstlisting}[style=ds]
        class MultisetElement { constructor(readonly tokens: Matchable[]) {} }

        class MatchableMultiset {
            constructor(
                readonly index: number,
                readonly elements: (Name | MultisetElement)[]
            ) {}
        }
    \end{lstlisting}
    Match a multiset of elements given by the \lstinline{elements} field. A member of \lstinline{elements} can either be a \lstinline{Name} or a \lstinline{MultisetElement}. A \lstinline{Name} refers to a multiset (e.g. $\Gamma$ and $\Delta$ in the example in the previous section) while a \lstinline{MultisetElement} refers to a multiset element that must be matched from the user-inputted multiset term (e.g. $A$ and $(A \to B)$ in the previous section). The \lstinline{elements} field is represented as an (ordered) array instead of an unordered set purely for simpler bookkeeping in the matching algorithm.
\end{itemize}
Finally, the \lstinline{Matchable} type is defined as
\begin{center}
    \lstinline{type Matchable = MatchableTerminal | Name | MatchableNonTerminal | MatchableMultiset;}
\end{center}
\subsubsection{Procedure}
Parsing abstract statements is extremely similar to parsing terms, as an abstract statement can be thought of as an instance of a statement in a different context. As with generating term parsers, a delayed parser is created  for every syntax rule using the \lstinline{later} combinator. Each of these delayed parsers is instantiated by iterating over the tokens in every alternative of the syntax rule and examining the type of the token:
\begin{itemize}
    \item \lstinline{Terminal}: parse the string verbatim and wrap the result in a \lstinline{MatchableTerminal} token.
    \item \lstinline{NonTerminal}: either parse a metavariable of the rule corresponding to the rule number (i.e. the \lstinline{index} field of the \lstinline{NonTerminal}), or an expanded version of the definition of the syntax rule using the (delayed) parser corresponding to the rule. The former produces a \lstinline{Name} token while the latter produces a \lstinline{MatchableNonTerminal} token.
    \item \lstinline{Multiset}: either parse \lstinline{\varnothing} (corresponding to the empty set) or a comma-separated list of multiset elements, as described in \Cref{inference:multisets}. If metavariables for multisets are applicable here (i.e. the \lstinline{Multiset} token is the only token of an alternative of the syntax rule definition, and the syntax rule does not define a statement), the comma-separated list can contain both individual multiset elements and metavariables for multisets. Since the \lstinline{Multiset} token does not store the rule number of the rule it is part of, the rule number is passed as an argument to the function which generates inference rule parsers.
\end{itemize}