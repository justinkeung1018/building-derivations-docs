\documentclass[a4paper, twoside, 11pt]{report}
\usepackage{derivation-nops}
\usepackage{main}

\title{Building Derivations}
\author{Justin Keung}

\begin{document}
\input{title/title.tex}

\begin{abstract}
Proof assistants are tools that help users create and verify formal proofs. They have been used to help write reliable software, verify security protocols, and prove mathematical theorems.

More recently, people have created proof assistants to teach logic. They are designed to be more intuitive for beginners than industrial-grade proof assistants, at the cost of expressive power and proving ability. However, these educational proof assistants  support their own syntax and cannot let users easily modify the proof system in which the derivations are built.

\projectname{} addresses both of these issues. It is a web-based proof assistant which supports \LaTeX{} syntax and lets users modify the proof system through its web interface on the fly. Users can build derivation trees and arbitrarily modify the syntax and inference rules of the proof system. \projectname{} verifies the given derivation based on the proof system defined by the user.

\end{abstract}

\renewcommand{\abstractname}{Acknowledgements}
\begin{abstract}
I would like to thank my supervisor, Steffen van Bakel, for his support and invaluable feedback throughout this project. I seem to have come out of every meeting thinking a little sharper.

I would also like to thank David Davies for his feedback on this report, as well as my friends Aboud, Hamish, and Jerome for helping me test the web application.

Finally, I would like to thank my friends and family for their support.
\end{abstract}

\pagestyle{toc}
\pagenumbering{roman}
\tableofcontents
\listoffigures
% \listoftables
% \cleardoublepage

\pagestyle{fancy}
\pagenumbering{arabic}
\chapter{Introduction}
At Imperial, all first-year undergraduate students in Computing and Joint Mathematics and Computing are required to study propositional logic and first-order logic. Students are introduced to Pandora \cite{pandora:2007, pandora}, an interactive tool developed by former Imperial undergraduates for constructing Fitch-style natural deduction proofs. Interactive tools like Pandora can facilitate learning in numerous ways:

\paragraph{Instant feedback}
Interactive tools provide instant feedback to students on the correctness of their proofs and derivations. Although lecturers can provide model answers to sample problems, there are often multiple correct solutions that would be tedious to exhaustively enumerate. Interactive tools ease the burden of teaching staff on verifying students' solutions. This is not to say they are unimportant in facilitating students' learning---they can help explain \textit{why} a proof or derivation is incorrect using various approaches \cite{nipkow:2012}; while interactive tools can also provide explanations to a certain degree, they are not necessarily tailored to the student's understanding.

\paragraph{More practice problems}
In the Department of Computing at Imperial, students are not provided with mark schemes or model answers to past papers. Although students have access to past paper solutions crowdsourced from seniors and former students, the solutions may not be complete or correct, especially for older papers. Students can use interactive tools to check their solutions, even in the lack of model answers. This extends beyond past papers at Imperial. Arbitrary new problems in natural deduction can be generated \cite{ahmed:2013} for as much practice as students want.

\paragraph{Enforcing methodical approaches}
Although students at Imperial who used Pandora scored similarly in the natural deduction parts of the final exam as those who did not use Pandora, the former approached the problems more methodically, were less likely to make arbitrary assumptions, and were more precise in their proofs \cite{pandora:2007}. This may be because Pandora makes it somewhat tedious to undo large parts of the proof by design. Even after adding the functionality to undo multiple proof steps as proposed in \cite{pandora:2007}, it is still more difficult than crossing out an erroneous solution on paper with a pen. This friction may prompt students to think more carefully before proceeding with the next proof step in Pandora, and with enough practice, on paper as well.

\mbox{}\\
As beneficial as these interactive tools can be, there are relatively few tools for formal derivation systems beyond logic, such as type systems. This may be because most computer science students around the world are required to take courses on logic but not type systems\todo{citation needed}. At Imperial, \textit{Type Systems for Programming Languages} (TSfPL) is an elective module offered from the third year onwards.

\section{Objectives}
This project aims to develop an interactive, web-based learning tool for building derivation trees. It should be able to verify the correctness of the derivation tree and give hints on how to fix a derivation tree when it is malformed. The tool should also support user-defined inference rules.

We will define a syntax for inference rules in extended Backus-Naur form (EBNF). Next, we will build a parser implementing the syntactic rules using parser combinator libraries. We will then define the semantics of the rules and ``compile'' the rules to verifiers that check whether the rule is applied correctly in a derivation tree. The correctness of our implementation will be checked against the type assignment systems in the Lambda Calculus and Milner's \textsc{ml}, as well as Gentzen's sequent calculus.

% \section{Challenges}
% \section{Contributions}
\chapter{Background}\label{background}
This chapter provides an overview of various examples of proof systems, the \lstinline{parjs} parsing library, and a comparison of existing educational proof assistants. \Cref{background:proof-systems} presents three proof systems: Natural Deduction, the simply typed $\lambda$-calculus, and the Sequent Calculus \textsc{lk}. They are chosen because the algorithms presented in \Cref{chapter:input} and \Cref{chapter:checking} are primarily tested on variations of these three systems. \Cref{background:parsing} explains why the \lstinline{parjs} library is chosen for the parsing tasks in this project and presents features of \lstinline{parjs} that guide certain algorithm design decisions. Finally, \Cref{background:comparison} evaluates numerous logic learning tools based on user experience and flexibility to support numerous proof systems, then summarises common features that guide the development of the application.

\section{Proof systems}
This section presents an overview of three proof systems that are pre-defined in the web application, and on which most of the algorithm development is based.

\subsection{Syntax}
The definitions in this section are written 

% The algorithms discussed in the subsequent chapters do not ``understand'' the semantics of the inference rules, whatever it means for an algorithm to ``understand''. Nonetheless, the reader may find it helpful to understand the intuition behind the inference rules, so that they can construct derivations by hand and compare the experience to using the web application.
\subsection{Natural deduction}
In 1934, Gerhard Gentzen proposed his system of natural deduction \cite{gentzen:1969}. This section considers a small subset of natural deduction which has $\to$ as its only logical connective. The syntax of terms in natural deduction is as follows:
\[
    A, B \Coloneqq x \alt (A \to B)
\]
The inference rules are defined as follows:
{
    \derivationfont
    \[
        (Ax): \frac{}{\Gamma, A \vdash A} \tquad (\arr I): \frac{\Gamma, A \vdash B}{\Gamma \vdash (A \to B)} \tquad (\arr E): \frac{\Gamma \vdash (A \to B) \quad \Gamma \vdash A}{\Gamma \vdash B}
    \]
}%
Here, each \textit{statement} is in the form $\Gamma \vdash A$, where $\Gamma$ is a \textit{context} and $A$ is a term as defined above. A \textit{context} $\Gamma$ is an \textit{unordered} multiset of terms as defined above, i.e. it may contain duplicate terms. The notation $\Gamma, A$ is equivalent to the multiset union $\Gamma \cup \{ A \}$, which adds the term $A$ to $\Gamma$ regardless of whether $A$ appears in $\Gamma$.

Of course, it is possible to extend the system with additional logical connectives. Each connective is associated with introduction rules ($\cdot I$) and elimination rules ($\cdot E$). An application of an introduction rule introduces the connective on the right of the turnstile, while an application of an elimination rule eliminates the connective on the right of the turnstile. For example, the subset above can easily be extended with the conjunction operator $\land$:
{
    \derivationfont
    \[
        (\land I): \frac{\Gamma \vdash A \quad \Gamma \vdash B}{\Gamma \vdash (A \land B)} \tquad (\land E_L): \frac{\Gamma \vdash (A \land B)}{\Gamma \vdash A} \tquad (\land E_R): \frac{\Gamma \vdash (A \land B)}{\Gamma \vdash B}
    \]
}%

\subsubsection{Choice of notation}
In Gentzen's original formulation \cite{gentzen:1969}, there are neither contexts nor turnstiles. Each statement only consists of a term. The $(\arr I)$ rule in the original formulation can refer to any term arbitrarily higher up. It is not \textit{localised}, since it does not always only depend on the premises and conclusion immediately above and below the dividing line. On the contrary, the formulation presented above is \textit{localised}, since all rules only depend on the premises and conclusion immediately above and below the dividing line. The differences between Gentzen's original formulation and the formulation above are highlighted in \Cref{fig:background:natural-deduction}.

\begin{figure}[!htbp]
    \centering
    \begin{subfigure}{.48\textwidth}
        \centering
        \[
            \Inf[\textcolor{ForestGreen}{\arr I^u}]
                {\Inf[\textcolor{blue}{\arr I^v}]
                     {\textcolor{blue}{[x]^v}
                      \quad \Inf[\land E]
                                {\textcolor{ForestGreen}{[x \land y]^u}
                                }{y}
                     }{\textcolor{blue}{x \to y}}
                }{\textcolor{ForestGreen}{(x \land y) \to (x \to y)}}
        \]
        \caption{Gentzen's original formulation}
    \end{subfigure}%
    \quad
    \begin{subfigure}{.48\textwidth}
        \centering
        \[
            \Inf[\arr I]
                {\Inf[\arr I]
                     {\Inf[\land E_R]
                          {\Inf[Ax]{x \land y, x \vdash x \land y}
                          }{x \land y, x \vdash y}
                     }{x \land y \vdash x \to y}
                }{\varnothing \vdash (x \land y) \to (x \to y)}
        \]
        \caption{Localised formulation}
    \end{subfigure}
    \caption{An example derivation highlighting the differences between Gentzen's original formulation of natural deduction and a localised formulation}
    \label{fig:background:natural-deduction}
\end{figure}

This project only considers \textit{localised} inference rules, since they are simpler to deal with and are applicable to a wider range of proof systems.

\subsection{Simply typed \texorpdfstring{$\lambda$-calculus}{Lambda Calculus}}
In the 1930s, Alonzo Church introduced the $\lambda$-calculus \cite{church:1936} as a model of computation that is Turing-complete \cite{turing:1937}. It is the basis of functional programming languages like Haskell, which all Computing students at Imperial are required to learn. The $\lambda$-calculus is taught as part of the mandatory second-year \textit{Models of Computation} module and the TSfPL elective.

Church later formulated the simply typed $\lambda$-calculus \cite{church:1940}, which has only one type constructor $\to$ representing function types. This section focuses on type assignment in the manner of Haskell Curry \cite{curry:1934}. In Curry-style type assignment, types are assigned to entire $\lambda$-terms. In Church-style type assignment as presented in \cite{church:1940}, types are embedded as variable annotations.

\subsubsection{Why types?}
\begin{itemize}
    \item Types can guarantee the absence of certain errors. For example, the code snippet \lstinline{4.0 / "three"} will not compile in most statically-typed programming languages because \lstinline{"three"} is not a number. However, types can also reject programs that are well-behaved at run-time. For example, a conservative type system may reject the code snippet \lstinline{if true then 10 else "ten"} because \lstinline{10} and \lstinline{"ten"} have different types.
    \item Type annotations can make code more readable. They help clarify the intended usage of variables and functions by supplementing information that are missing from names.
    \item Types can enable certain compiler optimisations, such as using specialised machine instructions for arithmetic operations and determining whether a variable can be allocated on the stack. They allow the compiler to produce more efficient machine code.
\end{itemize}

\subsubsection{\texorpdfstring{$\lambda$}{Lambda}-terms}
\label{lambda:lambda-terms}
$\lambda$-terms are defined as follows \cite{church:1941}:
\[
    M,N \Coloneqq x \alt \underbracket[0.6pt]{(\lambda x. M)}_\text{abstraction} \alt \underbracket[0.6pt]{(MN)}_\text{application}
\]
where $x$ can be any symbol from an infinite list of term variables $a, b, c, \ldots, x, y, z \ldots$. Observe that abstractions and applications contain brackets to avoid ambiguity.

\subsubsection{Curry types}
\label{lambda:curry-types}
The set of Curry \textit{types} is defined as follows \cite{van-bakel:2022}:
\[
    A, B \Coloneqq \varphi \alt (A \rightarrow B)
\]
where $\varphi$ can be any symbol from an infinite list of type variables $\varphi_1, \varphi_2, \ldots$. When writing type variables by hand, it is often more convenient to use the subscript alone to represent a type variable, e.g. the type $((1 \rightarrow 2) \rightarrow 1)$ represents the type $((\varphi_1 \rightarrow \varphi_2) \rightarrow \varphi_1)$.

\subsubsection{Type assignment rules}
\label{lambda:type-assignment}
$\lambda$-terms can be assigned types under Curry's type assignment system using the following derivation rules \cite{van-bakel:2022}:
{
    \derivationfont
    \[
        (Ax): \frac{}{\Gamma, x:A \vdash x:A} \quad (\arr I): \frac{\Gamma, x:A \vdash M:B}{\Gamma \vdash \lambda x. M: (A \to B)} \quad (\arr E): \frac{\Gamma \vdash M: (A \to B) \quad \Gamma \vdash N: A}{\Gamma \vdash MN: (A \to B)}
    \]
}%
A \textit{context} $\Gamma$ is a set containing elements in the form $x:A$, where $x$ is a variable and $A$ is a Curry type. All variables are assigned at most one type in any context. For example, $x:1, x:2$ is not a well-formed context since $x$ appears twice.

\subsection{Sequent Calculus \textsc{lk}}
In 1934, Gerhard Gentzen proposed the Sequent Calculus \textsc{lk} (standing for \textit{Logistische Kalkül}) \cite{gentzen:1969}, used to build proofs for classical first-order logic.

\subsubsection{Terms in propositional logic}
The subset of propositional formulas relevant to this section are defined as follows:
\[
    A, B \Coloneqq x \alt (A \to B) \alt (A \land B) \alt (A \lor B) \alt (\lnot A)
\]
where $x$ can be any symbol from an infinite list of term variables $a, b, c, \ldots, x, y, z \ldots$ as before.

\subsubsection{Inference rules in the system \textsc{lk} restricted to propositional logic}
Each statement (or \textit{sequent}) takes the form
\[
    \Gamma \vdash \Delta
\]
where $\Gamma$ and $\Delta$ represent possibly empty, unordered \textit{multiset}s of propositional formulas. Note that $\Gamma$ in the simply typed $\lambda$-calculus is a \textit{set} (i.e. it cannot contain duplicates), while the $\Gamma$ and $\Delta$ here in the system \textsc{lk} is a \textit{multiset} (i.e. it can contain duplicates). When expanded, a statement in the system \textsc{lk} looks like this:
\[
    A_1, \ldots, A_n \vdash B_1, \ldots B_m
\]
where $n, m \geq 0$. It means ``if every $A_1, \ldots, A_n$ is true, then at least one of $B_1, \ldots, B_m$ is true''. Alternatively, a statement holds if and only if
\[
    (A_1 \land \cdots \land A_n) \vdash (B_1 \lor \cdots \lor B_m)
\]
holds.

The inference rules are presented as follows:
\vspace{-11pt}
\begin{center}
    \derivationfont
    \begin{minipage}{.4\textwidth}
        \begin{align*}
            (Ax) &: \frac{}{A \vdash A} \\[1em]
            (\land L_1) &: \frac{\Gamma, A \vdash \Delta}{\Gamma, (A \land B) \vdash \Delta} \\[1em]
            (\land L_2) &: \frac{\Gamma, B \vdash \Delta}{\Gamma, (A \land B) \vdash \Delta} \\[1em]
            (\arr L) &: \frac{\Gamma, A \vdash \Delta \quad \Gamma, B \vdash \Delta}{\Gamma, (A \lor B) \vdash \Delta} \\[1em]
            (\arr L) &: \frac{\Gamma \vdash A, \Delta \quad \Sigma, B \vdash \Pi}{\Gamma, \Sigma, (A \to B) \vdash \Delta, \Pi} \\[1em]
            (\lnot L) &: \frac{\Gamma \vdash A, \Delta}{\Gamma, (\lnot A) \vdash \Delta} \\[1em]
            (WL) &: \frac{\Gamma \vdash \Delta}{\Gamma, A \vdash \Delta} \\[1em]
            (CL) &: \frac{\Gamma, A, A \vdash \Delta}{\Gamma, A \vdash \Delta}
        \end{align*}
    \end{minipage}%
    \begin{minipage}{.4\textwidth}
        \begin{align*}
            (Cut) &: \frac{\Gamma \vdash \Delta, A \quad A, \Sigma \vdash \Pi}{\Gamma, \Sigma \vdash \Delta, \Pi} \\[1em]
            (\lor R_1) &: \frac{\Gamma \vdash A, \Delta}{\Gamma \vdash (A \lor B), \Delta} \\[1em]
            (\lor R_2) &: \frac{\Gamma \vdash B, \Delta}{\Gamma \vdash (A \lor B), \Delta} \\[1em]
            (\land R) &: \frac{\Gamma \vdash A, \Delta \quad \Gamma \vdash B, \Delta}{\Gamma \vdash (A \land B), \Delta} \\[1em]
            (\arr R) &: \frac{\Gamma, A \vdash B, \Delta}{\Gamma \vdash (A \to B), \Delta} \\[1em]
            (\lnot R) &: \frac{\Gamma, A \vdash \Delta}{\Gamma \vdash (\lnot A), \Delta} \\[1em]
            (WR) &: \frac{\Gamma \vdash \Delta}{\Gamma \vdash A, \Delta} \\[1em]
            (CR) &: \frac{\Gamma \vdash A, A, \Delta}{\Gamma \vdash A, \Delta}
        \end{align*}
    \end{minipage}
\end{center}
Here, $A$ and $B$ represent propositional formulas as defined above, while $\Gamma$, $\Delta$, $\Sigma$, and $\Pi$ represent multisets of propositional formulas. Some formulations in the literature may include the following rules as well:
{
    \derivationfont
    \[
        (PL): \frac{\Gamma_1, A, B, \Gamma_2 \vdash \Delta}{\Gamma_1, B, A, \Gamma_2 \vdash \Delta} \tquad (PR): \frac{\Gamma \vdash \Delta_1, A, B, \Delta_2}{\Gamma \vdash \Delta_1, B, A, \Delta_2}
    \]
}%
These rules are unnecessary in this project. They are necessary in those formulations because $\Gamma_1$, $\Gamma_2$, $\Delta_1$, and $\Delta_2$ are treated as \textit{ordered} sequences of propositional formulas, but not \textit{unordered} multisets.

\subsubsection{Why the Sequent Calculus?}
Gentzen proved the cut-elimination theorem, also known as his \textit{Hauptsatz}, for the system \textsc{lk} \cite{gentzen:1969}. It states that any statement provable using the rule Cut in the system \textsc{lk} is provable without using the rule Cut.

An important consequence of the cut-elimination theorem is that every statement provable in the system \textsc{lk} has a proof which has the \textit{subformula property}: every subformula in the statement is a subformula of at least one of its premises. This follows from the cut-elimination theorem, and the observation that in every rule except Cut, all subformulas in the premises appear in the conclusion.

The subformula property implies the consistency of the system \textsc{lk}. The system \textsc{lk} is inconsistent if and only if the empty sequent $\varnothing \vdash \varnothing$ is provable. The empty sequent is not an axiom and no rules except Cut can be applied to the empty sequent. There is no proof for the empty sequent with the subformula property, so it is not provable in the system \textsc{lk}.

\section{Parsing}
Parsing is the process of breaking up strings or sequences of symbols into numerous parts and transforming these parts into formats more convenient for manipulation. In this project, parsing is done to syntax rules, inference rules, and the derivation tree. The user types the rule definitions and constructs the derivation tree using text inputs, which store values as strings. It is the job of the parsers to transform these string inputs into data structures which conveniently represent the information of the syntax rules, the inference rules, and the derivation tree, respectively. The data structures are described in more detail in the relevant chapters (\Cref{chapter:syntax} for syntax rules, \Cref{chapter:term} for user input in the derivation tree, and \Cref{chapter:inference} for inference rules).

\subsection{Choice of tooling}
Ideally, all parsing in this project is done client-side. In other words, all computations related to parsing are done on the user's machine but not by a server. Any parsing in this project should not be computationally expensive enough to warrant the computational resources of an external server. When parsing is done client-side, the user can use the tool without an Internet connection and will not experience unpredictable delays and latency issues due to web traffic conditions.

JavaScript is an obvious choice for writing client-side code. In fact, the frontend of this project is written using React, a JavaScript-based library for web user interfaces. If the user's device can load the web application, it can also load any parsing-related code written in JavaScript. Most modern browsers like Google Chrome, Safari, Microsoft Edge, and Firefox support JavaScript. As of April 2025, these four browsers take up 92.4\% of the global market share for desktops \cite{statcounter}.

TypeScript is a statically typed version of JavaScript and is transpiled to JavaScript before being run on browsers. TypeScript code ``runs anywhere JavaScript runs'' \cite{typescript}.

However, other programming languages like C++ and Rust can also be run in modern web browsers by compiling to WebAssembly. WebAssembly is an assembly-like language that can be run on most modern web browsers with ``near-native performance'' \cite{webassembly}. Functions written in languages like C++ and Rust can be compiled to WebAssembly and called in the frontend like any JavaScript function. WebAssembly is supported by browsers like Google Chrome, Safari, Microsoft Edge, and Firefox, and is available to 96.09\% of all browser users as of May 2025 \cite{webassembly:caniuse}.

A tool for transpiling TypeScript into JavaScript is needed regardless of the choice of the backend programming language, as the frontend is written in TypeScript. Therefore, writing the backend in TypeScript minimises additional tooling setup overhead and maximises development time. The only motivation to use another language is if all parsing libraries written in TypeScript either do not have enough features or are too slow. Of course, one could also not use any parsing libraries and write parsers by hand, though it would be more time-consuming than using a library.

The \lstinline{parjs} parser combinator library \cite{parjs}, written in TypeScript, is found to be capable of handling all parsing tasks in this project while the web application remains quick and responsive. The next subsection provides an overview of the parts of the library necessary to understand the inner workings of this project.

\subsection{\texorpdfstring{\lstinline{parjs}}{parjs} overview}
This subsection introduces the \lstinline{then} and \lstinline{or} combinators of the \lstinline{parjs} library. 

\subsubsection{Parsing failures}
Parsers in \lstinline{parjs} emit one of three types of failures:
\begin{itemize}
    \item \textbf{Soft} failures allow parsers at a higher level to catch the failure and backtrack. Soft failures are used when parsing a sequence of alternatives using the \lstinline{or} combinator.
    \item \textbf{Hard} failures cause parsing to halt immediately, unless a \lstinline{recover} combinator is explicitly used to catch the hard failure and downgrade it to a soft failure.
    \item \textbf{Fatal} failures are explicitly generated by the user to tell the parser to halt and catch fire. Fatal failures are not generated by default by any of the parsers.
\end{itemize}

\subsubsection{The \lstinline{pipe} operator}
The \lstinline{pipe} operator \cite{parjs:pipe} is applied to a parser (the \textit{source parser}) and takes a sequence of combinators as its argument. It takes the output of the source parser and feeds it into the first combinator, then takes the output of the first combinator and feeds it into the second combinator, and so on. The final output is the output of the last combinator.

\subsubsection{The \lstinline{or} combinator}
The \lstinline{or} combinator \cite{parjs:or} is used for parsing a sequence of alternatives. The parser
\begin{center}
    \lstinline{string("sleeping").pipe(or(string("eating")), or(string("drinking")))}
\end{center}
successfully parses the strings ``sleeping'', ``eating'', and ``drinking'', but nothing else. When trying to parse ``eating'', the parser first tries to parse ``sleeping'' and fails softly: the parser \lstinline{string("sleeping")} emits a soft failure. As the first combinator is the \lstinline{or} combinator, the parser backtracks and proceeds to apply the parser \lstinline{string("eating")}, which succeeds. The successful parsing result is fed into the second \lstinline{or} combinator, which does not apply its argument parser \lstinline{string("drinking")} and simply returns the successful result from the previous combinator.

\subsubsection{The \lstinline{then} combinator}
The \lstinline{then} combinator \cite{parjs:then} is used for chaining multiple parsers sequentially. The parser
\begin{center}
    \lstinline{string("I").pipe(then(string(" study")), then(string(" computing")))}
\end{center}
successfully parses the string ``I study computing'' and nothing else. It is important to note that if a \lstinline{then} combinator is ``reached'' (i.e. all of its previous parsers have parsed the input successfully so far) and its argument fails, the \lstinline{then} combinator returns a failure no less severe than a hard failure. This means if its argument returns a soft failure, the \lstinline{then} combinator ``upgrades'' the soft failure and returns a hard failure.

The given parser fails softly when given the string ``go home'', since the first parser \lstinline{string("I")} fails softly and does not ``reach'' the \lstinline{then} combinator. However, the parser fails hard when given the string ``Imperial College London''. The first parser \lstinline{string("I")} succeeds, so it proceeds to apply the argument to the first \lstinline{then} combinator, which is \lstinline{string(" study")}. It returns a soft failure, which is ``upgraded'' by the \lstinline{then} combinator to a hard failure. The hard failure is propagated through the second \lstinline{then} combinator, causing the overall parser to return a hard failure.

\subsubsection{Using the \lstinline{or} and \lstinline{then} combinators together}
\label{parsing:thenor}
The behaviour of the \lstinline{then} combinator ``upgrading'' soft failures to hard failures makes it tricky to use the \lstinline{or} combinator with more complex parsers. Consider the following syntax definition:
\begin{align*}
    S &\Coloneqq Xa \alt Xb \\
    X &\Coloneqq x
\end{align*}
One might mimic the structure of the definitions and create parsers as follows:
\begin{lstlisting}
    const x = string("x");
    const first = x.pipe(then(string("a")));
    const second = x.pipe(then(string("b")));
    const s = first.pipe(or(second));
\end{lstlisting}
However, the parser \lstinline{s} only successfully parses the string ``xa'' but not ``xb''. In the latter case, the parser \lstinline{s} first tries to apply the parser \lstinline{first}. The parser \lstinline{first} tries to apply the parser \lstinline{x}, which succeeds. The parser \lstinline{first} then tries to apply the parser \lstinline{string("a")}, which fails softly. The \lstinline{then} combinator ``upgrades'' the soft failure to a hard failure, causing the \lstinline{first} parser to return a hard failure as well. The hard failure is propagated to \lstinline{or(second)}, which simply passes the hard failure along and causes the parser \lstinline{s} to return a hard failure as well.

One solution is to manually recover from the hard failures returned by the \lstinline{then} combinators using the \lstinline{recover} combinator:
\begin{lstlisting}
    ...
    const first = x.pipe(then(string("a")), recover(() => ({ kind: "Soft" })));
    const second = x.pipe(then(string("b")), recover(() => ({ kind: "Soft" })));
    ...
\end{lstlisting}
Here, the \lstinline{recover} combinator ``downgrades'' any failure, including hard and fatal failures, returned from the previous step to a soft failure. However, not only is this quite cumbersome, it also masks hard failures returned by the arguments to the \lstinline{then} combinators that should actually be reported to the user and cause parsing to halt. (The \lstinline{string} parsers in this case do not return hard failures, but more complex parsers in general may return hard failures.)

The idiomatic solution, then, is to not create parsers by mimicking the structure of the definitions. When parsing definitions with multiple alternatives, the parser should first match the longest common prefix across all of the alternatives, then only apply the \lstinline{or} combinator when the subsequent parts of the alternatives are distinct \cite{parjs}. In this example, the parsers should be created like so:
\begin{lstlisting}
    const x = string("x");
    const s = x.pipe(or(string("a")), or(string("b")));
\end{lstlisting}
This corresponds to the factorised definitions
\begin{align*}
    S &\Coloneqq X(a|b) \\
    X &\Coloneqq x
\end{align*}
Syntax rules defined by the user may not always be factorised. Instead of throwing an error and telling the user to do the factorisation themselves, a factorisation algorithm processes the user-defined rules and outputs a factorised version of the rules. The factorisation algorithm is described in detail in \Cref{syntax:factorisation}.

Yet line 6 fails. The \lstinline{term} parser first tries to apply the \lstinline{variable} parser, which fails softly, then it backtracks and tries the next alternative, which is the \lstinline{abstraction} parser. The input begins with a left parenthesis \lstinline{(}, so the \lstinline{string("(")} parser succeeds. The subsequent input does not match \lstinline{\lambda}, so the \lstinline{string("\\lambda")} parser emits a soft failure. The \lstinline{then} combinator turns the soft failure into a hard failure, causing the \lstinline{abstraction} parser to stop parsing and emit the hard failure. The \lstinline{term} parser cannot recover from the hard failure and also stops parsing. If the \lstinline{application} parser had emitted a soft failure instead, the \lstinline{term} parser would have been able to backtrack and try the last alternative, the \lstinline{application} parser, which would have successfully parsed the input.


% The behaviour of the \lstinline{then} combinator ``upgrading'' soft failures to hard failures makes it tricky to use the \lstinline{or} combinator with more complex parsers. Consider the usual formulation of the definition of $\lambda$-terms \cite{church:1941}:
% \[
%     M,N \Coloneqq x \alt (\lambda x. M) \alt (MN)
% \]
% is not LL(0) since the second and third alternatives both begin with a left parenthesis \lstinline{(}. If one were to generate a parser for $\lambda$-terms without factorising the alternatives, one might write:
% \begin{lstlisting}
%     const variable = string("x");
%     const abstraction = string("(").pipe(then(string("\\lambda")...));
%     const application = string("(").pipe(then(...));
%     const term = variable.pipe(or(abstraction), or(application));

%     term.parse("((\\lambda x. x)(\\lambda x. x))");
% \end{lstlisting}
% Yet line 6 fails. The \lstinline{term} parser first tries to apply the \lstinline{variable} parser, which fails softly, then it backtracks and tries the next alternative, which is the \lstinline{abstraction} parser. The input begins with a left parenthesis \lstinline{(}, so the \lstinline{string("(")} parser succeeds. The subsequent input does not match \lstinline{\lambda}, so the \lstinline{string("\\lambda")} parser emits a soft failure. The \lstinline{then} combinator turns the soft failure into a hard failure, causing the \lstinline{abstraction} parser to stop parsing and emit the hard failure. The \lstinline{term} parser cannot recover from the hard failure and also stops parsing. If the \lstinline{application} parser had emitted a soft failure instead, the \lstinline{term} parser would have been able to backtrack and try the last alternative, the \lstinline{application} parser, which would have successfully parsed the input.

% Suppose we create a parser using the \lstinline{then} combinator like
% \begin{center}
%     \lstinline{parser = parser1.pipe(then(parser2))}
% \end{center}
% This returns a parser that tries to parse using \lstinline{parser1}, and if \lstinline{parser1} succeeds, then tries to parse the rest of the input using \lstinline{parser2}. However, if \lstinline{parser2} fails (softly), the \lstinline{then} combinator produces a hard failure, which propagates to the top level and halts parsing immediately by default, instead of a soft failure, which can be caught by higher level combinators without failing catastrophically. In order to backtrack to \lstinline{parser1} if \lstinline{parser2} fails (softly), we must manually recover from the \lstinline{then} combinator like so:
% \begin{center}
%     \lstinline|parser = parser1.pipe(then(parser2), recover(() => ({ kind: "Soft" })))|
% \end{center}
% which is inconvenient and advised against in the README of the \lstinline{parjs} repo \cite{parjs}. In the context of parsing syntax rules, the user may supply rules in which multiple alternatives begin with the same token (i.e. not LL(0)). 

% Build parser using ts-parsec (use ts so everything runs on client-side)
% Allow a range of input formats (current: Haskell-like, future: latex, ascii?)
% Convert all input formats into latex

\section{What makes a learning tool enjoyable to use?}\label{background:enjoyable}
In this section, we will discuss several learning tools for natural deduction in propositional and first-order logic, focusing on the features that make them enjoyable or not enjoyable to use.

\subsection{Pandora}
Pandora \cite{pandora:2007} is a tool that helps students learn Fitch-style natural deduction. The current version \cite{pandora} is written in Java by former Imperial students for their undergraduate capstone projects. At Imperial, it is presented during lectures in the first-year logic module.

\subsubsection{Unnatural user interactions}
Using detailed logs to keep track of clicks and other interactions with Pandora, \cite{pandora:2007} found that students made infrequent use of the help and tutorial functionalities, even though they often failed to apply the rules correctly. For example, many students did not select the necessary lines before applying a rule. We hypothesise that students make these frequent mistakes when using Pandora because the sequence of interactions for applying rules does not correspond to how they apply natural deduction rules when writing proofs by hand. Suppose a student wants to apply the $\rightarrow I$ rule to lines 1 and 2. The justification would look like $\rightarrow I(1, 2)$. It is natural to write it from left to right, starting from $\rightarrow I$, then perhaps one or both of the brackets, then writing the line number 1, and finally the line number 2. The natural translation of this sequence into Pandora interactions would be to first click on the $\rightarrow I$ rule, then click on line 1, and finally click on line 2. Clearly, there is a discrepancy between this sequence and the current design.

Therefore, we should design our tool such that the interactions required to build derivations mimic as closely as possible how students would be producing assessed or marked work. The more similar they are, the less assistance and explanation is needed for students to use the tool correctly.

\subsubsection{Installation required}
Pandora is \textit{not} a web-based application. It can be run either as a JAR executable or using Java Web Start, a deprecated framework for starting Java applications using a web browser. The former starts up but fails to start a proof correctly and is essentially useless. The latter is not supported from Java 11 onwards \cite{oracle:2020}, and even with Java 8 installed, the application throws an error on startup saying it is ``unsigned'' on the author's machine. Whatever means needed to open a functional Pandora version, if possible at all, is simply too complicated.

The last update of Pandora was more than 15 years ago, according to the timestamp ``200810031955'' on the website \cite{pandora}. No matter how carefully we choose our tech stack, we cannot guarantee how long native applications will work until it fails to be compatible with future hardware, or be as tedious to install for students two decades later as Pandora is today. A simple workaround is to build a web application. It does not require installation other than a suitable web browser and the current versions of the canonical web technologies---HTML, CSS, and JavaScript---should not cease to be forwards-compatible any time soon.

\subsection{Carnap}
Carnap \cite{carnap, carnap:2018} is an educational tool for a variety of formal reasoning systems and used by over 35 universities globally \cite{carnap:about}. It is written in Haskell and can be transpiled to JavaScript to be run on web browsers. The Carnap Book \cite{carnap:book} is a free web-based textbook with interactive widgets for practice problems on topics such as truth tables, formation trees, and natural deduction.

\subsubsection{Clear layout of all syntactic rules}
Although there are no syntax guides immediately surrounding the widgets in the textbook, the relevant syntax is introduced when a new widget first appears. This is natural when going through the textbook sequentially, but can be cumbersome for someone hoping to review exercises in later parts of the textbook but have forgotten the syntax. The input syntaxes for all formal systems supported by Carnap are enumerated on one webpage \cite{carnap:systems}, though it is somewhat unintuitive to access from the home page. It is possible that users may experience less friction if they can be quickly reminded of the input syntax in all situations.

\chapter{User input}
\label{chapter:input}
This chapter describes the algorithms and data structures used for parsing and representing user input in \projectname{}. User input in \projectname{} can be divided into three categories: syntax rules (\Cref{section:syntax}), concrete terms (\Cref{section:term}), and inference rules (\Cref{section:inference}).
\chapter{Syntax rules}
\label{chapter:syntax}
\section{Syntax of syntax rules}
\label{syntax:syntax}
A syntax rule consists of:
\begin{itemize}
    \item a list of comma-separated \textbf{placeholders}, and
    \item a \textbf{definition}, consisting of one or more non-empty alternatives separated by vertical bars \lstinline{|} 
\end{itemize}
In rule definitions, curly braces \lstinline|{}| represent a multiset. The structure of each element of a multiset is defined by the content enclosed within the curly braces. For example, the definition \lstinline|{ M: A }| represents a possibly empty collection of possibly duplicate elements following the structure \lstinline{M: A}.

Every valid syntax definition (i.e. collection of syntax rules) must contain at least one rule for a statement. A valid syntax definition can contain multiple rules for a statement. All non-statement rules must specify placeholders. Otherwise, there is no way to refer to the definition of such rules and they might as well be removed from the syntax definition.

The only special character in the definition of placeholders is the comma. The special characters in the definition of a syntax rule are the vertical bar \lstinline{|} and the curly braces \lstinline|{}|. Unlike in EBNF, characters like parentheses \lstinline{()} and brackets \lstinline{[]} do not carry any special meaning anywhere in this system and are no different from any other ordinary character. Certain groups of characters, such as \lstinline{|-} and \lstinline{->}, are treated as one symbol and shorthands for \LaTeX{} commands, such as \lstinline{\vdash} and \lstinline{\rightarrow}.

A rule definition can refer to itself or other rules by their placeholders. In the definition of syntax rules, there is no difference using one placeholder over another to refer to the same rule. For example, given the placeholders \lstinline{A} and \lstinline{B} for a rule, the definitions \lstinline{(A -> A)}, \lstinline{(B -> B)}, \lstinline{(A -> B)}, and \lstinline{(B -> A)} are all identical. However, it is useful to have multiple placeholders because in inference rules, placeholders are treated as names representing concrete objects.

\section{User interface}

\section{Parsing syntax rules}
The aim of this step is to generate a list of \lstinline{Token}s for each rule alternative, and thus a two-dimensional array of \lstinline{Token}s for each rule definition. These \lstinline{Token}s will be used to generate term parsers, i.e. parsers for parsing instances of each rule.

\subsection{Tokens}
The tokens are defined as follows:
\begin{lstlisting}
    class Terminal { constructor(readonly value: string) {} }
    class NonTerminal { constructor(readonly index: number) {} }
    class Multiset { constructor(readonly tokens: Token[]) {} }
    class Or { constructor(readonly alternatives: Token[][]) {} }
    class Maybe { constructor(readonly alternatives: Token[][]) {} }

    type Token = Terminal | NonTerminal | Multiset | Or | Maybe;
\end{lstlisting}
The \lstinline{index} field of \lstinline{NonTerminal} is the rule number of the placeholder. For example, if a syntax definition contains two rules
\begin{align*}
    S &\Coloneqq \ldots \\
    A, B &\Coloneqq \ldots
\end{align*}
Then any occurrences of the placeholder $S$ are represented by a \lstinline{NonTerminal} with \lstinline{index} 0 and any occurrences of either of the placeholders $A$ or $B$ are represented by a \lstinline{NonTerminal} with \lstinline{index} 1.

The \lstinline{tokens} field of \lstinline{Multiset} defines the structure of an element of the multiset. For example, consider a rule definition that contains a multiset:
\begin{align*}
    S &\Coloneqq \{ var: A \}
\end{align*}
Assuming $var$ and $A$ are placeholders, the rule definition is represented by a \lstinline{Multiset} with \lstinline{tokens} field set to something like \lstinline{[NonTerminal(...), Terminal(":"), NonTerminal(...)]}.

The \lstinline{Or} and \lstinline{Maybe} tokens are only generated by the factorisation algorithm, as described in \Cref{syntax:factorisation}.

\subsection{Procedure}
The process consists of three steps.

\subsubsection{Splitting placeholders and alternatives of rule definitions}
When the backend receives user input from the frontend, every rule is specified by a string representing the placeholders and a string representing the definition. The placeholder string is split on commas to give a list of placeholders. To preprocess the definition string, all occurrences of \lstinline{|-} are replaced with \lstinline{\vdash}, the \LaTeX{} command for a turnstile, then the string is split on vertical bars \lstinline{|} to give a list of alternatives. The replacement is necessary and correct as \lstinline{|-} contains a vertical bar \lstinline{|} while \lstinline{\vdash} does not.

\subsubsection{Normalise placeholders and rule definitions}
Aliases of a symbol are unified. For example, the symbol for an arrow \lstinline{->} can be represented by \lstinline{\rightarrow}, \lstinline{\to}, \lstinline{->}, and the Unicode character →. In this case, all aliases are replaced with \lstinline{->} (other than having the fewest characters and better readability, the choice is entirely arbitrary). Prior to the replacements, a space is added after all substrings that resemble a \LaTeX{} command, i.e. any substring beginning with \lstinline{\} and a string of non-whitespace characters. This ensures we do not accidentally replace parts of a \LaTeX{} command and e.g. turn \lstinline{\toI} into \lstinline{->I}. The user may have intended the latter, but the former is a typo nonetheless.

\subsubsection{Generate parser}
The parser outputs three types of tokens: \lstinline{Terminal}, \lstinline{NonTerminal}, and \lstinline{Multiset}. The final parser tries to match a non-terminal, then (if it fails) a multiset, then (if both fail) a terminal. Each of these sub-parsers is generated as follows:
\begin{itemize}
    \item \lstinline{Terminal}: it matches, in descending order of priority, any string that resembles a \LaTeX{} command, any special group of characters e.g. \lstinline{|-} and \lstinline{->}, or any single character.
    \item \lstinline{NonTerminal}: it matches any placeholder across all rules in descending order of length.
    \item \lstinline{Multiset}: it matches a sequence of zero or more \lstinline{Terminal}s or \lstinline{NonTerminal}s enclosed within curly braces \lstinline|{}|. The empty multiset definition \lstinline|{}| is technically valid but not very useful: the user can only supply the empty set or a string of commas.
\end{itemize}

\subsubsection{Factorise rule definitions}
\label{syntax:factorisation}
The need for this step is primarily motivated by the idiosyncrasies of \lstinline{parjs}, particularly its \lstinline{then} combinator, as explained in \Cref{parsing:thenor}. 

The factorisation algorithm takes a list of alternatives, each of which is represented by a list of tokens (\lstinline{Terminal}, \lstinline{NonTerminal}, or \lstinline{Multiset}), and outputs a new list of alternatives such that no two alternatives share a prefix. The algorithm is as follows:
\begin{enumerate}
    \item Partition the alternatives by their leading tokens. All alternatives beginning with the same \lstinline{Terminal} or \lstinline{NonTerminal} are grouped together. Recall that \lstinline{NonTerminal}s only stores the rule number and not the placeholder, so multiple alternatives beginning with different placeholders for the same rule are grouped together. Each partition now corresponds to a new alternative, because it does not share a leading token with any other partition. Since every member of a partition share a leading token, the new alternative correspodning to the partition must begin with that token. 
    \item Remove the first token from every member of every partition and recursively factorise each partition.
\end{enumerate}
Notice that the partitioning ignores \lstinline{Multiset} tokens. The algorithm assumes the definition of a multiset element does not share a leading token with any other alternative as it is otherwise difficult to handle:
\begin{itemize}
    \item If it shares a leading token with another multiset element, after the term parser parses one element and determines which multiset alternative the element belongs to, it must only accept this alternative from the second element onwards. It would be incorrect to treat the alternatives \lstinline|{ ab }| and \lstinline|{ ac }| as \lstinline*{ a(b|c) }* (by abuse of EBNF notation).
    \item The case where it shares a leading token with a non-multiset alternative is a variation of the case above. If the term parser parses a multiset 
\end{itemize}
Both of these cases are intentionally unsupported as they rarely occur. In the case of \lstinline{NonTerminal}s, the algorithm does not handle the case where different rules have non-disjoint \textit{first sets}. The \textit{first set} of a set of alternatives is the set containing every leading token of every alternative\footnote{If an alternative begins with a \lstinline{Terminal}, its contribution to the first set is the \lstinline{Terminal}. If an alternative begins with a \lstinline{NonTerminal}, its contribution to the first set is the first set of that \lstinline{NonTerminal}. If an alternative begins with a \lstinline{Multiset}, its contribution to the first set is the contribution of the first token of the element definition.}. For example, given the following rules:
\begin{align*}
    S &\Coloneqq Axx \alt Byy \\
    A &\Coloneqq x \alt y \\
    B &\Coloneqq x \alt z
\end{align*}
the factorisation algorithm would throw an error because \lstinline{A} and \lstinline{B} refer to different rules but have non-disjoint first sets. This scenario occurs more frequently and naturally than the previous case. Consider a more verbose formulation of the syntax of $\lambda$-terms:
\begin{align*}
    M, N &\Coloneqq \nonterm{var} \alt \nonterm{abstraction} \alt \nonterm{application} \\
    \nonterm{var} &\Coloneqq x \alt y \alt z \\
    \nonterm{abstraction} &\Coloneqq (\lambda x. M) \\
    \nonterm{application} &\Coloneqq (MN)
\end{align*}
The alternatives $\nonterm{abstraction}$ and $\nonterm{application}$ share non-disjoint first sets but belong to different rules. The factorisation algorithm would throw an error at this formulation.

One solution is to recursively replace every offending \lstinline{NonTerminal} with each of the \lstinline{NonTerminal}'s alternative definitions, but this destroys the original structure of the syntax rules and greatly complicates bookkeeping. Throwing an error instead of algorithmically handling this case forces the user to rewrite the syntax rules and allows the parsing result to resemble the original structure of the syntax rules as closely as possible.

In addition to the three types of tokens previously mentioned (\lstinline{Terminal}, \lstinline{NonTerminal}, and \lstinline{Multiset}), the factorisation algorithm may output two more types of tokens: \lstinline{Or} and \lstinline{Maybe}. The semantics of the new tokens is best illustrated through the examples in \Cref{table:factorisation}.

\begin{table}
    \centering
    \caption{The factorisation algorithm applied to various rule definitions}
    \begin{tblr}{ll}
        \toprule
        Rule definition & Factorisation output \\
        \midrule
        \lstinline{ab | ac} & \lstinline{[Terminal("a"), Or([Terminal("b")], [Terminal("c")])]} \\
        \lstinline{a | ab} & \lstinline{[Terminal("a"), Maybe([[Terminal("b")]])]} \\
        \lstinline{a | ab | ac} & \lstinline{[Terminal("a"), Maybe([[Terminal("b")][Terminal("c")]])]} \\
        \bottomrule
    \end{tblr}
    \label{table:factorisation}
\end{table}

The beginning of this section notes that after parsing, each rule definition becomes a two-dimensional array of tokens, in which each element (a one-dimensional array of tokens) represents a parsed rule alternative. One could also represent a collection of top-level alternatives as an \lstinline{Or} token. Here, the two-dimensional array is chosen over the top-level \lstinline{Or} token to distinguish between the output of the parsing step and the output of the factorisation algorithm, though otherwise the choice is arbitrary.

\section{Limitations}
\subsection{Placeholders cannot be used as terminals}
Suppose a user tries to define the syntax of $\lambda$-terms:
\begin{align*}
    M, N &\Coloneqq x \alt (\lambda x. M) \alt (MN) \\
    x &\Coloneqq x \alt y \alt z
\end{align*}
The second rule is necessary as the system does not ``know'' $x$ in the first rule is a variable: if a character is not a placeholder, it is treated as a \lstinline{Terminal}. However, contrary to what the user expects, the $x$ in the first alternative of the second rule is interpreted as a reference to its own rule. Since it is the first token of the alternative, the alternative is considered left-recursive by the parsing algorithm and causes the parsing algorithm to throw an error. The current workaround to this issue is as follows:
\begin{align*}
    M, N &\Coloneqq var \alt (\lambda var. M) \alt (MN) \\
    var &\Coloneqq x \alt y \alt z
\end{align*}
The use of $var$ is unconventional and differs from most textbook formulations. However, there is no good way to tell the parsing algorithm to treat $x$ as a literal string instead of a placeholder depending on where it appears. One solution is to treat lowercase placeholders as literal strings if it appears in the definition of its own rule and placeholders otherwise. However, this solution not only makes the system prejudiced against certain inputs, but is also equally problematic. A frustrated user may find the following definition of Curry types to be interpreted rather unexpectedly:
\[
  a, b \Coloneqq \varphi \alt (a \to b)
\]
\chapter{User-inputted terms}
\label{chapter:term}
In this report, a \textit{user-inputted term} refers to an instance or realisation of a syntax rule in the derivation tree. A \textit{user-inputted term} can correspond to a Curry type, a $\lambda$-term, a multiset of propositional formulas, or any syntax rule defined by the user. For example, $((1 \to 2) \to 3)$ is a valid user-inputted term corresponding to the rule $A \Coloneqq 1 \alt 2 \alt 3 \alt (A \to A)$, since it is a valid instance of the rule.

A \textit{user-inputted statement} is a \textit{user-inputted term} corresponding to the syntax rule defining a statement.

A valid derivation tree consists of valid user-inputted statements and rule names. The application must be able to verify whether each user-inputted statement conforms to the syntax rule defining a statement. This chapter describes the parsing algorithm that achieves this goal.

\section{Generating term parsers}
The aim of this step is to generate parsers for parsing terms, given as strings, into a list of abstract syntax trees (ASTs). One parser is generated for every syntax rule. Each parser outputs a list of ASTs. Each AST in the output list corresponds to a token in the parsed definition of the syntax rule.

\subsection{Abstract syntax trees (ASTs)}
The ASTs are defined as follows:
\begin{lstlisting}
    class TerminalAST { constructor(readonly value: string) {} }

    class NonTerminalAST {
        constructor(readonly index: number, readonly children: AST[]) {} }

    class MultisetAST { constructor(readonly elements: AST[][]) {} }

    type AST = TerminalAST | NonTerminalAST | MultisetAST;
\end{lstlisting}

\subsection{Procedure}
\label{term:procedure}
A list of \textit{delayed parsers} is generated using the \lstinline{later} combinator. A \textit{delayed parser}\footnote{In \lstinline{parjs}, a delayed parser has type \lstinline{DelayedParjser<T>}.} is a parser that does not carry any logic and must be initialised before being used for parsing \cite{parjs}. A delayed parser can be chained and combined with other parsers like a normal parser. There are two main benefits of generating delayed parsers and later initialising them:
\begin{itemize}
    \item It is possible to generate parsers for recursive rules. When generating the parser for a rule containing references to itself, the parser must be incomplete at any point of reference since the reference is part of the definition. A similar logic applies to mutually recursive rules.
    \item There is no need to devise an order in which the parsers for each rule is generated (e.g. by topologically sorting the references, such that every rule only references ``earlier'' rules whose parsers would have been all initialised). In fact, such an order or topological sort does not exist when there is recursion, since the reference graph contains cycles.
\end{itemize}
A term parser is generated by iterating over the tokens in every alternative of of the rule and examining the type of the token:
\begin{enumerate}
    \item \lstinline{Terminal}: parse the string verbatim.
    \item \lstinline{NonTerminal}: use the (delayed) parser corresponding to the rule number of the \lstinline{NonTerminal}.
    \item \lstinline{Multiset}: either parse the string \lstinline{\varnothing} (corresponding to the empty set) or a comma-separated list of one or more elements. To generate the parser for a multiset element, a parser is generated for each token of the element definition. Each of these parsers are chained together using the \lstinline{then} combinator.
\end{enumerate}
\section{Inference rules}
\label{section:inference}
Inference rules define the valid proof steps in a proof system. The application supports inference rules of the form
\[
    \Inf[\text{Rule name}]{\text{Zero or more premises}}{\text{Conclusion}}
\]
which lets us derive the conclusion if all the premises hold. Inference rules can use metavariables to match any concrete term. Inference rules can also destructure concrete terms and reject concrete terms that cannot be destructured according to the inference rule. The premises and conclusion of an inference rule are specialised statements that carry structural information and should be parsed more like concrete terms than syntax rules. These specialised statements are called \textit{abstract statements}, consisting of \textit{abstract terms}, which are either metavariables or destructurings of concrete terms.

Like syntax rules, inference rules may be added, deleted, or replaced by the user. \projectname{} provides an interface for modifying inference rules, checks whether the inference rules are defined sensibly, and verifies whether the user has applied an inference rule correctly.

\Cref{inference:syntax} defines the syntax in which a user expresses an inference rule in the application, describes the concept of \textit{abstract terms}, and contrasts it with \textit{concrete terms} introduced in the previous chapter. \Cref{inference:parsing} describes the procedure for transforming an arbitrary set of inference rules into intermediate internal representations, in preparation for verifying applications of the inference rules in the derivation tree.

\subsection{Syntax of inference rules}
\label{inference:syntax}
An inference rule consists of:
\begin{itemize}
    \item A \textbf{name},
    \item One \textbf{conclusion} following the structure of a statement, and
    \item Zero or more \textbf{premises}, each following the structure of a statement
\end{itemize}
\projectname{} only supports \textit{localised} inference rules: checking whether an inference rule is applied correctly only requires information immediately above and below the step of applying the inference rule, but not any information further up or down the derivation tree.

\subsubsection{Abstract terms and statements}
In this report, an \textit{abstract statement} refers to one of the premises or the conclusion of an inference rule. Abstract statements can contain metavariables, while concrete statements cannot. Consider the following syntax rule:
\[
    A, B \Coloneqq \varphi \alt (A \to B)
\]
In a derivation, the user can only input concrete terms like $\varphi$, $(\varphi \to \varphi)$, and $((\varphi \to \varphi) \to \varphi)$. In an inference rule, however, the user can input abstract terms like $A$, $B$, $(A \to B)$, $((A \to B) \to A)$, and even $((B \to \varphi) \to B)$. An abstract term is either a metavariable of the corresponding rule, or partially expanded versions of its definition. For the syntax rule above, the following grammar specifies the set of abstract terms:
\[
    A_{abs}, B_{abs} \Coloneqq \varphi \alt (A_{abs} \to B_{abs}) \alt \text{``$A$''} \alt \text{``$B$''}
\]

All valid concrete terms are valid abstract terms.

\paragraph{Metavariables in syntax rule definitions vs. abstract statements}
Recall from \Cref{syntax:syntax} that in syntax rule definitions, metavariables for a rule can be used interchangeably. The following syntax rule is semantically identical to the example above:
\[
    A, B \Coloneqq \varphi \alt (A \to A)
\]
Even though $A$ appears twice in different positions of the same alternative, the two concrete terms that take their positions can be different. For example, $(\varphi \to (\varphi \to \varphi))$ is a valid instance of both versions of the syntax rule. However, the abstract term $(A \to A)$ is compatible with a concrete term if and only if the two user-inputted sub-terms that take the positions of the two occurrences of $A$ are identical. In this case, $(\varphi \to (\varphi \to \varphi))$ is incompatible with the abstract term $(A \to A)$ because the sub-terms $\varphi$ and $(\varphi \to \varphi)$ are different. The process of matching abstract terms with concrete terms is explained in detail in \Cref{chapter:checking}.

\paragraph{Multisets in inference rules}
\label{inference:multisets}
Consider the following syntax rules:
\begin{align*}
    \Gamma, \Delta &\Coloneqq \{ A \} \\
    A, B &\Coloneqq \varphi \alt (A \to B)
\end{align*}
On its own, $\Gamma$ represents a complete multiset. Now, consider the term $\Gamma, A$. In this term, $\Gamma$ represents a multiset with one element removed. The removed element is given the name $A$. Since an element can be removed, the user-inputted multiset must contain at least one element: trying to match the concrete term $\varnothing$ against the abstract term $\Gamma, A$ results in an error. Note that $\Gamma$ can be empty. Analogously, it is possible to:
\begin{itemize}
    \item Match multiple multiset elements, such as $\Gamma, A, B$ and $\Gamma, A, A, A$.
    \item Match multiple multisets, such as $\Gamma, \Delta$.
    \item Match multiple multisets and multiset elements, such as $\Gamma, \Delta, A, B$.
\end{itemize}

It is also possible to match multiset elements with specific structures. The following are all valid abstract terms:
\[
    \Gamma, (A \to B) \qquad \Gamma, (A \to (A \to A)) \qquad \Gamma, ((\varphi \to B) \to A)
\]
since $(A \to B)$, $(A \to (A \to A))$, and $((\varphi \to B) \to A)$ are all valid expansions of the definition of the second syntax rule, of which $A$ is a metavariable.

The order of the multiset metavariables and multiset element metavariables does not matter. This means a multiset term is identical to any of its permutations. The following terms are all identical:
\[
    \Gamma, \Delta, A, B \qquad \Gamma, B, \Delta, A \qquad A, B, \Delta, \Gamma \qquad \Delta, B, A, \Gamma \qquad \ldots
\]

Note that a metavariables (in this case, $\Gamma$ or $\Delta$) can represent a multiset if and only if at least one of its alternatives only consists of a multiset and no other tokens. For example, if the first syntax rule in the example above is changed to
\[
    \Gamma, \Delta \Coloneqq \{ A \}: B
\]
instead, things like $\Gamma, A$ and $\Gamma, \Delta, A$ are not valid abstract terms for the modified syntax rule.

\subsection{Parsing inference rules}
\label{inference:parsing}
The aim of this step is to generate parsers for parsing an abstract statement into a list of \lstinline{Matchable} tokens.

\subsubsection{\texorpdfstring{\lstinline{Matchable}}{Matchable} tokens}
\lstinline{Matchable} tokens are to inference rules like ASTs are to terms and \lstinline{Token}s are to syntax rules: they represent the structure of an abstract statement. \lstinline{Matchable} tokens are defined as follows:
\begin{itemize}
    \item \begin{lstlisting}[style=ds]
        class MatchableTerminal { constructor(readonly value: string) {} }
    \end{lstlisting}
    Match the string stored in the \lstinline{value} field.
    \item \begin{lstlisting}[style=ds]
        class Name {
            constructor(readonly index: number, readonly name: string) {}
        }
    \end{lstlisting}
    Assign exactly one AST to \lstinline{name} and ensure this assignment is compatible with all other occurrences of the same \lstinline{name} in other statements of the same inference rule. The AST must represent an instance of the syntax rule at \lstinline{index}\footnote{The \lstinline{index} field can be deduced from the \lstinline{name} field since the metavariables are unique across all syntax rules. The \lstinline{index} field exists merely to simplify bookkeeping.}.
    \item \begin{lstlisting}[style=ds]
        class MatchableNonTerminal {
            constructor(readonly index: number, readonly children: Matchable[]) {}
        }
    \end{lstlisting}
    Match an abstract term corresponding to the syntax rule at \lstinline{index}. The structure of the abstract term is given by the \lstinline{children} field.
    \item \begin{lstlisting}[style=ds]
        class MultisetElement { constructor(readonly tokens: Matchable[]) {} }

        class MatchableMultiset {
            constructor(
                readonly index: number,
                readonly elements: (Name | MultisetElement)[]
            ) {}
        }
    \end{lstlisting}
    Match a multiset of elements given by the \lstinline{elements} field. A member of \lstinline{elements} can either be a \lstinline{Name} or a \lstinline{MultisetElement}. A \lstinline{Name} refers to a multiset (e.g. $\Gamma$ and $\Delta$ in the example in the previous section) while a \lstinline{MultisetElement} refers to a multiset element that must be matched from the user-inputted multiset term (e.g. $A$ and $(A \to B)$ in the previous section). The \lstinline{elements} field is represented as an (ordered) array instead of an unordered set purely for simpler bookkeeping in the matching algorithm.
\end{itemize}
Finally, the \lstinline{Matchable} type is defined as
\begin{center}
    \lstinline{type Matchable = MatchableTerminal | Name | MatchableNonTerminal | MatchableMultiset;}
\end{center}
\subsubsection{Procedure}
Parsing abstract statements is extremely similar to parsing terms, as an abstract statement can be thought of as an instance of a statement in a different context. As with generating term parsers, a delayed parser is created  for every syntax rule using the \lstinline{later} combinator. Each of these delayed parsers is instantiated by iterating over the tokens in every alternative of the syntax rule and examining the type of the token:
\begin{itemize}
    \item \lstinline{Terminal}: parse the string verbatim and wrap the result in a \lstinline{MatchableTerminal} token.
    \item \lstinline{NonTerminal}: either parse a metavariable of the rule corresponding to the rule number (i.e. the \lstinline{index} field of the \lstinline{NonTerminal}), or an expanded version of the definition of the syntax rule using the (delayed) parser corresponding to the rule. The former produces a \lstinline{Name} token while the latter produces a \lstinline{MatchableNonTerminal} token.
    \item \lstinline{Multiset}: either parse \lstinline{\varnothing} (corresponding to the empty set) or a comma-separated list of multiset elements, as described in \Cref{inference:multisets}. If metavariables for multisets are applicable here (i.e. the \lstinline{Multiset} token is the only token of an alternative of the syntax rule definition, and the syntax rule does not define a statement), the comma-separated list can contain both individual multiset elements and metavariables for multisets. Since the \lstinline{Multiset} token does not store the rule number of the rule it is part of, the rule number is passed as an argument to the function which generates inference rule parsers.
\end{itemize}
\chapter{Rule checking}
\label{chapter:checking}
As the user builds the derivation tree, the application checks whether the inference rules are applied correctly and whether the derivation is complete and correct. Since the application only supports \textit{localised} inference rules, it suffices to check at every dividing line of the derivation tree, whether the user-inputted premises immediately above the line and the user-inputted conclusion immediately below the line are compatible with the inference rule specified by the user-inputted rule name. A derivation is complete and correct if and only if all inference rules in the derivation are compatible with the user input.

An inference rule is compatible with the user input if and only if there is a mapping of names (i.e. metavariables) to ASTs such that replacing all the names in the inference rule with the corresponding ASTs according to the mapping gives the user input. There can be many such mappings for a given inference rule and user input. Since the inference rules are localised, each mapping only applies to one application of an inference rule and need not apply to the rest of the derivation.

\Cref{checking:matching} describes the procedure for obtaining a mapping of names to ASTs given a user-inputted statement and the structure of the corresponding inference rule statement. \Cref{checking:verifying} describes how the procedure described in \Cref{checking:matching} is used to check whether all user-inputted premises and the user-inputted conclusion are compatible with the inference rule.

\section{Matching}
\label{checking:matching}
The aim of this step is to produce a mapping of names to ASTs. Matching is done individually to every inference rule statement.

\subsection{Inputs and outputs of the matching algorithm}
The matching algorithm takes in two mappings as arguments and modifies them in place. The two mappings are:
\begin{itemize}
    \item \lstinline{names}: a mapping where each entry maps a name (as a string) to an AST. A name \textit{must} be mapped to the AST according to \lstinline{names} if the inference rule is applied correctly.
    \item \lstinline{unmatchedPossibilities}: a mapping where each entry maps a name (as a string) that does not appear in \lstinline{names} to a set of ASTs. A name \textit{can} be mapped to any AST in the set according to \lstinline{unmatchedPossibilities} if the inference rule is applied correctly, yet it is not known which of the ASTs the name will ultimately end up getting mapped to at the time of matching.
\end{itemize}
The matching algorithm checks whether the mappings deduced from the user input and the structure of the inference rule statement are consistent with \lstinline{names} and \lstinline{unmatchedPossibilities}, and throws an error whenever it first finds an inconsistency.

\subsection{Procedure outline}
\label{matching:procedure}
\begin{enumerate}
    \item Parse the user-inputted statement into a list of ASTs.
    \item Check the list of ASTs has the same length as the list of \lstinline{Matchable} tokens in the inference rule statement.
    \item Pair up each AST with the corresponding \lstinline{Matchable} token and recursively match each pair.
    \item Repeat step 3 until no new names are added to \lstinline{names}.
\end{enumerate}
Step 3 is repeated because the mappings generated by matching the ASTs further down the list may imply mappings that could not be deduced in an earlier pass. Consider the conclusion of the action rule in natural deduction $\Gamma, A \vdash A$ and the user input $x, y \vdash y$. Suppose the ASTs and \lstinline{Matchable} tokens are matched from left to right starting with \lstinline{names} empty. Step 3 above can be broken down into the following smaller steps:
\begin{enumerate}
    \item Match $x, y$ with $\Gamma, A$. Since there are no existing mappings, $A$ can either be mapped to $x$ or $y$ and $\Gamma$ can either be mapped to the multiset containing $y$ or the multiset containing $x$.
    \item Match $\vdash$ with $\vdash$.
    \item Match $y$ with $A$. Clearly, $A$ must be mapped to $y$. This mapping is stored in \lstinline{names}.
\end{enumerate}
In the second iteration of step 3, when trying to match $x, y$ with $\Gamma, A$ again, the mapping of $A$ to $y$ implies the mapping of $\Gamma$ to the multiset containing $x$. This deduction could not be made in the first iteration.

The subsequent subsections explain step 3 in greater detail. From this point onwards, \lstinline{ast} and \lstinline{token} refer to the AST and the \lstinline{Matchable} token that are being matched, respectively.

\subsection{Matching a \texorpdfstring{\lstinline{TerminalAST}}{TerminalAST}}
Matching succeeds if and only if \lstinline{token} is a \lstinline{MatchableTerminal} containing the same string as \lstinline{ast}.

\subsection{Matching a \texorpdfstring{\lstinline{NonTerminalAST}}{NonTerminalAST}}
If \lstinline{token} is neither a \lstinline{Name} or a \lstinline{MatchableNonTerminal}, matching fails. Otherwise, if \lstinline{token} is a:
\begin{itemize}
    \item \lstinline{Name}: if the name is in \lstinline{names}, matching succeeds if and only if the mapped AST (i.e. \lstinline{names[name]}) is the same as \lstinline{ast}. Otherwise, create a new mapping from the name to \lstinline{ast}.
    \item \lstinline{MatchableNonTerminal}: if either of the following does not hold, matching fails:
    \begin{itemize}
        \item \lstinline{ast.index} is equal to \lstinline{token.index}. In other words, the rule number of \lstinline{ast} is the same as the rule number of \lstinline{token}.
        \item \lstinline{ast.children} has the same length as \lstinline{token.children}. In other words, there are as many children ASTs representing the structure of \lstinline{ast} as there are children \lstinline{Matchable} tokens representing the structure of \lstinline{token}.
    \end{itemize}
    If both of the above are true, pair up each child AST with the corresponding child \lstinline{Matchable} token and recursively match each pair, like in step 3 of \Cref{matching:procedure}.
\end{itemize}
\subsection{Matching a \texorpdfstring{\lstinline{MultisetAST}}{MultisetAST}}
If \lstinline{token} is not a \lstinline{MatchableMultiset}, matching fails. Otherwise, split up the \lstinline{elements} field of \lstinline{token} (which has type \lstinline{(Name | MultisetElement)[]}) into \lstinline{Name}s and \lstinline{MultisetElement}s.

For every \lstinline{Name} in \lstinline{token.elements}:
\begin{itemize}
    \item If the name is in \lstinline{names} (i.e. it is already mapped to an AST), check the mapped AST (i.e. \lstinline{names[name]}) has type \lstinline{MultisetAST}. Remove all elements in the mapped AST from \lstinline{ast.elements}. If any element in the mapped AST cannot be found in \lstinline{ast.elements}, throw an error.
    \item Otherwise, matching is postponed.

\end{itemize}

For every \lstinline{MultisetElement} in \lstinline{token.elements}, find all \textit{actual elements} (i.e. an element of \lstinline{ast.elements} with type \lstinline{AST[]}) that are compatible with the \lstinline{MultisetElement}. The procedure for finding compatible actual elements will be described in the next subsection.
\begin{itemize}
    \item If there are no matches, i.e. none of the actual elements are compatible with the \lstinline{MultisetElement}, throw an error.
    \item If there is exactly one match, i.e. exactly one actual element is compatible with the \lstinline{MultisetElement}, match the actual element with the \lstinline{MultisetElement}.
    \item If there are at least two matches, postpone the matching and update \lstinline{unmatchedPossibilities} as follows:
    \begin{enumerate}
        \item For every actual element that is compatible with the \lstinline{MultisetElement}, assume the \lstinline{MultisetElement} should be mapped to the actual element and generate a mapping where each entry maps a name in the \lstinline{MultisetElement} to an AST.
        \item For every name in the \lstinline{MultisetElement} but not in \lstinline{names} (i.e. it has not been mapped to an AST for certain), if it exists in \lstinline{unmatchedPossibilities}, update the corresponding entry as follows:
        \begin{center}
            \lstinline|unmatchedPossibilities[name] = { names1[name], names2[name], ... } $\cap$ unmatchedPossibilities[name]|
        \end{center}
        where \lstinline{names1}, \lstinline{names2}, etc. are the mappings generated from the previous step. Otherwise, if the name does not exist in \lstinline{unmatchedPossibilities}, create a new entry in \lstinline{unmatchedPossibilities} as follows:
        \begin{center}
            \lstinline|unmatchedPossibilities[name] = { names1[name], names2[name], ... }|
        \end{center}
    \end{enumerate}
\end{itemize}

\subsubsection{Finding actual elements compatible with a \texorpdfstring{\lstinline{MultisetElement}}{MultisetElement}}
Create an empty set of possibilities. For every actual element remaining in the user-inputted multiset (i.e. after all multiset names are matched and the elements in the mapped AST are removed from \lstinline{ast.elements}):
\begin{enumerate}
    \item Check the actual element (which has type \lstinline{AST[]}) has the same length as the \lstinline{children} field of the \lstinline{MultisetElement}. In other words, check there are as many children ASTs representing the structure of the actual element as there are children \lstinline{Matchable} tokens representing the structure of the \lstinline{MultisetElement}.
    \item Check the actual element is not already in the set of possibilities. Recall the aim of finding compatible actual elements is to determine whether the \lstinline{MultisetElement} can be matched with an actual element with certainty. If the \lstinline{MultisetElement} is compatible with multiple identical actual elements, there is no difference between matching the \lstinline{MultisetElement} with one of the duplicates and matching it with any of the other duplicates, since any modifications to \lstinline{names} will be identical.
    \item Make a shallow copy of \lstinline{names}. The copy is a new mapping from names (as strings) to ASTs where the ASTs share the same references as in \lstinline{names}. Assigning a new name or removing an existing name of the shallow copy will not affect \lstinline{names}, but modifying any of the ASTs will cause the same changes in \lstinline{names}. A shallow copy is acceptable since the matching algorithm only adds names to \lstinline{names} and never modifies the ASTs in place.
    \item Pair up each AST in the actual element with the corresponding \lstinline{Matchable} token of the \lstinline{MultisetElement}. Try to match each (AST, \lstinline{Matchable}) pair using the shallow copy of \lstinline{names}. A shallow copy of \lstinline{names} instead of \lstinline{names} is used because the matching may fail. The matching algorithm assumes the given AST \lstinline{must} be compatible with the given \lstinline{Matchable} token, so any error thrown indicates the user input is incorrect. This assumption is reasonable when matching at the top level, i.e. matching the user input with the structure of the inference rule statement, but less so when matching is used as a means to explore possibilities.
    \item If the matching succeeds, add the actual element to the set of possibilities.
\end{enumerate}

\section{Verifying}
\label{checking:verifying}
The aim of this step is to verify whether an inference rule is applied correctly by the user.

\subsection{Procedure outline}
Given a string representing the user-inputted conclusion, a list of strings representing the user-inputted premises, and the parsed inference rule, the verification algorithm is as follows:
\begin{enumerate}
    \item Check the user has given the same number of premises as defined in the inference rule.
    \item Repeat the following steps until no new names are added:
    \begin{enumerate}
        \item Match the user-inputted conclusion with the structure of the conclusion as defined by the inference rule.
        \item Match each user-inputted premise with the structure of the corresponding premise as defined by the inference rule.
    \end{enumerate}
    Here, the order of matching the abstract statements is arbitrary. Without analysing the structure of the inference rule, any matching order~ when given an arbitrary inference rule works equally well, since the matching algorithm is idempotent. For example, it is equally effective to match the premises before the conclusion or to match the premises in a random order. The matching steps are repeated for a similar reason as repeating step 3 in \Cref{matching:procedure}: matching a later inference rule statement may give more information for generating additional mappings when re-matching an earlier inference rule statement that cannot be generated on an earlier iteration.
    \item If \lstinline{unmatchedPossibilities} is empty, verification succeeds if and only if none of the matching steps throws any errors.
    \item Otherwise, recursively explore all possible assignments of names to ASTs. Verification succeeds if and only if there is at least one assignment that does not result in any error when parsing.
\end{enumerate}
\chapter{Evaluation}
The application should be both reliable and intuitive to use. Here, reliability means the application correctly verifies a derivation if and only if the derivation is correct. If the application is not reliable, users may want to double-check their derivation by other means. If the application is not intuitive, users may need to spend more time learning how to use it. In both cases, users experience more friction and waste time on using the application, making the application less appealing than its humble competitor: pen and paper.

\Cref{evaluation:correctness} describes how the algorithms are tested and refined to support a large range of proof systems. \Cref{evaluation:ux} discusses how user testing is done to evaluate the intuitiveness of the application, the findings of user testing, and the improvements made to address these findings.

\section{Correctness of algorithms}
\label{evaluation:correctness}
The correctness of the parsing, matching, and verification algorithms are verified using unit tests. Some test cases are based on subsets of the pre-defined proof systems---natural deduction, Curry type assignment for the lambda calculus, and the sequent calculus system LK---while other test cases are based on variously modified rules from these systems and designed to catch edge cases. Some test cases based on the lambda calculus are taken from the course notes for the Type Systems module at Imperial \cite{van-bakel:2022}. Some test cases based on the system LK are taken from a set of online logic notes \cite{sequent}.

There is a high degree of confidence in the matching and verification algorithms, since they are based on custom data structures independent of the user input. To identify further edge cases in the parsing algorithm and determine how well the parsing algorithm adapts to a wide range of user inputs, the parsing algorithm is tested against extensions of the pre-defined systems and brand new proof systems.

\subsection{Lambda calculus with pairs}
\label{evaluation:lambda-pairs}
The syntaxes of $\lambda$-terms and Curry types are extended with the following constructors \cite{van-bakel:2022}:
\begin{align*}
    M, N &\Coloneqq \ldots \alt \langle M, N \rangle \alt \textsf{left}(M) \alt \textsf{right}(M) \\
    A, B &\Coloneqq \ldots \alt (A \times B)
\end{align*}
The three new constructors for $\lambda$-terms are accompanied with the following extensions to the type assignment rules:
\[
    (\textit{Pair}): \frac{\Gamma \vdash M: A \quad \Gamma \vdash N: B}{\Gamma \vdash \langle M, N \rangle: (A \times B)} \quad (\textsf{left}): \frac{\Gamma \vdash M: (A \times B)}{\Gamma \vdash \textsf{left}(M): A} \quad (\textsf{right}): \frac{\Gamma \vdash M: (A \times B)}{\Gamma \vdash \textsf{right}(M): B}
\]
Although the user can type ``left'' and ``right'' for the two constructors, they will be rendered in math mode in \LaTeX{} as $left$ and $right$, which do not look the nicest. Prior to testing with the lambda calculus with pairs, the parsing algorithm can only handle \LaTeX{} commands that do not take any arguments, such as ``\textbackslash Gamma'' for $\Gamma$ and ``\textbackslash varphi'' for $\varphi$. The lambda calculus with pairs motivates extending the parsing algorithm to parse \LaTeX{} commands as literal strings. In this case, ``\textbackslash textsf{left}(M)'' should be rendered as $\textsf{left}(M)$ and parsed into \lstinline{Token}s as
\begin{center}
    \lstinline|[Terminal("\textsf{left}"), Terminal("("), NonTerminal(...), Terminal(")")]|
\end{center}
A correct derivation using all three newly added constructors and type assignment rules is shown in \Cref{fig:lambda-with-pairs}.
\begin{figure}[!htbp]
    \centering
    \includegraphics[width=0.7\textwidth]{evaluation/lambda-with-pairs.png}
    \caption{A correct derivation }
    \label{fig:lambda-with-pairs}
\end{figure}

\subsection{\texorpdfstring{\lbm{}}{Lambda bar mu}}
The calculus \lbm{} as presented by Curien and Herbelin \cite{curien-herbelin:2000} defines three types of terms \cite{van-bakel:2024}:
\begin{align*}
    c &\Coloneqq \langle t | e \rangle &(\textit{commands}) \\
    t &\Coloneqq x \alt (\lambda x. t) \alt (\mu \beta. c) &(\textit{terms}) \\
    e &\Coloneqq \alpha \alt (t \cdot e) \alt (\tilde{\mu}x. c) &(\textit{environments})
\end{align*}
where $x$ can be any symbol from an infinite list of term variables $a, b, c, \ldots, x, y, z \ldots$. \todo{What are $\alpha$ and $\beta$?} \lbm{} defines three kinds of statements, each typing a different syntactic category:
\begin{align*}
    \text{Statement} \Coloneqq{} &c: \Gamma \vdash \Delta &(\textit{commands}) \\
    |\  &\Gamma \vdash t: A \alt \Delta &(\textit{terms}) \\
    |\  &\Gamma \alt e: A \vdash \Delta &(\textit{environments})
\end{align*}
where $A$ represents a Curry type, while $\Gamma$ and $\Delta$ represent a set of variable type assignments. The type assignment rules are defined as follows:
\[
    (\textit{cut}): \frac{\Gamma \vdash t: A \alt \Delta \quad \Gamma \alt e: A \vdash \Delta}{\langle t|e \rangle: \Gamma \vdash \Delta}
\]
\vspace{-24pt}
\begin{center}
    \begin{minipage}{.4\textwidth}
        \begingroup
        \addtolength{\jot}{1em}
        \begin{align*}
            (Ax_R)&: \frac{}{\Gamma, x: A \vdash x: A \alt \Delta} \\
            (\arr R)&: \frac{\Gamma, x: A \vdash t: B \alt \Delta}{\Gamma \vdash (\lambda x. t): (A \to B) \alt \Delta} \\
            (\mu)&: \frac{c: \Gamma \vdash \alpha: A, \Delta}{\Gamma \vdash (\mu \alpha. c): A \alt \Delta}
        \end{align*}
        \endgroup
    \end{minipage}%
    \begin{minipage}{.4\textwidth}
        \begingroup
        \addtolength{\jot}{1em}
        \begin{align*}
            (Ax_L)&: \frac{}{\Gamma \alt \alpha: A \vdash \alpha: A, \Delta} \\
            (\arr L)&: \frac{\Gamma \vdash t: A \alt \Delta \quad \Gamma \alt e: B \vdash \Delta}{\Gamma \alt (t \cdot e): (A \to B) \vdash \Delta} \\
            (\tilde{\mu})&: \frac{c: \Gamma, x: A \vdash \Delta}{\Gamma \alt (\tilde{\mu}x. c): A \vdash \Delta}
        \end{align*}
        \endgroup
    \end{minipage}
\end{center}

\section{User experience}
\label{evaluation:ux}
Three students who took the Type Systems module in autumn 2024 were invited to test the web application. User testing was conducted one-on-one over a video call. The users were asked to share their screens so that their interactions with the web application could be observed and analysed.

At the beginning of each session, the users were all told two things: that the web application supported \LaTeX{} input, and that full bracketing was necessary. Afterwards, they were given minimal guidance and navigated the web application mostly on their own, except when they did not know certain \LaTeX{} syntax or had questions about the application.

\subsection{User testing tasks}
The users were asked to complete three tasks: the first task involves deriving a conclusion in the Curry type assignment system for the lambda calculus, the second task involves extending the lambda calculus and the Curry type assignment system, and the third task involves deriving a conclusion in the system \textsc{lk}.

\subsubsection{Curry type assignment system for the lambda calculus}
In this task, the user is asked to derive the conclusion
\[
    \varnothing \vdash ((\lambda x. x)(\lambda y. y))
\]
in the Curry type assignment system using the web application. This particular conclusion is chosen because it is fairly simple and its derivation requires all three type assignment rules $(Ax)$, $(\arr I)$, and $(\arr E)$.

This task serves as a gentle introduction to the web application. The user only needs to focus on navigating the derivation building part of the web application and not other features, e.g. the syntax and inference rule editors. This task also serves as a quick refresher on building derivation trees, since the user testing was done around half a year after the Type Systems module had ended.

The goal of this task is to evaluate the intuitiveness of the derivation building part of the web application. In particular, the evaluation focuses on how confident the users can input conclusions, rule names, and add premises without guidance, as well as the usefulness of the error messages when the user provides an incorrect derivation.

\subsubsection{Lambda calculus with pairs}
In this task, the user is asked to extend the syntax of $\lambda$-terms and type assignment rules with pairs, as described in \Cref{evaluation:lambda-pairs}. Afterwards, the user is asked to derive the following conclusion:
\[
    x: (1 \times 2) \vdash \langle \textsf{left}(x), \textsf{right}(x) \rangle: (1 \times 2)
\]
This particular conclusion is chosen because it is fairly simple and its derivation uses all three newly added type assignment rules.

The goal of this task is to evaluate the intuitiveness of the syntax and inference rule editors. In particular, the evaluation focuses on how the users interact with the user interface to check the correctness of their definitions and the usefulness of the error messages and warnings.

\subsubsection{Sequent calculus system \textsc{lk}}
In this task, the user is asked to derive the following conclusion in the system \textsc{lk}:
\[
    ((x \lor y) \to z) \vdash (x \to z)
\]
This particular conclusion is chosen because its derivation requires the rule
\[
    (\arr L): \frac{\Gamma \vdash A, \Delta \quad \Sigma, B \vdash \Pi}{\Gamma, \Sigma, (A \rightarrow B) \vdash \Delta, \Pi}
\]
which may appear quite complicated to users unfamiliar with the system \textsc{lk}, as they would need to pattern match numerous placeholders.

The goal of this task is to evaluate the usefulness of the error messages in the derivation tree and the rule viewers in guiding the users to build a correct derivation in an unfamiliar proof system.

\subsection{User testing findings}
\subsubsection{Redundant parentheses around rule names}
One user asked whether parentheses were needed around the rule names, since there was no indication from the rule name inputs and parenthesised rule names are normally expected in handwritten derivations. The user only realised parentheses were not necessary after he typed the rule name with parentheses, clicked away from the input, and saw the \LaTeX{} display added an extra set of parentheses, as in \Cref{fig:evaluation:rule-name}.

% https://tex.stackexchange.com/questions/218378/forcing-subfigures-to-have-same-height-and-take-overall-x-of-linewidth-in-latex
\begin{figure}[!htbp]
    \sbox\twosubbox{%
    \resizebox{\dimexpr\textwidth-1em}{!}{%
        \includegraphics[height=3cm]{evaluation/rule-name-input.png}%
        \includegraphics[height=3cm]{evaluation/rule-name-latex.png}%
    }%
    }
    \setlength{\twosubht}{\ht\twosubbox}

    \centering

    \subcaptionbox{There are no parentheses around the rule name input.}{%
        \includegraphics[height=\twosubht]{evaluation/rule-name-input.png}%
    }\quad
    \subcaptionbox{The \LaTeX{} display adds parentheses around the rule name. Note the generic ``undefined rule'' error provided.}{%
        \includegraphics[height=\twosubht]{evaluation/rule-name-latex.png}%
    }
    \caption{It was unclear whether parentheses were needed around the rule name.}
    \label{fig:evaluation:rule-name}
\end{figure}

The solution is to both add an entry to the syntax guide and to display a specific error message when superfluous parentheses were added around the rule name input.

\subsubsection{Redundant parentheses around multiset terms}
All users asked whether parentheses were needed around multisets and individual multiset elements, such as $x:1, y:2$ versus $(x:1), (y:2)$ and $(x:1, y:2)$. The question may be common because in the Type Systems module and logic modules in the previous years, students were taught the bracketing conventions and operator associativity of $\lambda$-terms, Curry types, and logical connectives, but never explicitly the bracketing conventions of multisets, which were only introduced in the Type Systems module. Indeed, there was no need to formally introduce bracketing conventions for multisets in the Type Systems module since all derivations were handwritten and there was never any ambiguity when writing multisets in any of the proof systems introduced in the module.

\subsubsection{Unclear indication of correct derivations for colour-blind users}
One user with red-green colour blindness said it took multiple attempts to confirm whether his derivation was correct because the change in background colour from white to pale green appeared quite subtle to him. A comparison between the pale green background colour and a simulated version of what a green-blind user sees is shown in \Cref{fig:evaluation:colour-blind}.

\begin{figure}[!htbp]
    \centering
    \begin{subfigure}{.48\textwidth}
        \centering
        \includegraphics[width=\textwidth]{evaluation/background-normal.png}
        \caption{Pale green background colour}
    \end{subfigure}%
    \quad
    \begin{subfigure}{.48\textwidth}
        \centering
        \includegraphics[width=\textwidth]{evaluation/background-colour-blind.jpg}
        \caption{Simulated view of a green-blind user}
    \end{subfigure}
    \caption{What a green-blind person sees when the background colour changes to pale green}
    \label{fig:evaluation:colour-blind}
\end{figure}

The solution is to display additional visual cues, such as a toast, that alert the user when their derivation is correct.

\subsubsection{Mistakenly typing a statement into a rule name input}
Two users mistakenly typed the conclusion of the second premise of a rule into the input for the rule of the first premise, as illustrated in \Cref{fig:evaluation:wrong-premise-input}.

\begin{figure}[!htbp]
    \centering
    \begin{subfigure}{\textwidth}
        \centering
        \includegraphics[height=2.5cm]{evaluation/premise-input-wrong.png}
        \caption{The user typed the conclusion instead of the rule name.}
    \end{subfigure}%
    \newline
    \begin{subfigure}{\textwidth}
        \centering
        \includegraphics[height=2.5cm]{evaluation/premise-input-correct.png}
        \caption{The user should have added a new premise before typing the conclusion.}
    \end{subfigure}
    \caption{Two users mistakenly typed the conclusion of the second premise of a rule into the input for the rule of the first premise.}
    \label{fig:evaluation:wrong-premise-input}
\end{figure}

At the time of testing, there were only placeholders for the conclusion and rule name inputs at the root of the tree: the expectation was that once users input the first conclusion and rule name, they should be able to navigate the rest of the derivation tree without additional guidance. However, user testing showed otherwise. The solution was to add placeholders for all inputs in the derivation tree to indicate whether it should be a conclusion or a rule name.

\subsubsection{Displaying parsing errors before rules are specified}
When completing the first user testing task, one user inputted all conclusions of the tree before inputting the rule names. Since the web application only displayed errors when the user has specified the rule at the time of testing, the user received no feedback on syntactically incorrect inputs when inputting the conclusions and only received multiple error messages once he started typing the rule names.

The solution is to display two types of errors: general errors regardless of whether the user has specified a rule, and additional specific errors when the user has specified a rule. General errors are displayed when the user inputs a syntactically incorrect statement. Specific errors are displayed when the user inputs a syntactically correct statement which is not compatible with the structure of the corresponding inference rule statement, such as when a multiset element with a certain structure cannot be found in the user input, or if some names are incompatible.

\subsubsection{Persisting rules and derivations across reloads}
Two users accidentally closed the tab for the web application during the second task, which required extending the syntax and inference rules. When they opened the application again, the rule definitions and derivation tree were reset since data was not persisted across reloads. Although the users could quickly re-input the rules in this case, it may be more frustrating in other cases when the user accidentally closes the page when defining larger and more complicated proof systems.

The solution is to encode the definitions of the syntax and inference rules as query parameters in the URL and use the \lstinline{localStorage} property \cite{localstorage} to persist derivations across reloads. Though it is possible to store the rule definitions in \lstinline{localStorage} as well, the query parameters are already used to relay information about the rule definitions from the landing page to the derivation builder page. Suppose the rule definitions are only persisted in \lstinline{localStorage}. When the user chooses the pre-defined rules for the lambda calculus on the landing page, the URL query parameters encode this information and are passed to the derivation builder page, which parses the parameters and sets the rule definitions accordingly. If the user changes to those for the sequent calculus in the rule editor, then refreshes the page, the rule definitions and the query parameters become out of sync: the query parameters say the user wants to use the rules for the lambda calculus, yet the persisted rule definitions are actually those of the sequent calculus.
\chapter{Conclusion}
We have developed a web-based proof assistant which lets users define syntax and inference rules in \LaTeX{} and verifies derivations built in the web interface against the user-defined rules. To support these functionalities, we have devised algorithms for parsing syntax rules, inference rules, and user-inputted terms, as well as verifying derivations by matching names to ASTs. \Cref{fig:conclusion:flowchart} summarises the roles of the various algorithms in this project.

\begin{figure}[!htbp]
    \centering
    \includegraphics[width=\textwidth]{conclusion/flowchart.png}
    \caption{A flowchart showing the process of verifying whether an inference rule is applied correctly}
    \label{fig:conclusion:flowchart}
\end{figure}

The algorithms were evaluated and refined by testing against example derivations in natural deduction, the simply typed $\lambda$-calculus, the system \textsc{lk}, the $\lambda$-calculus with pairs and product types, and the calculus \lbm. The user experience of the application was evaluated and refined by asking users to build derivations and observing their behaviour.

\section{Future work}
\subsection{Bracketing conventions and abbreviations}
One of the biggest limitations preventing the current web application from becoming a fully ergonomic and intuitive extension of writing derivations by hand is its inability to parse abbreviated terms, including terms that are not fully bracketed.

When writing derivations by hand, we often drop the outermost parentheses and any parentheses that are implied by associativity conventions. For example, the Curry type $(1 \to (2 \to 3))$ is usually abbreviated as $1 \to 2 \to 3$, since $\to$ is right-associative. One solution may be to treat parentheses as special characters and let the user specify operator associativity whenever the definition contains a pair of parentheses. However, it is difficult to determine the operator within the input enclosed by the parentheses. Suppose the user adds symbols to the definition of Curry types:
\[
    A, B \Coloneqq \varphi \alt (A* \to{} +B)
\]
Unless the application treats $\to$ differently from $*$ and $+$, all three symbols are equally likely to be the principal connective in this case. Therefore, the application must either define a set of special symbols as operators or allow the user to specify the operator.

A better solution is to let the user define rewriting rules that capture both associativity and the convention of dropping the outermost parentheses. In the case of Curry types, the user might specify
\begin{align*}
    (A \to (B \to C)) \quad &\to \quad A \to B \to C \\
    (A \to B) \quad &\to \quad A \to B
\end{align*}
where the $\to$ in the middle means ``can be abbreviated as''. One way to preprocess the input according to the re-writing rules is as follows:
\begin{enumerate}
    \item Generate parsers for each of the abbreviations, i.e. the terms on the right of the $\to$.
    \item Apply the abbreviation parsers from top to bottom and stop as soon as a parser succeeds.
    \item Replace the abbreviation with the fully parenthesised form as specified on the left of the $\to$.
    \item Repeat steps 2 to 3 until no more re-writing rules can be applied.
    \item Proceed to the next part of the input.
    \item Repeat steps 2 to 5 until all input is consumed.
\end{enumerate}
Once the input is fully preprocessed, it should be fully parenthesised and ready to be parsed in the manner described in the previous chapters. If the preprocessing algorithm is properly implemented, it should be able to accept both inference rule statements and user-inputted terms as input.

One issue with this solution is that preprocessing from left to right does not always produce the correct result. Consider the abbreviated term $1 \to 2 \to 3 \to 4$. When applied to this term, the preprocessing algorithm described above proceeds as follows:
\begin{enumerate}
    \item Apply the first re-writing rule to get $(1 \to (2 \to 3)) \to 4$.
    \item Apply the second re-writing rule to get $((1 \to (2 \to 3)) \to 4)$.
    \item Neither of the re-writing rules can be applied, so the algorithm terminates.
\end{enumerate}
However, the output should have been $(1 \to (2 \to (3 \to 4)))$ instead. The preprocessing algorithm would have obtained the correct output if it were applied from right to left, since the first step of the algorithm would have returned $1 \to (2 \to (3 \to 4))$.

One way around this issue is to define a special symbol $\cdots$ meaning ``and so on''. The re-writing rules above can be simplified as the rule
\[
    (A_1 \to (A_2 \to \cdots (A_{n-1} \to A_n))) \quad \to \quad A_1 \to A_2 \to \cdots \to A_n
\]
But then the main difficulty becomes interpreting $\cdots$ correctly. Another way around this issue is to enumerate the re-writing rules up to a certain point and not handling any abbreviations beyond that point. For example, the user may choose to only handle abbreviations for Curry types with up to four arrows:
\begin{align*}
    (A \to (B \to (C \to (D \to E)))) \quad &\to \quad A \to B \to C \to D \to E \\
    (A \to (B \to (C \to D))) \quad &\to \quad A \to B \to C \to D \\
    (A \to (B \to C)) \quad &\to \quad A \to B \to C \\
    (A \to B) \quad &\to \quad A \to B
\end{align*}
The latter workaround also works for abbreviations of abstractions in the lambda calculus, too. In general, the abstraction $(\lambda x_1. (\lambda x_2. \cdots (\lambda x_n. M)))$ can be abbreviated as $\lambda x_1 x_2 \cdots x_n. M$. Ideally, the abbreviation can be expressed by the following re-writing rule:
\[
     (\lambda var_1. (\lambda var_2. \cdots (\lambda var_n. M))) \quad \to \quad \lambda var_1 var_2 \cdots var_n. M
\]
But the same difficulty of interpreting $\cdots$ remains. Instead, enumerating abbreviations up to a certain number of variables would work well:
\begin{align*}
    (\lambda var_1. (\lambda var_2. (\lambda var_3. (\lambda var_4. (\lambda var_5. M))))) \quad &\to \quad \lambda var_1\ var_2\ var_3\ var_4\ var_5. M \\
    &\vdotswithin{\to} \\
    (\lambda var_1. M) \quad &\to \quad \lambda var_1. M
\end{align*}

\subsection{Using and proving lemmas}

% \input{appendix/appendix.tex}

% Handle Dutch surnames with "van", see https://tex.stackexchange.com/questions/40747/bibtex-handling-of-the-dutch-van-name-prefix-with-natbib/40750#40750
\DeclareRobustCommand{\VAN}[3]{#3}
\printbibliography[heading=bibintoc, title={Bibliography}]

\pagebreak
\chapter*{Declarations}
\addcontentsline{toc}{chapter}{Declarations}
\section*{Use of generative AI}
I acknowledge the use of ChatGPT 4.0 (OpenAI, \url{https://chatgpt.com/}) and Claude Sonnet 4 (Anthropic, \url{https://claude.ai}) for finding background information, debugging, and checking whether certain implementation decisions adhere to best practices. I confirm that no content generated by AI has been presented as my own work.

\section*{Ethical considerations}
There are no ethical issues associated with this project.

\section*{Sustainability}
The web application is hosted on Vercel \cite{vercel}. It is run on a serverless architecture, which is only run on-demand and, therefore, more energy-efficient than traditional servers, which run all the time even when there are no pending jobs \cite{vercel-green}.
\section*{Availability of data and materials}
The source code is available as a public repository via \url{https://github.com/justinkeung1018/building-derivations}. The web application is currently available via \url{https://building-derivations.vercel.app}, but this may change in the future. The GitHub repository will contain the updated link.

\end{document}