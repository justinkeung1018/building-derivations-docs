\chapter{Introduction}
Proof assistants, or interactive theorem provers, are software tools that help users create formal proofs. Although proof assistants like Rocq (formerly Coq) are used industrially to prove important mathematical results like the four colour theorem \cite{gonthier:2008}, this report focuses on their applications as educational tools.

Proof assistants like Pandora \cite{pandora:2007, pandora}, Carnap \cite{carnap, carnap:2018}, Holbert \cite{oconnor:2022}, and Logitext \cite{yang:2022} are designed to help students learn logic. They help students learn by providing instant feedback on their proofs. Students at Imperial who used Pandora, a proof assistant for Fitch-style \cite{fitch:1952} natural deduction, were found to approach exam problems more methodically, were less likely to make arbitrary assumptions, and were more precise in their proofs \cite{pandora:2007}.

However, these educational proof assistants share several issues:
\begin{itemize}
    \item Each proof assistant supports a unique syntax. For example, the universal quantifier $\forall$ is represented by \lstinline{A} in Carnap \cite{carnap:systems}, \lstinline{all} in Holbert, and \lstinline{forall} in Logitext \cite{yang:2022}. Students need to learn unfamiliar syntax before they can use the proof assistants, yet this knowledge is not transferrable across proof assistants.
    \item None of the proof assistants investigated in this report lets users modify proof systems arbitrarily in a user-friendly manner. Among them, Carnap supports the widest range of proof systems, including natural deduction for first-order logic \cite{carnap:systems}, the Sequent Calculus \cite{carnap:systems}, and possibly the simply typed $\lambda$-calculus\footnote{As of June 2025, the $\lambda$-calculus is not a pre-defined proof system in Carnap. However, Carnap-Core (the libraries which allow users to specify proof systems) supports proof systems with abstractions and applications \cite{carnap:2018}, so it should be able to support the simply typed $\lambda$-calculus.}. However, users must define proof systems via the core libraries written in Haskell and cannot modify proof systems on the fly.
\end{itemize}

This project presents a web-based educational proof assistant that addresses both issues:
\begin{itemize}
    \item Users can build derivations entirely using \LaTeX{} syntax. Although none of the proof assistants investigated uses \LaTeX{}, knowledge of \LaTeX{} is useful and highly transferrable beyond the proof assistant.
    \item Users can modify proof systems on the fly without any understanding of the implementation details. All modifications can be done in the web application.
\end{itemize}

\section{Contributions}
The contribution of this project is a web application which has the following features:
\begin{itemize}
    \item Users can define syntax rules (\Cref{chapter:syntax}) and inference rules (\Cref{chapter:inference}). The application supports \LaTeX{} syntax and several shorthand notations.
    \item Users can build derivations in the web application. The application verifies whether the derivation is correct according to the syntax and inference rules defined by the user (\Cref{chapter:checking}). Prior to verification, both the inference rules (\Cref{chapter:inference}) and the derivation inputted by the user (\Cref{chapter:term}) are transformed into intermediate internal representations.
\end{itemize}