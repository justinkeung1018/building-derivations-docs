\chapter{Introduction}
At Imperial, all first-year undergraduate students in Computing and Joint Mathematics and Computing are required to study propositional logic and first-order logic. Students are introduced to Pandora \cite{pandora:2007, pandora}, an interactive tool developed by former Imperial undergraduates for constructing Fitch-style natural deduction proofs. Interactive tools like Pandora can facilitate learning in numerous ways:

\paragraph{Instant feedback}
Interactive tools provide instant feedback to students on the correctness of their proofs and derivations. Although lecturers can provide model answers to sample problems, there are often multiple correct solutions that would be tedious to exhaustively enumerate. Interactive tools ease the burden of teaching staff on verifying students' solutions. This is not to say they are unimportant in facilitating students' learning---they can help explain \textit{why} a proof or derivation is incorrect using various approaches \cite{nipkow:2012}; while interactive tools can also provide explanations to a certain degree, they are not necessarily tailored to the student's understanding.

\paragraph{More practice problems}
In the Department of Computing at Imperial, students are not provided with mark schemes or model answers to past papers. Although students have access to past paper solutions crowdsourced from seniors and former students, the solutions may not be complete or correct, especially for older papers. Students can use interactive tools to check their solutions, even in the lack of model answers. This extends beyond past papers at Imperial. Arbitrary new problems in natural deduction can be generated \cite{ahmed:2013} for as much practice as students want.

\paragraph{Enforcing methodical approaches}
Although students at Imperial who used Pandora scored similarly in the natural deduction parts of the final exam as those who did not use Pandora, the former approached the problems more methodically, were less likely to make arbitrary assumptions, and were more precise in their proofs \cite{pandora:2007}. This may be because Pandora makes it somewhat tedious to undo large parts of the proof by design. Even after adding the functionality to undo multiple proof steps as proposed in \cite{pandora:2007}, it is still more difficult than crossing out an erroneous solution on paper with a pen. This friction may prompt students to think more carefully before proceeding with the next proof step in Pandora, and with enough practice, on paper as well.

\mbox{}\\
As beneficial as these interactive tools can be, there are relatively few tools for formal derivation systems beyond logic, such as type systems. This may be because most computer science students around the world are required to take courses on logic but not type systems\todo{citation needed}. At Imperial, \textit{Type Systems for Programming Languages} (TSfPL) is an elective module offered from the third year onwards.

\section{Objectives}
This project aims to develop an interactive, web-based learning tool for building derivation trees. It should be able to verify the correctness of the derivation tree and give hints on how to fix a derivation tree when it is malformed. The tool should also support user-defined inference rules.

We will define a syntax for inference rules in extended Backus-Naur form (EBNF). Next, we will build a parser implementing the syntactic rules using parser combinator libraries. We will then define the semantics of the rules and ``compile'' the rules to verifiers that check whether the rule is applied correctly in a derivation tree. The correctness of our implementation will be checked against the type assignment systems in the Lambda Calculus and Milner's \textsc{ml}, as well as Gentzen's sequent calculus.

% \section{Challenges}
% \section{Contributions}