\section{Lambda Calculus}
In the 1930s, Alonzo Church introduced the Lambda Calculus \cite{church:1936} as a model of computation that is Turing-complete \cite{turing:1937}. It is the basis of functional programming languages like Haskell, which all Computing students at Imperial are required to learn. In addition, the Lambda Calculus is taught as part of the mandatory second-year \textit{Models of Computation} module and the TSfPL elective. In this section, we will define $\lambda$-terms and the Curry type assignment system, then formulate alternative syntaxes in extended Backus-Naur form (EBNF) for constructing the relevant parsers.

\subsection{\texorpdfstring{$\lambda$}{Lambda}-terms}\label{lambda:lambda-terms}
$\lambda$-terms are defined as follows \cite{church:1941}:
\[
    M,N \Coloneqq x \alt \underbracket[0.6pt]{(\lambda x. M)}_\text{abstraction} \alt \underbracket[0.6pt]{(MN)}_\text{application}
\]
where $x$ can be any symbol from an infinite list of term variables $a, b, c, \ldots, x, y, z \ldots$. Observe that abstractions and applications contain brackets to avoid ambiguity. We can alternatively express the syntax of $\lambda$-terms in Backus-Naur form (BNF):
\begin{align*}
    \nonterm{term} &\Coloneqq \nonterm{var} \alt \nonterm{abstraction} \alt \nonterm{application} \\
    \nonterm{var} &\Coloneqq \textit{any lowercase letter} \\
    \nonterm{abstraction} &\Coloneqq \term{(} \lambda \nonterm{var}. \nonterm{term} \term{)} \\
    \nonterm{application} &\Coloneqq \term{(} \nonterm{term} \nonterm{term} \term{)}
\end{align*}
While it is straightforward to parse $\lambda$-terms using this syntax, it may become inconvenient to write large $\lambda$-terms by hand, which is an important part of a student's learning experience. Therefore, we introduce the following conventions:
\begin{itemize}
    \item Applications are \textit{left-associative}, so $(MNP)$ is equivalent to $((MN)P)$.
    \item Outermost brackets can be dropped, so $\lambda x. x$ is equivalent to $(\lambda x. x)$.
    \item Repeated abstractions can be shortened, so $\lambda xyz. M$ is equivalent to $(\lambda x. (\lambda y. (\lambda z. M)))$.
\end{itemize}
We allow partial adherence to these conventions as students inexperienced with the $\lambda$ calculus may find additional brackets instructive, e.g. $\lambda x. ((\lambda x. x) y)$ drops the outermost brackets but not the brackets around the body of the abstraction. With these conventions in mind, we formulate a new syntax in Backus-Naur form:
\begin{align*}
    \nonterm{term} &\Coloneqq \nonterm{var} \alt \nonterm{unbracketed} \alt \nonterm{bracketed} \\
    \nonterm{var}  &\Coloneqq \textit{any lowercase letter} \\
    \nonterm{abstraction} &\Coloneqq \lambda \nonterm{var}. \nonterm{term} \\
    \nonterm{application} &\Coloneqq (\nonterm{application} \alt \nonterm{arg}) \nonterm{arg} \\
    \nonterm{arg} &\Coloneqq \nonterm{var} \alt \nonterm{bracketed} \\
    \nonterm{unbracketed} &\Coloneqq \nonterm{abstraction} \alt \nonterm{application} \\
    \nonterm{bracketed} &\Coloneqq \term{(} \nonterm{unbracketed} \term{)}
\end{align*}
Notice that the rule for $\nonterm{application}$ is left-recursive. This may cause infinite recursion in recursive descent parsers, which is certainly undesirable in a web application. We mitigate this by rewriting $\nonterm{application}$ in EBNF. The complete syntax of $\lambda$-terms with optional bracketing conventions in EBNF is presented here---the only change is made to the rule for $\nonterm{application}$.
\begin{align*}
    \nonterm{term} &\Coloneqq \nonterm{var} \alt \nonterm{unbracketed} \alt \nonterm{bracketed} \\
    \nonterm{var}  &\Coloneqq \textit{any lowercase letter} \\
    \nonterm{abstraction} &\Coloneqq \lambda \nonterm{var}. \nonterm{term} \\
    \nonterm{application} &\Coloneqq \nonterm{arg} \nonterm{arg} \{\nonterm{arg}\} \\
    \nonterm{arg} &\Coloneqq \nonterm{var} \alt \nonterm{bracketed} \\
    \nonterm{unbracketed} &\Coloneqq \nonterm{abstraction} \alt \nonterm{application} \\
    \nonterm{bracketed} &\Coloneqq \term{(} \nonterm{unbracketed} \term{)}
\end{align*}
Left-associativity of $\nonterm{application}$ will be enforced in the generation of the abstract syntax tree (AST).

\subsection{Curry types}\label{lambda:curry-types}
The set of Curry \textit{types} is defined as follows \cite{van-bakel:2022}:
\[
    A, B \Coloneqq \varphi \alt (A \rightarrow B)
\]
where $\varphi$ can be any symbol from an infinite list of type variables $\varphi_1, \varphi_2, \ldots$. When writing type variables by hand, it is often more convenient to use the subscript alone to represent a type variable, e.g. the type $((1 \rightarrow 2) \rightarrow 1)$ represents the type $((\varphi_1 \rightarrow \varphi_2) \rightarrow \varphi_1)$. We can express the syntax of Curry types in BNF:
\begin{align*}
    \nonterm{type} &\Coloneqq \nonterm{typevar} \alt \term{(} \nonterm{type} \rightarrow \nonterm{type} \term{)} \\
    \nonterm{typevar} &\Coloneqq \varphi_{\textit{any positive integer}}
\end{align*}
As with $\lambda$-terms, bracketing conventions are adopted for Curry types:
\begin{itemize}
    \item Arrow types are \textit{right-associative}, so $(1 \rightarrow 2 \rightarrow 1)$ is equivalent to $(1 \rightarrow (2 \rightarrow 1))$.
    \item Outermost brackets can be dropped, so $1 \rightarrow 2$ is equivalent to $(1 \rightarrow 2)$.
\end{itemize}
We reformulate the syntax of Curry types in BNF with optional bracketing conventions:
\begin{align*}
    \nonterm{type} &\Coloneqq \nonterm{typevar} \alt \nonterm{arrow} \\
    \nonterm{arrow} &\Coloneqq \nonterm{arg} \rightarrow \nonterm{arrow} \\
    \nonterm{arg} &\Coloneqq \nonterm{typevar} \alt \term{(} \nonterm{arrow} \term{)}
\end{align*}
We can rewrite the rule for $\nonterm{arrow}$ in EBNF as follows:
\begin{align*}
    \nonterm{type} &\Coloneqq \nonterm{typevar} \alt \nonterm{arrow} \\
    \nonterm{arrow} &\Coloneqq \nonterm{arg} \rightarrow \nonterm{arg} \{\rightarrow \nonterm{arg}\} \\
    \nonterm{arg} &\Coloneqq \nonterm{typevar} \alt \term{(} \nonterm{arrow} \term{)}
\end{align*}
As with $\lambda$-terms, the EBNF rule for $\nonterm{arrow}$ does not capture right-associativity, which can be handled when generating the AST.

\subsection{Type inference rules}\label{lambda:type-assignment}
$\lambda$-terms can be assigned types under Curry's type assignment system using the following derivation rules \cite{van-bakel:2022}:
\[
    (Ax): \frac{}{\Gamma, x:A \vdash x:A} \quad (\rightarrow I): \frac{\Gamma, x:A \vdash M:B}{\Gamma \vdash \lambda x. M: A \rightarrow B} \quad (\rightarrow E): \frac{\Gamma \vdash M: A \rightarrow B \quad \Gamma \vdash N: A}{\Gamma \vdash MN: A \rightarrow B}
\]
A \textit{context} $\Gamma$ is a set of statements in the form $x:A$, where $x$ is a variable and $A$ is a Curry type. All variables are assigned at most one type in any context, so $x:1, x:2$ is not a well-formed context since $x$ appears twice.

The syntax of a \textit{conclusion}\todo{Check with Steffen on terminology} in the form of $\Gamma \vdash M: A$ can be expressed in EBNF as follows:
\begin{align*}
    \nonterm{conclusion} &\Coloneqq \nonterm{context} \vdash \nonterm{term} : \nonterm{type} \\
    \nonterm{context} &\Coloneqq \varnothing \alt \nonterm{varassignment} \{, \nonterm{varassignment} \} \\
    \nonterm{varassignment} &\Coloneqq \nonterm{var} : \nonterm{type}
\end{align*}
where $\nonterm{term}$, $\nonterm{type}$, and $\nonterm{var}$ are taken from the syntax of $\lambda$-terms and Curry types.