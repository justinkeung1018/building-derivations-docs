\chapter{Background}
This chapter provides an overview of the background knowledge necessary to understand the project.

\section{Lambda Calculus}
In the 1930s, Alonzo Church introduced the Lambda Calculus \cite{church:1936} as a model of computation that is Turing-complete \cite{turing:1937}. It is the basis of functional programming languages like Haskell, which all Computing students at Imperial are required to learn. In addition, the Lambda Calculus is taught as part of the mandatory second-year \textit{Models of Computation} module and the TSfPL elective. In this section, we will define $\lambda$-terms and the Curry type assignment system, then formulate alternative syntaxes in extended Backus-Naur form (EBNF) for constructing the relevant parsers.

\subsection{\(\lambda\)-terms}\label{lambda:lambda-terms}
\textit{$\lambda$-terms} are defined as follows \cite{church:1941}:
\[
    M,N \Coloneqq x \alt \underbracket[0.6pt]{(\lambda x. M)}_\text{abstraction} \alt \underbracket[0.6pt]{(MN)}_\text{application}
\]
where $x$ can be any symbol from an infinite list of term variables $a, b, c, \ldots, x, y, z \ldots$. Observe that abstractions and applications contain brackets to avoid ambiguity. We can alternatively express the syntax of $\lambda$-terms in Backus-Naur form (BNF):
\begin{align*}
    \nonterm{term} &\Coloneqq \nonterm{var} \alt \nonterm{abstraction} \alt \nonterm{application} \\
    \nonterm{var} &\Coloneqq \textit{any lowercase letter} \\
    \nonterm{abstraction} &\Coloneqq \term{(} \lambda \nonterm{var}. \nonterm{term} \term{)} \\
    \nonterm{application} &\Coloneqq \term{(} \nonterm{term} \nonterm{term} \term{)}
\end{align*}
While it is straightforward to parse $\lambda$-terms using this syntax, it may become inconvenient to write large $\lambda$-terms by hand, which is an important part of a student's learning experience. Therefore, we introduce the following conventions:
\begin{itemize}
    \item Applications are \textit{left-associative}, so $(MNP)$ is equivalent to $((MN)P)$.
    \item Outermost brackets can be dropped, so $\lambda x. x$ is equivalent to $(\lambda x. x)$.
    \item Repeated abstractions can be shortened, so $\lambda xyz. M$ is equivalent to $(\lambda x. (\lambda y. (\lambda z. M)))$.
\end{itemize}
We allow partial adherence to these conventions as students inexperienced with the $\lambda$ calculus may find additional brackets instructive, e.g. $\lambda x. ((\lambda x. x) y)$ drops the outermost brackets but not the brackets around the body of the abstraction. With these conventions in mind, we formulate a new syntax in Backus-Naur form:
\begin{align*}
    \nonterm{term} &\Coloneqq \nonterm{var} \alt \nonterm{unbracketed} \alt \nonterm{bracketed} \\
    \nonterm{var}  &\Coloneqq \textit{any lowercase letter} \\
    \nonterm{abstraction} &\Coloneqq \lambda \nonterm{var}. \nonterm{term} \\
    \nonterm{application} &\Coloneqq (\nonterm{application} \alt \nonterm{arg}) \nonterm{arg} \\
    \nonterm{arg} &\Coloneqq \nonterm{var} \alt \nonterm{bracketed} \\
    \nonterm{unbracketed} &\Coloneqq \nonterm{abstraction} \alt \nonterm{application} \\
    \nonterm{bracketed} &\Coloneqq \term{(} \nonterm{unbracketed} \term{)}
\end{align*}
Notice that the rule for $\nonterm{application}$ is left-recursive. This may cause infinite recursion in recursive descent parsers, which is certainly undesirable in a web application. We mitigate this by rewriting $\nonterm{application}$ in EBNF. The complete syntax of $\lambda$-terms with optional bracketing conventions in EBNF is presented here---the only change is made to the rule for $\nonterm{application}$.
\begin{align*}
    \nonterm{term} &\Coloneqq \nonterm{var} \alt \nonterm{unbracketed} \alt \nonterm{bracketed} \\
    \nonterm{var}  &\Coloneqq \textit{any lowercase letter} \\
    \nonterm{abstraction} &\Coloneqq \lambda \nonterm{var}. \nonterm{term} \\
    \nonterm{application} &\Coloneqq \nonterm{arg} \nonterm{arg} \{\nonterm{arg}\} \\
    \nonterm{arg} &\Coloneqq \nonterm{var} \alt \nonterm{bracketed} \\
    \nonterm{unbracketed} &\Coloneqq \nonterm{abstraction} \alt \nonterm{application} \\
    \nonterm{bracketed} &\Coloneqq \term{(} \nonterm{unbracketed} \term{)}
\end{align*}
Left-associativity of $\nonterm{application}$ will be enforced in the generation of the abstract syntax tree (AST).

\subsection{Curry types}\label{lambda:curry-types}
The set of Curry \textit{types} is defined as follows \cite{van-bakel:2022}:
\[
    A, B \Coloneqq \varphi \alt (A \rightarrow B)
\]
where $\varphi$ can be any symbol from an infinite list of type variables $\varphi_1, \varphi_2, \ldots$. When writing type variables by hand, it is often more convenient to use the subscript alone to represent a type variable, e.g. the type $((1 \rightarrow 2) \rightarrow 1)$ represents the type $((\varphi_1 \rightarrow \varphi_2) \rightarrow \varphi_1)$. We can express the syntax of Curry types in BNF:
\begin{align*}
    \nonterm{type} &\Coloneqq \nonterm{typevar} \alt \term{(} \nonterm{type} \rightarrow \nonterm{type} \term{)} \\
    \nonterm{typevar} &\Coloneqq \varphi_{\textit{any positive integer}}
\end{align*}
As with $\lambda$-terms, bracketing conventions are adopted for Curry types:
\begin{itemize}
    \item Arrow types are \textit{right-associative}, so $(1 \rightarrow 2 \rightarrow 1)$ is equivalent to $(1 \rightarrow (2 \rightarrow 1))$.
    \item Outermost brackets can be dropped, so $1 \rightarrow 2$ is equivalent to $(1 \rightarrow 2)$.
\end{itemize}
We reformulate the syntax of Curry types in BNF with optional bracketing conventions:
\begin{align*}
    \nonterm{type} &\Coloneqq \nonterm{typevar} \alt \nonterm{arrow} \\
    \nonterm{arrow} &\Coloneqq \nonterm{arg} \rightarrow \nonterm{arrow} \\
    \nonterm{arg} &\Coloneqq \nonterm{typevar} \alt \term{(} \nonterm{arrow} \term{)}
\end{align*}
We can rewrite the rule for $\nonterm{arrow}$ in EBNF as follows:
\begin{align*}
    \nonterm{type} &\Coloneqq \nonterm{typevar} \alt \nonterm{arrow} \\
    \nonterm{arrow} &\Coloneqq \nonterm{arg} \rightarrow \nonterm{arg} \{\rightarrow \nonterm{arg}\} \\
    \nonterm{arg} &\Coloneqq \nonterm{typevar} \alt \term{(} \nonterm{arrow} \term{)}
\end{align*}
As with $\lambda$-terms, the EBNF rule for $\nonterm{arrow}$ does not capture right-associativity, which can be handled when generating the AST.

\subsection{Type inference rules}\label{lambda:type-assignment}
$\lambda$-terms can be assigned types under Curry's type assignment system using the following derivation rules \cite{van-bakel:2022}:
\[
    (Ax): \frac{}{\Gamma, x:A \vdash x:A} \quad (\rightarrow I): \frac{\Gamma, x:A \vdash M:B}{\Gamma \vdash \lambda x. M: A \rightarrow B} \quad (\rightarrow E): \frac{\Gamma \vdash M: A \rightarrow B \quad \Gamma \vdash N: A}{\Gamma \vdash MN: A \rightarrow B}
\]
A \textit{context} $\Gamma$ is a set of statements in the form $x:A$, where $x$ is a variable and $A$ is a Curry type. All variables are assigned at most one type in any context, so $x:1, x:2$ is not a well-formed context since $x$ appears twice.

The syntax of a \textit{conclusion}\todo{Check with Steffen on terminology} in the form of $\Gamma \vdash M: A$ can be expressed in EBNF as follows:
\begin{align*}
    \nonterm{conclusion} &\Coloneqq \nonterm{context} \vdash \nonterm{term} : \nonterm{type} \\
    \nonterm{context} &\Coloneqq \varnothing \alt \nonterm{varassignment} \{, \nonterm{varassignment} \} \\
    \nonterm{varassignment} &\Coloneqq \nonterm{var} : \nonterm{type}
\end{align*}
where $\nonterm{term}$, $\nonterm{type}$, and $\nonterm{var}$ are taken from the syntax of $\lambda$-terms and Curry types.
\section{Milner's \textsc{ML}}
% type systems module, what is covered

\subsection{ML expressions}
ML expressions are defined as follows \cite{van-bakel:2022}:
\[
    E \Coloneqq x \alt c \alt (\lambda x. E) \alt (E_1 E_2) \alt (\texttt{let}\ x = E_1 \ \texttt{in}\  E_2) \alt (\texttt{fix}\ g. E)
\]
where $x$ represents term variables as in \ref{lambda:lambda-terms}, and $c$ can be any constant. In BNF:
\begin{align*}
    \nonterm{expr} &\Coloneqq \nonterm{var} \alt \nonterm{const} \alt \nonterm{abstraction} \alt \nonterm{application} \alt \nonterm{let} \alt \nonterm{fix} \\
    \nonterm{const} &\Coloneqq \textit{any constant} \\
    \nonterm{let} &\Coloneqq \term{(} \texttt{let}\ \nonterm{var} = \nonterm{expr}\ \texttt{in}\ \nonterm{expr} \term{)} \\
    \nonterm{fix} &\Coloneqq \term{(} \texttt{fix}\ \nonterm{var}. \nonterm{expr} \term{)}
\end{align*}
Bracketing conventions are the same as in \ref{lambda:lambda-terms}. We can trivially extend the EBNF syntax with bracketing conventions in \ref{lambda:lambda-terms} with the additional \textsc{ML} term constructs. The complete EBNF syntax is as follows:
\begin{align*}
    \nonterm{term} &\Coloneqq \nonterm{var} \alt \nonterm{unbracketed} \alt \nonterm{bracketed} \\
    \nonterm{var}  &\Coloneqq \textit{any lowercase letter} \\
    \nonterm{const} &\Coloneqq \textit{any constant} \\
    \nonterm{abstraction} &\Coloneqq \lambda \nonterm{var}. \nonterm{term} \\
    \nonterm{application} &\Coloneqq \nonterm{arg} \nonterm{arg} \{\nonterm{arg}\} \\
    \nonterm{let} &\Coloneqq \texttt{let}\ \nonterm{var} = \nonterm{expr}\ \texttt{in}\ \nonterm{expr} \\
    \nonterm{fix} &\Coloneqq \texttt{fix}\ \nonterm{var}. \nonterm{expr} \\
    \nonterm{arg} &\Coloneqq \nonterm{var} \alt \nonterm{const} \alt \nonterm{bracketed} \\
    \nonterm{unbracketed} &\Coloneqq \nonterm{abstraction} \alt \nonterm{application} \alt \nonterm{let} \alt \nonterm{fix} \\
    \nonterm{bracketed} &\Coloneqq \term{(} \nonterm{unbracketed} \term{)}
\end{align*}

\subsection{\textsc{ML} types}
\textsc{ML} types are defined as follows \cite{van-bakel:2022}:
\begin{align*}
    \sigma, \tau &\Coloneqq A \alt (\forall \varphi. \tau) \\
    A, B &\Coloneqq \varphi \alt c \alt (A \rightarrow B)
\end{align*}
where $c$ can be any type constant, e.g. \texttt{Int} and \texttt{Bool}. We develop a BNF syntax for \textsc{ML} types.
\begin{align*}
    \nonterm{type} &\Coloneqq \nonterm{basic} \alt \nonterm{quantified} \\
    \nonterm{basic} &\Coloneqq \nonterm{typevar} \alt \nonterm{typeconst} \alt \nonterm{arrow} \\
    \nonterm{typevar} &\Coloneqq \varphi_{\textit{any positive integer}} \\
    \nonterm{typeconst} &\Coloneqq \textit{any type constant} \\
    \nonterm{arrow} &\Coloneqq \term{(} \nonterm{basic} \rightarrow \nonterm{basic} \term{)} \\
    \nonterm{quantified} &\Coloneqq \term{(} \forall \nonterm{typevar}. \nonterm{type} \term{)}
\end{align*}
In addition to the bracketing conventions in \ref{lambda:curry-types}, we abbreviate quantified types like $(\forall \varphi_1. (\forall \varphi_2. \cdots (\forall \varphi_n. A) \cdots ))$ as $\forall \Vec{\varphi}. A$, where $\Vec{\varphi}$ represents a vector of type variables $\varphi_1, \ldots, \varphi_n$. We can also only drop brackets but not vectorise the type variables, so the same quantified type can be written as $\forall \varphi_1. \forall \varphi_2. \cdots \forall \varphi_n. A$. In practice, students of TSfPL rarely encounter quantified types with more than one bound type variable (i.e. $n > 1$), though such support may be beneficial regardless. Here, we allow mixing of type variables and vectors of type variables, so $\forall \varphi_1. \forall \Vec{\varphi}. \forall \varphi_2. A$ is well-formed.

We formalise the (optional) bracketing and abbreviation conventions in EBNF as follows:
\begin{align*}
    \nonterm{type} &\Coloneqq \nonterm{basic} \alt \nonterm{quantified} \\
    \nonterm{basic} &\Coloneqq \nonterm{typevar} \alt \nonterm{typeconst} \alt \nonterm{arrow} \\
    \nonterm{typevar} &\Coloneqq \varphi_{\textit{any positive integer}} \\
    \nonterm{typeconst} &\Coloneqq \textit{any type constant} \\
    \nonterm{typevec} &\Coloneqq \Vec{\varphi}_{\textit{any positive integer}} \\
    \nonterm{arrow} &\Coloneqq \nonterm{arg} \rightarrow \nonterm{arg} \{\rightarrow \nonterm{arg}\} \\
    \nonterm{arg} &\Coloneqq \nonterm{typevar} \alt \nonterm{typeconst} \alt \term{(} \nonterm{arrow} \term{)} \\
    \nonterm{quantified} &\Coloneqq \forall (\nonterm{typevar} \alt \nonterm{typevec}). \{\forall (\nonterm{typevar} \alt \nonterm{typevec}).\} \nonterm{type}
\end{align*}

\subsection{Type inference rules}
The derivation rules for types in \textsc{ML} are as follows:
\begin{center}
    \begin{minipage}{.4\textwidth}
        \begin{align*}
            (Ax) &: \frac{}{\Gamma, x: \tau \vdash x: \tau} \\[1em]
            (\rightarrow I) &: \frac{\Gamma, x: A \vdash E: B}{\Gamma \vdash \lambda x. E: A \rightarrow B} \\[1em]
            (let) &: \frac{\Gamma \vdash E_1: \tau \quad \Gamma, x: \tau \vdash E_2: B}{\Gamma \vdash \texttt{let}\ x = E_1\ \texttt{in}\ E_2: B} \\[1em]
            (\forall I) &: \frac{\Gamma \vdash E: \tau}{\Gamma \vdash E: \forall \varphi. \tau} (\text{$\varphi$ not in $\Gamma$})
        \end{align*}
    \end{minipage}%
    \begin{minipage}{.4\textwidth}
        \begin{align*}
            (\mathcal{C}) &: \frac{}{\Gamma \vdash c: vc} \\[1em]
            (\rightarrow E) &: \frac{\Gamma \vdash E_1: A \rightarrow B \quad \Gamma \vdash E_2: A}{\Gamma \vdash E_1 E_2: B} \\[1em]
            (fix) &: \frac{\Gamma, g: A \vdash E: A}{\Gamma \vdash \texttt{fix}\ g. E: A} \\[1em]
            (\forall E) &: \frac{\Gamma \vdash E: \forall \varphi. \tau}{\Gamma \vdash E: \tau \{A / \varphi\}}
        \end{align*}
    \end{minipage}
\end{center}
The syntax of conclusions in \textsc{ML} is essentially identical to that of the Lambda Calculus, where the rules for $\lambda$-terms and Curry types are replaced by those for \textsc{ML} terms and \textsc{ML} types, respectively.

\section{What makes a learning tool enjoyable to use?}
There are relatively few interactive learning tools for type systems compared to logic and natural deduction. This may be because most computer science students around the world are required to take courses on logic but not type systems \todo{citation needed}. At Imperial, all first-year undergraduate students in Computing and Joint Mathematics and Computing are required to study propositional logic and first-order logic, while Type Systems for Programming Languages (TSfPL) is an elective module offered from the third year onwards. In this section, we will discuss several learning tools for natural deduction in propositional and first-order logic, focusing on the features that make them enjoyable or not enjoyable to use.

\subsection{Pandora}
Pandora \cite{pandora:2007} is a tool that helps students learn Fitch-style natural deduction. The current version \cite{pandora} is written in Java by former Imperial students for their undergraduate capstone projects. At Imperial, it is presented during lectures in the first-year logic module.

\paragraph{Unnatural user interactions}
Using detailed logs to keep track of clicks and other interactions with Pandora, \cite{pandora:2007} found that students made infrequent use of the help and tutorial functionalities, even though they often failed to apply the rules correctly. For example, many students did not select the necessary lines before applying a rule. We hypothesise that students make these frequent mistakes when using Pandora because the sequence of interactions for applying rules does not correspond to how they apply natural deduction rules when writing proofs by hand. Suppose a student wants to apply the $\rightarrow I$ rule to lines 1 and 2. The justification would look like $\rightarrow I(1, 2)$. It is natural to write it from left to right, starting from $\rightarrow I$, then perhaps one or both of the brackets, then writing the line number 1, and finally the line number 2. The natural translation of this sequence into Pandora interactions would be to first click on the $\rightarrow I$ rule, then click on line 1, and finally click on line 2. Clearly, there is a discrepancy between this sequence and the current design.

Therefore, we should design our tool such that the interactions required to build derivations mimic as closely as possible how students would be producing assessed or marked work. The more similar they are, the less assistance and explanation is needed for students to use the tool correctly.

\paragraph{Installation necessary}
Pandora is \textit{not} a web-based application. It can be run either as a JAR executable or using Java Web Start, a deprecated framework for starting Java applications using a web browser. The former starts up but fails to start a proof correctly and is essentially useless. The latter is not supported from Java 11 onwards and the installation of supported Java versions can be extremely tedious, especially on machines not using Intel-based architectures, which are prevalent today \todo{citation needed}.

The current version of Pandora is written over two decades ago \todo{fact check}. No matter how carefully we choose our tech stack, we cannot guarantee how long native applications will work until it fails to be compatible with future hardware, or be as tedious to install for students two decades later as Pandora is today. A simple workaround is to build a web application. It does not require installation other than a suitable web browser and the current versions of the canonical web technologies---HTML, CSS, and JavaScript---should not cease to be forwards-compatible any time soon. \todo{citation needed}

\subsection{Carnap}
Carnap \cite{carnap:2018} is an educational tool for a variety of formal reasoning systems. It is written in Haskell and can be transpiled to JavaScript to be run on web browsers.

\section{How do interactive tools help students learn?}
\paragraph{Instant feedback}
Interactive tools provide instant feedback to students on the correctness of their proofs and derivations. Although lecturers can provide model answers to sample problems, there are often multiple correct solutions that would be tedious to exhaustively enumerate. Interactive tools ease the burden of teaching staff on verifying students' solutions. This is not to say they are unimportant in facilitating students' learning---they can help explain \textit{why} a proof or derivation is incorrect using various approaches \cite{nipkow:2012}; while interactive tools can also provide explanations to a certain degree, they are not necessarily tailored to the student's understanding.

\paragraph{More practice problems}
In the Department of Computing at Imperial, students are not provided with mark schemes or model answers to past papers. Although students have access to past paper solutions crowdsourced from seniors and former students, the solutions may not be complete or correct, especially for older papers. Students can use interactive tools to check their solutions, even in the lack of model answers. This extends beyond past papers at Imperial. Arbitrary new problems in natural deduction can be generated \cite{ahmed:2013} for as much practice as students want.

\paragraph{Enforcing methodical approaches}
Although students at Imperial who used Pandora scored similarly in the natural deduction parts of the final exam as those who did not use Pandora, the former approached the problems more methodically, were less likely to make arbitrary assumptions, and were more precise in their proofs \cite{pandora:2007}. This may be because Pandora makes it somewhat tedious to undo large parts of the proof by design. Even after adding the functionality to undo multiple proof steps as proposed in \cite{pandora:2007}, it is still more difficult than crossing out an erroneous solution on paper with a pen. This friction may prompt students to think more carefully before proceeding with the next proof step in Pandora, and with enough practice, on paper as well.

\section{Parsing}
\label{background:parsing}
Parsing is the process of breaking up flat strings of symbols or tokens into a hierarchical structure which can be analysed more easily. \projectname{} must be able to parse syntax rules, inference rules, and the derivation tree. In \projectname{}, the user types the rule definitions and constructs the derivation tree using text inputs, which store values as strings. It is the job of the parsers to transform these string inputs into data structures which conveniently represent the information of the syntax rules, the inference rules, and the derivation tree, respectively. The data structures are described in more detail in the relevant sections (\Cref{section:syntax} for syntax rules, \Cref{section:term} for user input in the derivation tree, and \Cref{section:inference} for inference rules).

\subsection{Choice of tooling}
Ideally, all parsing in this project is done client-side. In other words, all computations related to parsing are performed on the user's machine, rather than on a server. Parsing is not computationally expensive enough to warrant the resources of an external server. The user can use the tool without an internet connection and will not experience unpredictable delays and latency issues due to web traffic conditions.

JavaScript \cite{javascript} is an obvious choice for writing client-side code to be run in browsers. In fact, the frontend of this project is written using React \cite{react}, a JavaScript-based library for web user interfaces. If the user's device can load the web application, it can also load any parsing-related code written in JavaScript. Most modern browsers like Google Chrome \cite{chrome}, Safari \cite{safari}, Microsoft Edge \cite{edge}, and Firefox \cite{firefox} support JavaScript. As of April 2025, these four browsers take up 92\% of the global market share for desktops \cite{statcounter}.

TypeScript \cite{typescript} is a statically typed version of JavaScript and is transpiled to JavaScript before being run on browsers. TypeScript code ``runs anywhere JavaScript runs'' \cite{typescript}.

However, other programming languages like C++ and Rust can also be run in modern web browsers by compiling to WebAssembly \cite{webassembly}. WebAssembly is an assembly-like language that can be run on most modern web browsers with ``near-native performance'' \cite{webassembly}. Functions written in languages like C++ and Rust can be compiled to WebAssembly and called in the frontend like any JavaScript function. WebAssembly is supported by browsers like Google Chrome, Safari, Microsoft Edge, and Firefox, and is available to 96\% of all browser users as of May 2025 \cite{webassembly:caniuse}.

A tool for transpiling TypeScript into JavaScript is needed regardless of the choice of the backend programming language, as the frontend is written in TypeScript. Therefore, writing the backend in TypeScript minimises the overhead from setting up additional tooling and maximises development time. The only motivation to use another language is if all parsing libraries written in TypeScript either do not have enough features or are too slow. Of course, one could also not use any parsing libraries and write parsers by hand, though it would be more time-consuming than using a library.

\projectname{} uses the \lstinline{parjs} parser combinator library \cite{parjs}, written in TypeScript. It can handle all parsing tasks in this project while the web application remains quick and responsive. The next SECTION provides an overview of the parts of the library necessary to understand the inner workings of this project.

\subsection{\texorpdfstring{\lstinline{parjs}}{parjs} overview}
A parser combinator is a function that combines one or more simpler parsers into a more complex parser. For example, a parser combinator may combine several parsers into a parser that applies them sequentially and collects the results of each parser into a list. This section introduces the \lstinline{then} and \lstinline{or} combinators of the \lstinline{parjs} library.

\subsubsection{Parsing failures}
Parsers in \lstinline{parjs} emit one of three types of failures \cite{parjs}:
\begin{itemize}
    \item \textit{Soft} failures allow parsers at a higher level to catch the failure and backtrack. Soft failures are used when parsing a sequence of alternatives using the \lstinline{or} combinator.
    \item \textit{Hard} failures cause parsing to halt immediately, unless a \lstinline{recover} combinator is explicitly used to catch the hard failure and downgrade it to a soft failure.
    \item \textit{Fatal} failures are explicitly generated by the user to tell the parser to halt and catch fire. Fatal failures are not generated by default by any of the parsers.
\end{itemize}

\subsubsection{The \lstinline{pipe} operator}
The \lstinline{pipe} operator \cite{parjs:pipe} is applied to a parser (the \textit{source parser}) and takes a sequence of combinators as its argument. It takes the output of the source parser and feeds it into the first combinator, then takes the output of the first combinator and feeds it into the second combinator, and so on. The final output is the output of the last combinator.

\subsubsection{The \lstinline{or} combinator}
The \lstinline{or} combinator \cite{parjs:or} is used for parsing a sequence of alternatives. The parser
\begin{center}
    \lstinline{string("sleeping").pipe(or(string("eating")), or(string("drinking")))}
\end{center}
successfully parses the strings ``sleeping'', ``eating'', and ``drinking'', but nothing else. When trying to parse ``eating'', the parser first tries to parse ``sleeping'' and fails softly: the parser \lstinline{string("sleeping")} emits a soft failure. As the first combinator is the \lstinline{or} combinator, the parser backtracks and proceeds to apply the parser \lstinline{string("eating")}, which succeeds. The successful parsing result is fed into the second \lstinline{or} combinator, which does not apply its argument parser \lstinline{string("drinking")} and simply returns the successful result from the previous combinator.

\subsubsection{The \lstinline{then} combinator}
The \lstinline{then} combinator \cite{parjs:then} is used for chaining multiple parsers sequentially. The parser
\begin{center}
    \lstinline{string("I").pipe(then(string(" study")), then(string(" computing")))}
\end{center}
successfully parses the string ``I study computing'' and nothing else. It is important to note that if a \lstinline{then} combinator is ``reached'' (i.e. all of its previous parsers have parsed the input successfully so far) and its argument fails, the \lstinline{then} combinator returns a failure no less severe than a hard failure. This means if its argument returns a soft failure, the \lstinline{then} combinator ``upgrades'' the soft failure and returns a hard failure.

The given parser fails softly when given the string ``go home'', since the first parser \lstinline{string("I")} fails softly and does not ``reach'' the \lstinline{then} combinator. However, the parser fails hard when given the string ``Imperial College London''. The first parser \lstinline{string("I")} succeeds, so it proceeds to apply the argument to the first \lstinline{then} combinator, which is \lstinline{string(" study")}. It returns a soft failure, which is ``upgraded'' by the \lstinline{then} combinator to a hard failure. The hard failure is propagated through the second \lstinline{then} combinator, causing the overall parser to return a hard failure.

\subsubsection{Using the \lstinline{or} and \lstinline{then} combinators together}
\label{parsing:thenor}
The behaviour of the \lstinline{then} combinator ``upgrading'' soft failures to hard failures makes it tricky to use the \lstinline{or} combinator with more complex parsers. Consider the following syntax definition:
\begin{align*}
    S &\Coloneqq Xa \alt Xb \\
    X &\Coloneqq x
\end{align*}
One might mimic the structure of the definitions and create parsers as follows:
\begin{lstlisting}
    const x = string("x");
    const first = x.pipe(then(string("a")));
    const second = x.pipe(then(string("b")));
    const s = first.pipe(or(second));
\end{lstlisting}
However, the parser \lstinline{s} only successfully parses the string ``xa'' but not ``xb''. In the latter case, the parser \lstinline{s} first tries to apply the parser \lstinline{first}. The parser \lstinline{first} tries to apply the parser \lstinline{x}, which succeeds. The parser \lstinline{first} then tries to apply the parser \lstinline{string("a")}, which fails softly. The \lstinline{then} combinator ``upgrades'' the soft failure to a hard failure, causing the \lstinline{first} parser to return a hard failure as well. The hard failure is propagated to \lstinline{or(second)}, which simply passes the hard failure along and causes the parser \lstinline{s} to return a hard failure as well.

According to the \lstinline{parjs} documentation, the idiomatic solution is to not create parsers by mimicking the structure of the definitions. When parsing definitions with multiple alternatives, the parser should first match the longest common prefix across all of the alternatives, then only apply the \lstinline{or} combinator when the subsequent parts of the alternatives are distinct \cite{parjs}. In this example, the parsers should be created like so:
\begin{lstlisting}
    const x = string("x");
    const s = x.pipe(or(string("a")), or(string("b")));
\end{lstlisting}
This corresponds to the \textit{left-factored} definitions
\begin{align*}
    S &\Coloneqq X(a|b) \\
    X &\Coloneqq x
\end{align*}
Left-factoring is the process of extracting common prefixes and rewriting alternatives such that no two alternatives share a leading unit of parsing. In this case, no two alternatives of the definition $X(a|b)$ begin with the same character, since $a$ is not equal to $b$. However, as later explained in \Cref{syntax:factorisation}, it is difficult to write a left-factoring algorithm that handles all possible cases with respect to the parsing tasks in this project.

A less idiomatic solution (indeed, advised against by the \lstinline{parjs} documentation \cite{parjs}) is to manually recover from the hard failures returned by the \lstinline{then} combinators using the \lstinline{recover} combinator:
\begin{lstlisting}
    ...
    const first = x.pipe(then(string("a")), recover(() => ({ kind: "Soft" })));
    const second = x.pipe(then(string("b")), recover(() => ({ kind: "Soft" })));
    ...
\end{lstlisting}
Here, the \lstinline{recover} combinator ``downgrades'' any failure, including hard and fatal failures, returned from the previous step to a soft failure. This solution is far easier to implement than the idiomatic solution and handles all possible user inputs.


