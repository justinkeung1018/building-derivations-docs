\chapter{Background}\label{background}
This chapter provides an overview of the background knowledge necessary to understand the project.

\section{Lambda Calculus}
In the 1930s, Alonzo Church introduced the Lambda Calculus \cite{church:1936} as a model of computation that is Turing-complete \cite{turing:1937}. It is the basis of functional programming languages like Haskell, which all Computing students at Imperial are required to learn. In addition, the Lambda Calculus is taught as part of the mandatory second-year \textit{Models of Computation} module and the TSfPL elective. In this section, we will define $\lambda$-terms and the Curry type assignment system, then formulate alternative syntaxes in extended Backus-Naur form (EBNF) for constructing the relevant parsers.

\subsection{\(\lambda\)-terms}\label{lambda:lambda-terms}
\textit{$\lambda$-terms} are defined as follows \cite{church:1941}:
\[
    M,N \Coloneqq x \alt \underbracket[0.6pt]{(\lambda x. M)}_\text{abstraction} \alt \underbracket[0.6pt]{(MN)}_\text{application}
\]
where $x$ can be any symbol from an infinite list of term variables $a, b, c, \ldots, x, y, z \ldots$. Observe that abstractions and applications contain brackets to avoid ambiguity. We can alternatively express the syntax of $\lambda$-terms in Backus-Naur form (BNF):
\begin{align*}
    \nonterm{term} &\Coloneqq \nonterm{var} \alt \nonterm{abstraction} \alt \nonterm{application} \\
    \nonterm{var} &\Coloneqq \textit{any lowercase letter} \\
    \nonterm{abstraction} &\Coloneqq \term{(} \lambda \nonterm{var}. \nonterm{term} \term{)} \\
    \nonterm{application} &\Coloneqq \term{(} \nonterm{term} \nonterm{term} \term{)}
\end{align*}
While it is straightforward to parse $\lambda$-terms using this syntax, it may become inconvenient to write large $\lambda$-terms by hand, which is an important part of a student's learning experience. Therefore, we introduce the following conventions:
\begin{itemize}
    \item Applications are \textit{left-associative}, so $(MNP)$ is equivalent to $((MN)P)$.
    \item Outermost brackets can be dropped, so $\lambda x. x$ is equivalent to $(\lambda x. x)$.
    \item Repeated abstractions can be shortened, so $\lambda xyz. M$ is equivalent to $(\lambda x. (\lambda y. (\lambda z. M)))$.
\end{itemize}
We allow partial adherence to these conventions as students inexperienced with the $\lambda$ calculus may find additional brackets instructive, e.g. $\lambda x. ((\lambda x. x) y)$ drops the outermost brackets but not the brackets around the body of the abstraction. With these conventions in mind, we formulate a new syntax in Backus-Naur form:
\begin{align*}
    \nonterm{term} &\Coloneqq \nonterm{var} \alt \nonterm{unbracketed} \alt \nonterm{bracketed} \\
    \nonterm{var}  &\Coloneqq \textit{any lowercase letter} \\
    \nonterm{abstraction} &\Coloneqq \lambda \nonterm{var}. \nonterm{term} \\
    \nonterm{application} &\Coloneqq (\nonterm{application} \alt \nonterm{arg}) \nonterm{arg} \\
    \nonterm{arg} &\Coloneqq \nonterm{var} \alt \nonterm{bracketed} \\
    \nonterm{unbracketed} &\Coloneqq \nonterm{abstraction} \alt \nonterm{application} \\
    \nonterm{bracketed} &\Coloneqq \term{(} \nonterm{unbracketed} \term{)}
\end{align*}
Notice that the rule for $\nonterm{application}$ is left-recursive. This may cause infinite recursion in recursive descent parsers, which is certainly undesirable in a web application. We mitigate this by rewriting $\nonterm{application}$ in EBNF. The complete syntax of $\lambda$-terms with optional bracketing conventions in EBNF is presented here---the only change is made to the rule for $\nonterm{application}$.
\begin{align*}
    \nonterm{term} &\Coloneqq \nonterm{var} \alt \nonterm{unbracketed} \alt \nonterm{bracketed} \\
    \nonterm{var}  &\Coloneqq \textit{any lowercase letter} \\
    \nonterm{abstraction} &\Coloneqq \lambda \nonterm{var}. \nonterm{term} \\
    \nonterm{application} &\Coloneqq \nonterm{arg} \nonterm{arg} \{\nonterm{arg}\} \\
    \nonterm{arg} &\Coloneqq \nonterm{var} \alt \nonterm{bracketed} \\
    \nonterm{unbracketed} &\Coloneqq \nonterm{abstraction} \alt \nonterm{application} \\
    \nonterm{bracketed} &\Coloneqq \term{(} \nonterm{unbracketed} \term{)}
\end{align*}
Left-associativity of $\nonterm{application}$ will be enforced in the generation of the abstract syntax tree (AST).

\subsection{Curry types}\label{lambda:curry-types}
The set of Curry \textit{types} is defined as follows \cite{van-bakel:2022}:
\[
    A, B \Coloneqq \varphi \alt (A \rightarrow B)
\]
where $\varphi$ can be any symbol from an infinite list of type variables $\varphi_1, \varphi_2, \ldots$. When writing type variables by hand, it is often more convenient to use the subscript alone to represent a type variable, e.g. the type $((1 \rightarrow 2) \rightarrow 1)$ represents the type $((\varphi_1 \rightarrow \varphi_2) \rightarrow \varphi_1)$. We can express the syntax of Curry types in BNF:
\begin{align*}
    \nonterm{type} &\Coloneqq \nonterm{typevar} \alt \term{(} \nonterm{type} \rightarrow \nonterm{type} \term{)} \\
    \nonterm{typevar} &\Coloneqq \varphi_{\textit{any positive integer}}
\end{align*}
As with $\lambda$-terms, bracketing conventions are adopted for Curry types:
\begin{itemize}
    \item Arrow types are \textit{right-associative}, so $(1 \rightarrow 2 \rightarrow 1)$ is equivalent to $(1 \rightarrow (2 \rightarrow 1))$.
    \item Outermost brackets can be dropped, so $1 \rightarrow 2$ is equivalent to $(1 \rightarrow 2)$.
\end{itemize}
We reformulate the syntax of Curry types in BNF with optional bracketing conventions:
\begin{align*}
    \nonterm{type} &\Coloneqq \nonterm{typevar} \alt \nonterm{arrow} \\
    \nonterm{arrow} &\Coloneqq \nonterm{arg} \rightarrow \nonterm{arrow} \\
    \nonterm{arg} &\Coloneqq \nonterm{typevar} \alt \term{(} \nonterm{arrow} \term{)}
\end{align*}
We can rewrite the rule for $\nonterm{arrow}$ in EBNF as follows:
\begin{align*}
    \nonterm{type} &\Coloneqq \nonterm{typevar} \alt \nonterm{arrow} \\
    \nonterm{arrow} &\Coloneqq \nonterm{arg} \rightarrow \nonterm{arg} \{\rightarrow \nonterm{arg}\} \\
    \nonterm{arg} &\Coloneqq \nonterm{typevar} \alt \term{(} \nonterm{arrow} \term{)}
\end{align*}
As with $\lambda$-terms, the EBNF rule for $\nonterm{arrow}$ does not capture right-associativity, which can be handled when generating the AST.

\subsection{Type inference rules}\label{lambda:type-assignment}
$\lambda$-terms can be assigned types under Curry's type assignment system using the following derivation rules \cite{van-bakel:2022}:
\[
    (Ax): \frac{}{\Gamma, x:A \vdash x:A} \quad (\rightarrow I): \frac{\Gamma, x:A \vdash M:B}{\Gamma \vdash \lambda x. M: A \rightarrow B} \quad (\rightarrow E): \frac{\Gamma \vdash M: A \rightarrow B \quad \Gamma \vdash N: A}{\Gamma \vdash MN: A \rightarrow B}
\]
A \textit{context} $\Gamma$ is a set of statements in the form $x:A$, where $x$ is a variable and $A$ is a Curry type. All variables are assigned at most one type in any context, so $x:1, x:2$ is not a well-formed context since $x$ appears twice.

The syntax of a \textit{conclusion}\todo{Check with Steffen on terminology} in the form of $\Gamma \vdash M: A$ can be expressed in EBNF as follows:
\begin{align*}
    \nonterm{conclusion} &\Coloneqq \nonterm{context} \vdash \nonterm{term} : \nonterm{type} \\
    \nonterm{context} &\Coloneqq \varnothing \alt \nonterm{varassignment} \{, \nonterm{varassignment} \} \\
    \nonterm{varassignment} &\Coloneqq \nonterm{var} : \nonterm{type}
\end{align*}
where $\nonterm{term}$, $\nonterm{type}$, and $\nonterm{var}$ are taken from the syntax of $\lambda$-terms and Curry types.
\section{Milner's \textsc{ML}}
% type systems module, what is covered

\subsection{ML expressions}
ML expressions are defined as follows \cite{van-bakel:2022}:
\[
    E \Coloneqq x \alt c \alt (\lambda x. E) \alt (E_1 E_2) \alt (\texttt{let}\ x = E_1 \ \texttt{in}\  E_2) \alt (\texttt{fix}\ g. E)
\]
where $x$ represents term variables as in \ref{lambda:lambda-terms}, and $c$ can be any constant. In BNF:
\begin{align*}
    \nonterm{expr} &\Coloneqq \nonterm{var} \alt \nonterm{const} \alt \nonterm{abstraction} \alt \nonterm{application} \alt \nonterm{let} \alt \nonterm{fix} \\
    \nonterm{const} &\Coloneqq \textit{any constant} \\
    \nonterm{let} &\Coloneqq \term{(} \texttt{let}\ \nonterm{var} = \nonterm{expr}\ \texttt{in}\ \nonterm{expr} \term{)} \\
    \nonterm{fix} &\Coloneqq \term{(} \texttt{fix}\ \nonterm{var}. \nonterm{expr} \term{)}
\end{align*}
Bracketing conventions are the same as in \ref{lambda:lambda-terms}. We can trivially extend the EBNF syntax with bracketing conventions in \ref{lambda:lambda-terms} with the additional \textsc{ML} term constructs. The complete EBNF syntax is as follows:
\begin{align*}
    \nonterm{term} &\Coloneqq \nonterm{var} \alt \nonterm{unbracketed} \alt \nonterm{bracketed} \\
    \nonterm{var}  &\Coloneqq \textit{any lowercase letter} \\
    \nonterm{const} &\Coloneqq \textit{any constant} \\
    \nonterm{abstraction} &\Coloneqq \lambda \nonterm{var}. \nonterm{term} \\
    \nonterm{application} &\Coloneqq \nonterm{arg} \nonterm{arg} \{\nonterm{arg}\} \\
    \nonterm{let} &\Coloneqq \texttt{let}\ \nonterm{var} = \nonterm{expr}\ \texttt{in}\ \nonterm{expr} \\
    \nonterm{fix} &\Coloneqq \texttt{fix}\ \nonterm{var}. \nonterm{expr} \\
    \nonterm{arg} &\Coloneqq \nonterm{var} \alt \nonterm{const} \alt \nonterm{bracketed} \\
    \nonterm{unbracketed} &\Coloneqq \nonterm{abstraction} \alt \nonterm{application} \alt \nonterm{let} \alt \nonterm{fix} \\
    \nonterm{bracketed} &\Coloneqq \term{(} \nonterm{unbracketed} \term{)}
\end{align*}

\subsection{\textsc{ML} types}
\textsc{ML} types are defined as follows \cite{van-bakel:2022}:
\begin{align*}
    \sigma, \tau &\Coloneqq A \alt (\forall \varphi. \tau) \\
    A, B &\Coloneqq \varphi \alt c \alt (A \rightarrow B)
\end{align*}
where $c$ can be any type constant, e.g. \texttt{Int} and \texttt{Bool}. We develop a BNF syntax for \textsc{ML} types.
\begin{align*}
    \nonterm{type} &\Coloneqq \nonterm{basic} \alt \nonterm{quantified} \\
    \nonterm{basic} &\Coloneqq \nonterm{typevar} \alt \nonterm{typeconst} \alt \nonterm{arrow} \\
    \nonterm{typevar} &\Coloneqq \varphi_{\textit{any positive integer}} \\
    \nonterm{typeconst} &\Coloneqq \textit{any type constant} \\
    \nonterm{arrow} &\Coloneqq \term{(} \nonterm{basic} \rightarrow \nonterm{basic} \term{)} \\
    \nonterm{quantified} &\Coloneqq \term{(} \forall \nonterm{typevar}. \nonterm{type} \term{)}
\end{align*}
In addition to the bracketing conventions in \ref{lambda:curry-types}, we abbreviate quantified types like $(\forall \varphi_1. (\forall \varphi_2. \cdots (\forall \varphi_n. A) \cdots ))$ as $\forall \Vec{\varphi}. A$, where $\Vec{\varphi}$ represents a vector of type variables $\varphi_1, \ldots, \varphi_n$. We can also only drop brackets but not vectorise the type variables, so the same quantified type can be written as $\forall \varphi_1. \forall \varphi_2. \cdots \forall \varphi_n. A$. In practice, students of TSfPL rarely encounter quantified types with more than one bound type variable (i.e. $n > 1$), though such support may be beneficial regardless. Here, we allow mixing of type variables and vectors of type variables, so $\forall \varphi_1. \forall \Vec{\varphi}. \forall \varphi_2. A$ is well-formed.

We formalise the (optional) bracketing and abbreviation conventions in EBNF as follows:
\begin{align*}
    \nonterm{type} &\Coloneqq \nonterm{basic} \alt \nonterm{quantified} \\
    \nonterm{basic} &\Coloneqq \nonterm{typevar} \alt \nonterm{typeconst} \alt \nonterm{arrow} \\
    \nonterm{typevar} &\Coloneqq \varphi_{\textit{any positive integer}} \\
    \nonterm{typeconst} &\Coloneqq \textit{any type constant} \\
    \nonterm{typevec} &\Coloneqq \Vec{\varphi}_{\textit{any positive integer}} \\
    \nonterm{arrow} &\Coloneqq \nonterm{arg} \rightarrow \nonterm{arg} \{\rightarrow \nonterm{arg}\} \\
    \nonterm{arg} &\Coloneqq \nonterm{typevar} \alt \nonterm{typeconst} \alt \term{(} \nonterm{arrow} \term{)} \\
    \nonterm{quantified} &\Coloneqq \forall (\nonterm{typevar} \alt \nonterm{typevec}). \{\forall (\nonterm{typevar} \alt \nonterm{typevec}).\} \nonterm{type}
\end{align*}

\subsection{Type inference rules}
The derivation rules for types in \textsc{ML} are as follows:
\begin{center}
    \begin{minipage}{.4\textwidth}
        \begin{align*}
            (Ax) &: \frac{}{\Gamma, x: \tau \vdash x: \tau} \\[1em]
            (\rightarrow I) &: \frac{\Gamma, x: A \vdash E: B}{\Gamma \vdash \lambda x. E: A \rightarrow B} \\[1em]
            (let) &: \frac{\Gamma \vdash E_1: \tau \quad \Gamma, x: \tau \vdash E_2: B}{\Gamma \vdash \texttt{let}\ x = E_1\ \texttt{in}\ E_2: B} \\[1em]
            (\forall I) &: \frac{\Gamma \vdash E: \tau}{\Gamma \vdash E: \forall \varphi. \tau} (\text{$\varphi$ not in $\Gamma$})
        \end{align*}
    \end{minipage}%
    \begin{minipage}{.4\textwidth}
        \begin{align*}
            (\mathcal{C}) &: \frac{}{\Gamma \vdash c: vc} \\[1em]
            (\rightarrow E) &: \frac{\Gamma \vdash E_1: A \rightarrow B \quad \Gamma \vdash E_2: A}{\Gamma \vdash E_1 E_2: B} \\[1em]
            (fix) &: \frac{\Gamma, g: A \vdash E: A}{\Gamma \vdash \texttt{fix}\ g. E: A} \\[1em]
            (\forall E) &: \frac{\Gamma \vdash E: \forall \varphi. \tau}{\Gamma \vdash E: \tau \{A / \varphi\}}
        \end{align*}
    \end{minipage}
\end{center}
The syntax of conclusions in \textsc{ML} is essentially identical to that of the Lambda Calculus, where the rules for $\lambda$-terms and Curry types are replaced by those for \textsc{ML} terms and \textsc{ML} types, respectively.
\section{Sequent calculus}
In 1934, Gerhard Gentzen proposed the sequent calculus \textsc{lk}\cite{gentzen:1969}, which is sound and complete with respect to classical first-order logic.

\subsection{Why the sequent calculus?}
Derivations in the sequent calculus are similar to the type derivations in the Curry type assignment system and for Milner's \textsc{ml} in two ways. Firstly, they follow the same tree structure with premises at the top and the conclusion at the bottom of each node. Secondly, they can be constructed by a computer as long as the syntax of the inference rules is supplied. A computer does not need to ``understand'' the semantics of the rules to build a correct derivation.

The sequent calculus is not taught to first-year students at Imperial, in favour of Fitch-style natural deduction. Although the rules of natural deduction are originally formulated in the familiar tree-like notation \cite{gentzen:1969}, they are more challenging to verify because a conclusion does not necessarily only follow from its immediate premises (i.e. the statements immediately above the dividing line), but may depend on assumptions made in a nested subtree.

In short, adapting our tool to support the sequent calculus allows us to generalise our syntax of inference rules to classical logic without deviating much from the structure of the derivations we have seen so far.

\subsection{Inference rules in the system \textsc{lk}}
For simplicity, we will only consider the subset of the calculus for propositional logic and restrict any formulas to those in propositional logic.
\begin{center}
    \begin{minipage}{.4\textwidth}
        \begin{align*}
            (\land L_1) &: \frac{\Gamma, A \vdash \Delta}{\Gamma, A \land B \vdash \Delta} \\[1em]
            (\land L_2) &: \frac{\Gamma, B \vdash \Delta}{\Gamma, A \land B \vdash \Delta} \\[1em]
            (\lor L) &: \frac{\Gamma, A \vdash \Delta \quad \Gamma, B \vdash \Delta}{\Gamma, A \lor B \vdash \Delta} \\[1em]
            (\rightarrow L) &: \frac{\Gamma \vdash A, \Delta \quad \Sigma, B \vdash \Pi}{\Gamma, \Sigma, A \rightarrow B \vdash \Delta, \Pi} \\[1em]
            (\lnot L) &: \frac{\Gamma \vdash A, \Delta}{\Gamma, \lnot A \vdash \Delta}
        \end{align*}
    \end{minipage}%
    \begin{minipage}{.4\textwidth}
        \begin{align*}
            (\lor R_1) &: \frac{\Gamma \vdash A, \Delta}{\Gamma \vdash A \lor B, \Delta} \\[1em]
            (\lor R_2) &: \frac{\Gamma \vdash B, \Delta}{\Gamma \vdash A \lor B, \Delta} \\[1em]
            (\land R) &: \frac{\Gamma \vdash A, \Delta \quad \Gamma \vdash B, \Delta}{\Gamma \vdash A \land B, \Delta} \\[1em]
            (\rightarrow R) &: \frac{\Gamma, A \vdash B, \Delta}{\Gamma \vdash A \rightarrow B, \Delta} \\[1em]
            (\lnot R) &: \frac{\Gamma, A \vdash \Delta}{\Gamma \vdash \lnot A, \Delta}
        \end{align*}
    \end{minipage}
\end{center}
In the rules above, $\Gamma$, $\Delta$, $\Sigma$, and $\Pi$ represent a possibly empty, \textit{unordered} set of propositional formulas, while $A$ and $B$ represent a propositional formula.
% Explain semantics of the rules, and reiterate our system doesn't need to understand the semantics, as long as it recognises the commas as unordered sets

\section{What makes a learning tool enjoyable to use?}\label{background:enjoyable}
In this section, we will discuss several learning tools for natural deduction in propositional and first-order logic, focusing on the features that make them enjoyable or not enjoyable to use.

\subsection{Pandora}
Pandora \cite{pandora:2007} is a tool that helps students learn Fitch-style natural deduction. The current version \cite{pandora} is written in Java by former Imperial students for their undergraduate capstone projects. At Imperial, it is presented during lectures in the first-year logic module.

\subsubsection{Unnatural user interactions}
Using detailed logs to keep track of clicks and other interactions with Pandora, \cite{pandora:2007} found that students made infrequent use of the help and tutorial functionalities, even though they often failed to apply the rules correctly. For example, many students did not select the necessary lines before applying a rule. We hypothesise that students make these frequent mistakes when using Pandora because the sequence of interactions for applying rules does not correspond to how they apply natural deduction rules when writing proofs by hand. Suppose a student wants to apply the $\rightarrow I$ rule to lines 1 and 2. The justification would look like $\rightarrow I(1, 2)$. It is natural to write it from left to right, starting from $\rightarrow I$, then perhaps one or both of the brackets, then writing the line number 1, and finally the line number 2. The natural translation of this sequence into Pandora interactions would be to first click on the $\rightarrow I$ rule, then click on line 1, and finally click on line 2. Clearly, there is a discrepancy between this sequence and the current design.

Therefore, we should design our tool such that the interactions required to build derivations mimic as closely as possible how students would be producing assessed or marked work. The more similar they are, the less assistance and explanation is needed for students to use the tool correctly.

\subsubsection{Installation required}
Pandora is \textit{not} a web-based application. It can be run either as a JAR executable or using Java Web Start, a deprecated framework for starting Java applications using a web browser. The former starts up but fails to start a proof correctly and is essentially useless. The latter is not supported from Java 11 onwards \cite{oracle:2020}, and even with Java 8 installed, the application throws an error on startup saying it is ``unsigned'' on the author's machine. Whatever means needed to open a functional Pandora version, if possible at all, is simply too complicated.

The last update of Pandora was more than 15 years ago, according to the timestamp ``200810031955'' on the website \cite{pandora}. No matter how carefully we choose our tech stack, we cannot guarantee how long native applications will work until it fails to be compatible with future hardware, or be as tedious to install for students two decades later as Pandora is today. A simple workaround is to build a web application. It does not require installation other than a suitable web browser and the current versions of the canonical web technologies---HTML, CSS, and JavaScript---should not cease to be forwards-compatible any time soon.

\subsection{Carnap}
Carnap \cite{carnap, carnap:2018} is an educational tool for a variety of formal reasoning systems and used by over 35 universities globally \cite{carnap:about}. It is written in Haskell and can be transpiled to JavaScript to be run on web browsers. The Carnap Book \cite{carnap:book} is a free web-based textbook with interactive widgets for practice problems on topics such as truth tables, formation trees, and natural deduction.

\subsubsection{Clear layout of all syntactic rules}
 Although there are no syntax guides immediately surrounding the widgets in the textbook, the relevant syntax is introduced when a new widget first appears. This is natural when going through the textbook sequentially, but can be cumbersome for someone hoping to review exercises in later parts of the textbook but have forgotten the syntax. The input syntaxes for all formal systems supported by Carnap are enumerated on one webpage \cite{carnap:systems}, though it is somewhat unintuitive to access from the home page. It is possible that users may experience less friction if they can be quickly reminded of the input syntax in all situations.

\section{Parsing inputs}
For the minimal working example at the time of writing, the parser for conclusions in the Lambda Calculus (with syntax defined in \Cref{lambda:type-assignment}) is written using the TypeScript-based parser combinator library \texttt{ts-parsec} \cite{tsparsec}. It does not appear to give very informative error messages. If parsing errors prove useful for giving hints on how to fix a malformed derivation tree or other purposes, we may explore other parser combinator libraries written in JavaScript or TypeScript, such as \texttt{parjs} \cite{parjs}, or less ideally, libraries written in other languages.

Parser combinator libraries written in JavaScript or TypeScript are preferred as they can be bundled with the rest of the client-side code and served to the client. This means the web application would be run entirely on the client's machine, including parsing inputs. If libraries in other languages were used instead, we would need to make API requests over the internet during parsing. This not only limits how often we can parse inputs to make the application more responsive, but also may not work well on machines with poor internet connections.

Once the parser combinator library is finalised, we will include a description of what parser combinators are (illustrated with code snippets using the chosen library), weigh the pros and cons of the various libraries available, and explain the significance of error messages.

% Build parser using ts-parsec (use ts so everything runs on client-side)
% Allow a range of input formats (current: Haskell-like, future: latex, ascii?)
% Convert all input formats into latex

