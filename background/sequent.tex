\section{Sequent calculus}
In 1934, Gerhard Gentzen proposed the sequent calculus \textsc{lk}\cite{gentzen:1969}, which is sound and complete with respect to classical first-order logic.

\subsection{Why the sequent calculus?}
Derivations in the sequent calculus are similar to the type derivations in the Curry type assignment system and for Milner's \textsc{ml} in two ways. Firstly, they follow the same tree structure with premises at the top and the conclusion at the bottom of each node. Secondly, they can be constructed by a computer as long as the syntax of the inference rules is supplied. A computer does not need to ``understand'' the semantics of the rules to build a correct derivation.

The sequent calculus is not taught to first-year students at Imperial, in favour of Fitch-style natural deduction. Although the rules of natural deduction are originally formulated in the familiar tree-like notation \cite{gentzen:1969}, they are more challenging to verify because a conclusion does not necessarily only follow from its immediate premises (i.e. the statements immediately above the dividing line), but may depend on assumptions made in a nested subtree.

In short, adapting our tool to support the sequent calculus allows us to generalise our syntax of inference rules to classical logic without deviating much from the structure of the derivations we have seen so far.

\subsection{Inference rules in the system \textsc{lk}}
For simplicity, we will only consider the subset of the calculus for propositional logic and restrict any formulas to those in propositional logic.
\begin{center}
    \begin{minipage}{.4\textwidth}
        \begin{align*}
            (\land L_1) &: \frac{\Gamma, A \vdash \Delta}{\Gamma, A \land B \vdash \Delta} \\[1em]
            (\land L_2) &: \frac{\Gamma, B \vdash \Delta}{\Gamma, A \land B \vdash \Delta} \\[1em]
            (\lor L) &: \frac{\Gamma, A \vdash \Delta \quad \Gamma, B \vdash \Delta}{\Gamma, A \lor B \vdash \Delta} \\[1em]
            (\rightarrow L) &: \frac{\Gamma \vdash A, \Delta \quad \Sigma, B \vdash \Pi}{\Gamma, \Sigma, A \rightarrow B \vdash \Delta, \Pi} \\[1em]
            (\lnot L) &: \frac{\Gamma \vdash A, \Delta}{\Gamma, \lnot A \vdash \Delta}
        \end{align*}
    \end{minipage}%
    \begin{minipage}{.4\textwidth}
        \begin{align*}
            (\lor R_1) &: \frac{\Gamma \vdash A, \Delta}{\Gamma \vdash A \lor B, \Delta} \\[1em]
            (\lor R_2) &: \frac{\Gamma \vdash B, \Delta}{\Gamma \vdash A \lor B, \Delta} \\[1em]
            (\land R) &: \frac{\Gamma \vdash A, \Delta \quad \Gamma \vdash B, \Delta}{\Gamma \vdash A \land B, \Delta} \\[1em]
            (\rightarrow R) &: \frac{\Gamma, A \vdash B, \Delta}{\Gamma \vdash A \rightarrow B, \Delta} \\[1em]
            (\lnot R) &: \frac{\Gamma, A \vdash \Delta}{\Gamma \vdash \lnot A, \Delta}
        \end{align*}
    \end{minipage}
\end{center}
In the rules above, $\Gamma$, $\Delta$, $\Sigma$, and $\Pi$ represent a possibly empty, \textit{unordered} set of propositional formulas, while $A$ and $B$ represent a propositional formula.
% Explain semantics of the rules, and reiterate our system doesn't need to understand the semantics, as long as it recognises the commas as unordered sets