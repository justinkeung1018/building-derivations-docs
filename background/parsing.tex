\section{Parsing}
\label{background:parsing}
Parsing is the process of breaking up flat strings of symbols or tokens into a hierarchical structure which can be analysed more easily. \projectname{} must be able to parse syntax rules, inference rules, and the derivation tree. The user types the rule definitions and constructs the derivation tree using text inputs, which store values as strings. It is the job of the parsers to transform these string inputs into data structures which conveniently represent the information of the syntax rules, the inference rules, and the derivation tree, respectively. The data structures are described in more detail in the relevant sections (\Cref{section:syntax} for syntax rules, \Cref{section:term} for user input in the derivation tree, and \Cref{section:inference} for inference rules).

\subsection{Choice of tooling}
Ideally, all parsing in this project is done client-side In other words, all computations related to parsing are performed on the user's machine, rather than on a server. Parsing is not computationally expensive enough to warrant the resources of an external server. The user can use the tool without an internet connection and will not experience unpredictable delays and latency issues due to web traffic conditions.

JavaScript \cite{javascript} is an obvious choice for writing client-side code to be run in browsers. In fact, the frontend of this project is written using React \cite{react}, a JavaScript-based library for web user interfaces. If the user's device can load the web application, it can also load any parsing-related code written in JavaScript. Most modern browsers like Google Chrome \cite{chrome}, Safari \cite{safari}, Microsoft Edge \cite{edge}, and Firefox \cite{firefox} support JavaScript. As of April 2025, these four browsers take up 92\% of the global market share for desktops \cite{statcounter}.

TypeScript \cite{typescript} is a statically typed version of JavaScript and is transpiled to JavaScript before being run on browsers. TypeScript code ``runs anywhere JavaScript runs'' \cite{typescript}.

However, other programming languages like C++ and Rust can also be run in modern web browsers by compiling to WebAssembly \cite{webassembly}. WebAssembly is an assembly-like language that can be run on most modern web browsers with ``near-native performance'' \cite{webassembly}. Functions written in languages like C++ and Rust can be compiled to WebAssembly and called in the frontend like any JavaScript function. WebAssembly is supported by browsers like Google Chrome, Safari, Microsoft Edge, and Firefox, and is available to 96\% of all browser users as of May 2025 \cite{webassembly:caniuse}.

A tool for transpiling TypeScript into JavaScript is needed regardless of the choice of the backend programming language, as the frontend is written in TypeScript. Therefore, writing the backend in TypeScript minimises additional tooling setup overhead and maximises development time. The only motivation to use another language is if all parsing libraries written in TypeScript either do not have enough features or are too slow. Of course, one could also not use any parsing libraries and write parsers by hand, though it would be more time-consuming than using a library.

\projectname{} uses the \lstinline{parjs} parser combinator library \cite{parjs}, written in TypeScript. It can handle all parsing tasks in this project while the web application remains quick and responsive. The next subsection provides an overview of the parts of the library necessary to understand the inner workings of this project.

\subsection{\texorpdfstring{\lstinline{parjs}}{parjs} overview}
A parser combinator is a function that combines one or more simpler parsers into a more complex parser. For example, a parser combinator may combine several parsers into a parser that applies them sequentially and collects the results of each parser into a list. This subsection introduces the \lstinline{then} and \lstinline{or} combinators of the \lstinline{parjs} library.

\subsubsection{Parsing failures}
Parsers in \lstinline{parjs} emit one of three types of failures \cite{parjs}:
\begin{itemize}
    \item \textit{Soft} failures allow parsers at a higher level to catch the failure and backtrack. Soft failures are used when parsing a sequence of alternatives using the \lstinline{or} combinator.
    \item \textit{Hard} failures cause parsing to halt immediately, unless a \lstinline{recover} combinator is explicitly used to catch the hard failure and downgrade it to a soft failure.
    \item \textit{Fatal} failures are explicitly generated by the user to tell the parser to halt and catch fire. Fatal failures are not generated by default by any of the parsers.
\end{itemize}

\subsubsection{The \lstinline{pipe} operator}
The \lstinline{pipe} operator \cite{parjs:pipe} is applied to a parser (the \textit{source parser}) and takes a sequence of combinators as its argument. It takes the output of the source parser and feeds it into the first combinator, then takes the output of the first combinator and feeds it into the second combinator, and so on. The final output is the output of the last combinator.

\subsubsection{The \lstinline{or} combinator}
The \lstinline{or} combinator \cite{parjs:or} is used for parsing a sequence of alternatives. The parser
\begin{center}
    \lstinline{string("sleeping").pipe(or(string("eating")), or(string("drinking")))}
\end{center}
successfully parses the strings ``sleeping'', ``eating'', and ``drinking'', but nothing else. When trying to parse ``eating'', the parser first tries to parse ``sleeping'' and fails softly: the parser \lstinline{string("sleeping")} emits a soft failure. As the first combinator is the \lstinline{or} combinator, the parser backtracks and proceeds to apply the parser \lstinline{string("eating")}, which succeeds. The successful parsing result is fed into the second \lstinline{or} combinator, which does not apply its argument parser \lstinline{string("drinking")} and simply returns the successful result from the previous combinator.

\subsubsection{The \lstinline{then} combinator}
The \lstinline{then} combinator \cite{parjs:then} is used for chaining multiple parsers sequentially. The parser
\begin{center}
    \lstinline{string("I").pipe(then(string(" study")), then(string(" computing")))}
\end{center}
successfully parses the string ``I study computing'' and nothing else. It is important to note that if a \lstinline{then} combinator is ``reached'' (i.e. all of its previous parsers have parsed the input successfully so far) and its argument fails, the \lstinline{then} combinator returns a failure no less severe than a hard failure. This means if its argument returns a soft failure, the \lstinline{then} combinator ``upgrades'' the soft failure and returns a hard failure.

The given parser fails softly when given the string ``go home'', since the first parser \lstinline{string("I")} fails softly and does not ``reach'' the \lstinline{then} combinator. However, the parser fails hard when given the string ``Imperial College London''. The first parser \lstinline{string("I")} succeeds, so it proceeds to apply the argument to the first \lstinline{then} combinator, which is \lstinline{string(" study")}. It returns a soft failure, which is ``upgraded'' by the \lstinline{then} combinator to a hard failure. The hard failure is propagated through the second \lstinline{then} combinator, causing the overall parser to return a hard failure.

\subsubsection{Using the \lstinline{or} and \lstinline{then} combinators together}
\label{parsing:thenor}
The behaviour of the \lstinline{then} combinator ``upgrading'' soft failures to hard failures makes it tricky to use the \lstinline{or} combinator with more complex parsers. Consider the following syntax definition:
\begin{align*}
    S &\Coloneqq Xa \alt Xb \\
    X &\Coloneqq x
\end{align*}
One might mimic the structure of the definitions and create parsers as follows:
\begin{lstlisting}
    const x = string("x");
    const first = x.pipe(then(string("a")));
    const second = x.pipe(then(string("b")));
    const s = first.pipe(or(second));
\end{lstlisting}
However, the parser \lstinline{s} only successfully parses the string ``xa'' but not ``xb''. In the latter case, the parser \lstinline{s} first tries to apply the parser \lstinline{first}. The parser \lstinline{first} tries to apply the parser \lstinline{x}, which succeeds. The parser \lstinline{first} then tries to apply the parser \lstinline{string("a")}, which fails softly. The \lstinline{then} combinator ``upgrades'' the soft failure to a hard failure, causing the \lstinline{first} parser to return a hard failure as well. The hard failure is propagated to \lstinline{or(second)}, which simply passes the hard failure along and causes the parser \lstinline{s} to return a hard failure as well.

According to the \lstinline{parjs} documentation, the idiomatic solution is to not create parsers by mimicking the structure of the definitions. When parsing definitions with multiple alternatives, the parser should first match the longest common prefix across all of the alternatives, then only apply the \lstinline{or} combinator when the subsequent parts of the alternatives are distinct \cite{parjs}. In this example, the parsers should be created like so:
\begin{lstlisting}
    const x = string("x");
    const s = x.pipe(or(string("a")), or(string("b")));
\end{lstlisting}
This corresponds to the \textit{left-factored} definitions
\begin{align*}
    S &\Coloneqq X(a|b) \\
    X &\Coloneqq x
\end{align*}
Left-factoring is the process of extracting common prefixes and rewriting alternatives such that no two alternatives share a leading unit of parsing. In this case, no two alternatives of the definition $X(a|b)$ begin with the same character, since $a$ is not equal to $b$. However, as later explained in \Cref{syntax:factorisation}, it is difficult to write a left-factoring algorithm that handles all possible cases with respect to the parsing tasks in this project.

A less idiomatic solution (indeed, advised against by the \lstinline{parjs} documentation \cite{parjs}) is to manually recover from the hard failures returned by the \lstinline{then} combinators using the \lstinline{recover} combinator:
\begin{lstlisting}
    ...
    const first = x.pipe(then(string("a")), recover(() => ({ kind: "Soft" })));
    const second = x.pipe(then(string("b")), recover(() => ({ kind: "Soft" })));
    ...
\end{lstlisting}
Here, the \lstinline{recover} combinator ``downgrades'' any failure, including hard and fatal failures, returned from the previous step to a soft failure. This solution is far easier to implement than the idiomatic solution and handles all possible user inputs.
