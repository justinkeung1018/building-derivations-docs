\section{Milner's \textsc{ml}}
This section may be completely left out in the final report, depending on whether we intend to support Milner's \textsc{ml} as a built-in derivation system for our tool. Otherwise, we will draft an introduction to this section and include essentially the content presented below, perhaps with some examples to illustrate how to build type derivations for \textsc{ml} terms.

\subsection{\textsc{ml} expressions}
\textsc{ml} \textit{expressions} are defined as follows \cite{van-bakel:2022}:
\[
    E \Coloneqq x \alt c \alt (\lambda x. E) \alt (E_1 E_2) \alt (\texttt{let}\ x = E_1 \ \texttt{in}\  E_2) \alt (\texttt{fix}\ g. E)
\]
where $x$ represents term variables as in \ref{lambda:lambda-terms}, and $c$ can be any constant. In BNF:
\begin{align*}
    \nonterm{expr} &\Coloneqq \nonterm{var} \alt \nonterm{const} \alt \nonterm{abstraction} \alt \nonterm{application} \alt \nonterm{let} \alt \nonterm{fix} \\
    \nonterm{const} &\Coloneqq \textit{any constant} \\
    \nonterm{let} &\Coloneqq \term{(} \texttt{let}\ \nonterm{var} = \nonterm{expr}\ \texttt{in}\ \nonterm{expr} \term{)} \\
    \nonterm{fix} &\Coloneqq \term{(} \texttt{fix}\ \nonterm{var}. \nonterm{expr} \term{)}
\end{align*}
Bracketing conventions are the same as in \ref{lambda:lambda-terms}. We can trivially extend the EBNF syntax with bracketing conventions in \ref{lambda:lambda-terms} with the additional \textsc{ml} term constructs. The complete EBNF syntax is as follows:
\begin{align*}
    \nonterm{term} &\Coloneqq \nonterm{var} \alt \nonterm{unbracketed} \alt \nonterm{bracketed} \\
    \nonterm{var}  &\Coloneqq \textit{any lowercase letter} \\
    \nonterm{const} &\Coloneqq \textit{any constant} \\
    \nonterm{abstraction} &\Coloneqq \lambda \nonterm{var}. \nonterm{term} \\
    \nonterm{application} &\Coloneqq \nonterm{arg} \nonterm{arg} \{\nonterm{arg}\} \\
    \nonterm{let} &\Coloneqq \texttt{let}\ \nonterm{var} = \nonterm{expr}\ \texttt{in}\ \nonterm{expr} \\
    \nonterm{fix} &\Coloneqq \texttt{fix}\ \nonterm{var}. \nonterm{expr} \\
    \nonterm{arg} &\Coloneqq \nonterm{var} \alt \nonterm{const} \alt \nonterm{bracketed} \\
    \nonterm{unbracketed} &\Coloneqq \nonterm{abstraction} \alt \nonterm{application} \alt \nonterm{let} \alt \nonterm{fix} \\
    \nonterm{bracketed} &\Coloneqq \term{(} \nonterm{unbracketed} \term{)}
\end{align*}

\subsection{\textsc{ml} types}
\textsc{ml} \textit{types} are defined as follows \cite{van-bakel:2022}:
\begin{align*}
    \sigma, \tau &\Coloneqq A \alt (\forall \varphi. \tau) \\
    A, B &\Coloneqq \varphi \alt c \alt (A \rightarrow B)
\end{align*}
where $c$ can be any type constant, e.g. \texttt{Int} and \texttt{Bool}. We develop a BNF syntax for \textsc{ml} types.
\begin{align*}
    \nonterm{type} &\Coloneqq \nonterm{basic} \alt \nonterm{quantified} \\
    \nonterm{basic} &\Coloneqq \nonterm{typevar} \alt \nonterm{typeconst} \alt \nonterm{arrow} \\
    \nonterm{typevar} &\Coloneqq \varphi_{\textit{any positive integer}} \\
    \nonterm{typeconst} &\Coloneqq \textit{any type constant} \\
    \nonterm{arrow} &\Coloneqq \term{(} \nonterm{basic} \rightarrow \nonterm{basic} \term{)} \\
    \nonterm{quantified} &\Coloneqq \term{(} \forall \nonterm{typevar}. \nonterm{type} \term{)}
\end{align*}
In addition to the bracketing conventions in \ref{lambda:curry-types}, we abbreviate quantified types like
\[
    (\forall \varphi_1. (\forall \varphi_2. \cdots (\forall \varphi_n. A) \cdots ))
\]
as $\forall \Vec{\varphi}. A$, where $\Vec{\varphi}$ represents a vector of type variables $\varphi_1, \ldots, \varphi_n$. We can also only drop brackets but not vectorise the type variables, so the same quantified type can be written as $\forall \varphi_1. \forall \varphi_2. \cdots \forall \varphi_n. A$. In practice, students of TSfPL rarely encounter quantified types with more than one bound type variable (i.e. $n > 1$), though such support may be beneficial regardless. Here, we allow mixing of type variables and vectors of type variables, so $\forall \varphi_1. \forall \Vec{\varphi}. \forall \varphi_2. A$ is well-formed.

We formalise the (optional) bracketing and abbreviation conventions in EBNF as follows:
\begin{align*}
    \nonterm{type} &\Coloneqq \nonterm{basic} \alt \nonterm{quantified} \\
    \nonterm{basic} &\Coloneqq \nonterm{typevar} \alt \nonterm{typeconst} \alt \nonterm{arrow} \\
    \nonterm{typevar} &\Coloneqq \varphi_{\textit{any positive integer}} \\
    \nonterm{typeconst} &\Coloneqq \textit{any type constant} \\
    \nonterm{typevec} &\Coloneqq \Vec{\varphi}_{\textit{any positive integer}} \\
    \nonterm{arrow} &\Coloneqq \nonterm{arg} \rightarrow \nonterm{arg} \{\rightarrow \nonterm{arg}\} \\
    \nonterm{arg} &\Coloneqq \nonterm{typevar} \alt \nonterm{typeconst} \alt \term{(} \nonterm{arrow} \term{)} \\
    \nonterm{quantified} &\Coloneqq \forall (\nonterm{typevar} \alt \nonterm{typevec}). \{\forall (\nonterm{typevar} \alt \nonterm{typevec}).\} \nonterm{type}
\end{align*}

\subsection{Type inference rules}
The derivation rules for types in \textsc{ml} are as follows \cite{van-bakel:2022}:
\begin{center}
    \begin{minipage}{.4\textwidth}
        \begin{align*}
            (\mathit{Ax}) &: \frac{}{\Gamma, x: \tau \vdash x: \tau} \\[1em]
            (\rightarrow I) &: \frac{\Gamma, x: A \vdash E: B}{\Gamma \vdash \lambda x. E: A \rightarrow B} \\[1em]
            (\mathit{let}) &: \frac{\Gamma \vdash E_1: \tau \quad \Gamma, x: \tau \vdash E_2: B}{\Gamma \vdash \texttt{let}\ x = E_1\ \texttt{in}\ E_2: B} \\[1em]
            (\forall I) &: \frac{\Gamma \vdash E: \tau}{\Gamma \vdash E: \forall \varphi. \tau} (\text{$\varphi$ not in $\Gamma$})
        \end{align*}
    \end{minipage}%
    \begin{minipage}{.4\textwidth}
        \begin{align*}
            (\mathcal{C}) &: \frac{}{\Gamma \vdash c: vc} \\[1em]
            (\rightarrow E) &: \frac{\Gamma \vdash E_1: A \rightarrow B \quad \Gamma \vdash E_2: A}{\Gamma \vdash E_1 E_2: B} \\[1em]
            (\mathit{fix}) &: \frac{\Gamma, g: A \vdash E: A}{\Gamma \vdash \texttt{fix}\ g. E: A} \\[1em]
            (\forall E) &: \frac{\Gamma \vdash E: \forall \varphi. \tau}{\Gamma \vdash E: \tau \{A / \varphi\}}
        \end{align*}
    \end{minipage}
\end{center}
The syntax of conclusions in \textsc{ml} is essentially identical to that of the Lambda Calculus, where the rules for $\lambda$-terms and Curry types are replaced by those for \textsc{ml} terms and \textsc{ml} types, respectively.